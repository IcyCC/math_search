<html>
<head>
<title>LaTeX4Web 1.4 OUTPUT</title>
<style type="text/css">
<!--
 body {color: black;  background:"#FFCC99";  }
 div.p { margin-top: 7pt;}
 td div.comp { margin-top: -0.6ex; margin-bottom: -1ex;}
 td div.comb { margin-top: -0.6ex; margin-bottom: -.6ex;}
 td div.norm {line-height:normal;}
 td div.hrcomp { line-height: 0.9; margin-top: -0.8ex; margin-bottom: -1ex;}
 td.sqrt {border-top:2 solid black;
          border-left:2 solid black;
          border-bottom:none;
          border-right:none;}
 table.sqrt {border-top:2 solid black;
             border-left:2 solid black;
             border-bottom:none;
             border-right:none;}
-->
</style>
</head>
<body>
\documentclassarticle
\usepackagexeCJK
\usepackagetikz
\usepackageindentfirst
\setlength<br>indent2em
\usepackageenumitem
\usepackagelatexsym
\usepackagegraphicx
\usepackageindentfirst
\usepackageamsmath
	
\begindocument

	<h1>1.2 有理数</h1>
	<h2>1.2.1 有理数</h2>
	思考<br>

	<p>&nbsp;&nbsp;&nbsp;&nbsp; 回想一下,我们认识了那些数?<br>

	<p>&nbsp;&nbsp;&nbsp;&nbsp; 我们学过的数有:<br>

	<p>&nbsp;&nbsp;&nbsp;&nbsp; 正整数,如1,2,3,...;<br>

	<p>&nbsp;&nbsp;&nbsp;&nbsp; 零,0;<br>

	<p>&nbsp;&nbsp;&nbsp;&nbsp; 负整数,如<font face=symbol>-</font>1,<font face=symbol>-</font>2,<font face=symbol>-</font>3,...;<br>

	<p>&nbsp;&nbsp;&nbsp;&nbsp; 正分数,如(1)/(2),(2)/(3),(15)/(7),0.1,5.32,...;<br>

	<p>&nbsp;&nbsp;&nbsp;&nbsp; 负分数,如 <font face=symbol>-</font>0.5,<font face=symbol>-</font>(5)/(2),<font face=symbol>-</font>(2)/(3),<font face=symbol>-</font>(1)/(7),<font face=symbol>-</font>150.25,...;<br>

	<p>&nbsp;&nbsp;&nbsp;&nbsp; 正整数、0、负整数统称为正整数;正分数、负分数统称为分数.<br>

	\definition有理数
		\begindefinition
			整数和分数统称为有理数(rational number).<br>

		\enddefinition

	<p>&nbsp;&nbsp;&nbsp;&nbsp; 从小学开始,我们首先认识了正整数和负分数后,对数的认识就扩充到了有理数范围.<br>

	
	\exercise练习
		\beginexercise
			1.所有正整数组成正数集合,所有负数组成负数集合,把下面的有理数填入它属于的集合的范围内:<br>

			<p>&nbsp;&nbsp;&nbsp;&nbsp; 15,<font face=symbol>-</font>(1)/(9),<font face=symbol>-</font>5,(2)/(15),<font face=symbol>-</font>(13)/(8),0.1,<font face=symbol>-</font>5.32,<font face=symbol>-</font>80,123,2.333.<br>

			
			\begintikzpicture
			\draw (0,0) ellipse (1cm and 0.5cm);
			\draw (8,0) ellipse (1cm and 0.5cm);
			\endtikzpicture
			
			2.指出下列各数中的正数、负数、整数、分数:<br>

			<p>&nbsp;&nbsp;&nbsp;&nbsp; <font face=symbol>-</font>15,+6,<font face=symbol>-</font>2,<font face=symbol>-</font>0.9,1,(3)/(5),0,(1)/(4),0.63,<font face=symbol>-</font>4.95.<br>

		\endexercise
	
	<h2>1.2.2 数轴</h2>
	问题 在一条东西向的马路上,有一个汽车站牌,汽车站东3m和7.5m处分别有一棵柳树和一棵杨树,汽车站牌西3m和4.8m处分别有一颗槐树和一根电线杆,试画图表示这一情景.<br>

	<p>&nbsp;&nbsp;&nbsp;&nbsp; 如图1.2-1,画一条直线表示马路,从左到右表示从西到东的方向,在直线上任取一点O表示汽车站牌的位置,规定1个单位长度(线段OA的长)代表1m长,于是,在点O右边,与点O距离3个和7.5个单位长度的点B和点C,分别表示柳树和杨树的位置:点O左边,与点O距离3个和4.8个单位长度的点D和点E,分别表示槐树和电线杆的位置.<br>

	\beginfigure[ht!]
		\centering
		 <font face=symbol>Î</font> cludegraphics[width=1\textwidth,natwidth=400,natheight=63]./1.2-1.PNG
		<font face=symbol>Ç</font>tion1.2-1
	\endfigure

	思考<br>

	<p>&nbsp;&nbsp;&nbsp;&nbsp; 怎样用数简明地表示这些树,"东"与"西"、"左"与"右"都具有相反 意义.如图1.2-2,在一条直线上取一个点O为基准点,用0表示它,在用负数表示O左边的点,用正数表示点O右边的点.这样,我们就用负数、0、正数表示出了这条直线上的点.<br>

	\beginfigure[ht!]
		\centering
		 <font face=symbol>Î</font> cludegraphics[width=1\textwidth,natwidth=400,natheight=60]./1.2-2.PNG
		<font face=symbol>Ç</font>tion1.2-2
	\endfigure
	<p>&nbsp;&nbsp;&nbsp;&nbsp; 用上述方法,我们就可以把这些树电线杆与汽车站牌的相对位置关系表示出来了.例如,-4.8表示位于汽车站牌西侧4.8m处的电线杆,等等.<br>

	
	思考<br>

	<p>&nbsp;&nbsp;&nbsp;&nbsp; 图1.2-3中的温度计可以看作表示正数、0和负数的直线.它和图1.2.2有什么共同点,有什么不同点?
	\beginfigure[ht!]
		\centering
		 <font face=symbol>Î</font> cludegraphics[width=1\textwidth,natwidth=600,natheight=140]./1.2-3.PNG
		<font face=symbol>Ç</font>tion1.2-3
	\endfigure

	\definition数轴
	\begindefintion
		在数学中,可以用一条直线上的点表示数,则条直线叫做数轴(number axis).它满足以下要求:<br>

	\enddefintion
	
	Phys.Rev. operty性质
	\beginpropertory
		<p>&nbsp;&nbsp;&nbsp;&nbsp; (1)在一条直线上任取一个点表示数0,这个点叫做原点(origin);<br>

		<p>&nbsp;&nbsp;&nbsp;&nbsp; (2)通常规定直线上从原点向右(或上)为正方向,从原点向左(向下)为负方向;<br>

		<p>&nbsp;&nbsp;&nbsp;&nbsp; (3)选取适当的长度为单位长度,直线上从原点向右,每隔一个长度单位取一个点,一次表示1,2,3,...;从原点向左,用类似方法一次表示-1,-2,-3,...(图1.2-4).<br>

	\endpropertory
	
	\beginfigure[ht!]
		\centering
		 <font face=symbol>Î</font> cludegraphics[width=1\textwidth,natwidth=400,natheight=40]./1.2-4.PNG
		<font face=symbol>Ç</font>tion1.2-4
	\endfigure
	<p>&nbsp;&nbsp;&nbsp;&nbsp; 分数或小数也可以用数轴上的点表示,例如从原点向右6.5个单位长度的点表示小数6.5,从原点向左(3)/(2)个单位长度的点表示分数<font face=symbol>-</font>(3)/(2)(图1.2-4).<br>

	
	归纳<br>

	<p>&nbsp;&nbsp;&nbsp;&nbsp; 一般的,设a是一个正数,则数轴上表示数a的点再原点的\underline\hbox to 10mm边,与原点的距离是\underline\hbox to 10mm个单位长度;表示-a的点在原点的\underline\hbox to 10mm边,与原点的距离是\underline\hbox to 10mm个单位长度.<br>

	<p>&nbsp;&nbsp;&nbsp;&nbsp; 用数轴上的点表示对数学的发展起到了重要作用,以它做基础,可以借助图直观地表示很多与数相关的问题.<br>

	
	\exercise练习
	\beginexercise
		1.如图,写出数轴上点A,B,C,D,E表示的数.<br>

		\beginfigure[ht!]
			\centering
			 <font face=symbol>Î</font> cludegraphics[width=1\textwidth,natwidth=400,natheight=40]./exe1.1.PNG
			<font face=symbol>Ç</font>tion1
		\endfigure
		2.画出数轴并表示下列有理数:<br>

		<p>&nbsp;&nbsp;&nbsp;&nbsp; 1.5,<font face=symbol>-</font>2,2,<font face=symbol>-</font>2.5,(9)/(2),<font face=symbol>-</font>(3)/(4),0.<br>

		3.数轴上,如果表示a的点在原点的左边,那么a是一个\underline\hbox to 10mm数;如果表示数b的点在原点的右边,那么b是一个\underline\hbox to 10mm数.<br>

	\endexercise

	<h2>1.2.3有理数</h2>
	探究<br>

	<p>&nbsp;&nbsp;&nbsp;&nbsp; 在数轴上,与原点的距离是2的点有几个?这些点表示哪个数?<br>

	设a是一个正数,数轴上与原点的距离等于a的点有几个?这些点表示的数有什么关系?<br>

	<p>&nbsp;&nbsp;&nbsp;&nbsp; 可以发现,数轴上与原点的距离是2的点有两个,他们表示的数是-2和2.<br>

	
	归纳<br>

	<p>&nbsp;&nbsp;&nbsp;&nbsp; 一般地,设a是一个正数,数轴上与原点的距离是a的点有两个,它们分别在原点左右,表示-a和a(图1.2-5),我们说这两点关于原点对称.<br>

	\beginfigure[ht!]
		\centering
		 <font face=symbol>Î</font> cludegraphics[width=1\textwidth,natwidth=400,natheight=40]./1.2-5.PNG
		<font face=symbol>Ç</font>tion1.2-5
	\endfigure
	<p>&nbsp;&nbsp;&nbsp;&nbsp; 像2和-2,5和-5这样,只有符号不同的两个数叫做互为相反数(oposite number).这就是说,2的相反数是-2,-2的相反数是2;5的相反数是-5,-5的相反数是5.<br>

	<p>&nbsp;&nbsp;&nbsp;&nbsp; 一般地,a和-a互为相反数.特别地,0的相反数是0.这里,a表示任意一个数,可以是正数、负数,也可以是0.例如:<br>

	<p>&nbsp;&nbsp;&nbsp;&nbsp; 当a=1时,-a=-1,1的相反数是-1;同时,-1的相反数是1.<br>

	思考<br>

	<p>&nbsp;&nbsp;&nbsp;&nbsp; 设a表示一个数,-a一定是负数吗?<br>

	<p>&nbsp;&nbsp;&nbsp;&nbsp; 容易看出,在正数前面添加"-"号,就得到这个正数的相反数.在任意一个数前面添上"-"号,新的数表示原数的相反数.例如.<br>

	<p>&nbsp;&nbsp;&nbsp;&nbsp; <font face=symbol>-</font>(+5)=<font face=symbol>-</font>5, <font face=symbol>-</font>(<font face=symbol>-</font>5)=+5, <font face=symbol>-</font>0=0.<br>

	\exercise练习
	\beginexercise
		1.判断下列说法是否正确:<br>
	
		\newcommand\fourch[4]
			<p>&nbsp;&nbsp;&nbsp;&nbsp;\makebox[262pt][l]&nbsp;&nbsp;&nbsp;&nbsp;&nbsp;&nbsp;(A) #1<br>

			<p>&nbsp;&nbsp;&nbsp;&nbsp;\makebox[262pt][l]&nbsp;&nbsp;&nbsp;&nbsp;&nbsp;&nbsp;(B) #2<br>

			<p>&nbsp;&nbsp;&nbsp;&nbsp;\makebox[262pt][l]&nbsp;&nbsp;&nbsp;&nbsp;&nbsp;&nbsp;(C) #3<br>

			<p>&nbsp;&nbsp;&nbsp;&nbsp;\makebox[262pt][l]&nbsp;&nbsp;&nbsp;&nbsp;&nbsp;&nbsp;(D) #4<br>

		2.写出下列各数的相反数:<br>

		<p>&nbsp;&nbsp;&nbsp;&nbsp; 6,-8,-3.9,(5)/(2),<font face=symbol>-</font>(2)/(11),100,0.<br>

		3.如果a=-a,那么表示a的点在数轴上的什么位置?<br>

		4.化简下列各数:<br>

		<p>&nbsp;&nbsp;&nbsp;&nbsp; <font face=symbol>-</font>(<font face=symbol>-</font>68),<font face=symbol>-</font>(+0.75),<font face=symbol>-</font>(<font face=symbol>-</font>(3)/(5)),<font face=symbol>-</font>(+3.8).<br>

	\endexercise	

\enddocument</body>
</html>
