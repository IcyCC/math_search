<html>
<head>
<title>LaTeX4Web 1.4 OUTPUT</title>
<style type="text/css">
<!--
 body {color: black;  background:"#FFCC99";  }
 div.p { margin-top: 7pt;}
 td div.comp { margin-top: -0.6ex; margin-bottom: -1ex;}
 td div.comb { margin-top: -0.6ex; margin-bottom: -.6ex;}
 td div.norm {line-height:normal;}
 td div.hrcomp { line-height: 0.9; margin-top: -0.8ex; margin-bottom: -1ex;}
 td.sqrt {border-top:2 solid black;
          border-left:2 solid black;
          border-bottom:none;
          border-right:none;}
 table.sqrt {border-top:2 solid black;
             border-left:2 solid black;
             border-bottom:none;
             border-right:none;}
-->
</style>
</head>
<body>
\documentclassarticle
\usepackage[utf8]ctex

\begindocument

\maketitle

\beginexample
解:设安排x人先做4h.\newline
根据先后两个时段的工作量之和应等于总工作量,列出方程\newline
(4x)/(40)+(8(x+2))/(40)=1.\newline
解方程,得\newline
4x+8(x+2)=40,\newline
4x+8x+16=40,\newline
12x=24,\newline
x=2.\newline
答:应安排2人先做4h.\newline

这类问题中常常把工作总量看做1,并利用“工作量=人均效率&times;人数&times;时间”的关系考虑问题.\newline

归纳\newline

用一元一次方程解决实际问题的基本过程如下:\newline

这一过程一般包括设、列、解、答等步骤,即设未知数,列方程,解方程,检验所得结果,确定答案. 正确分析问题中的相等关系是列方程的基础.\newline
\endexample
\beginexercise
练习\newline

1. 一套仪器由一个A部件和三个B部件构成. 用1m<sup>2</sup>钢材可做40个A部件或240个B部件. 现要用6m<sup>2</sup>钢材制作这种仪器,应用多少钢材做A部件,多少钢材做B部件,恰好配成这种仪器多少套?\newline

2. 一条地下管线由甲工程队单独铺设需要12天,由乙工程队单独铺设需要24天. 如果由这两个工程队从两端同时施工,要多少天可以铺好这条管线?\newline
\endexercise
\beginexample

有些实际问题中,数量关系比较隐蔽,需要仔细分析才能列出方程. 下面我们进一步探究几个这样的问题. \newline

探究1\newline

销售中的盈亏\newline

一商店在某一时间以每件60元的价格卖出两件衣服,其中一件盈利25\%,另一件亏损25\%,卖这两件衣服总的是盈利还是亏损,或是不盈不亏?\newline

分析:两件衣服共卖了120(=60&times;2)元,是盈是亏要着这家商店买进这两件衣服时花了多少钱,如果进价大于售价就亏损,反之就盈利. \newline

先大体估算盈亏,再通过准确计算检验你的判断. \newline

假设一件商品的进价是40元,如果卖出后盈利25\%,那么商品利润是40X25\%元;如果卖出后亏损25\%,商品利润是40&times;(<font face=symbol>-</font>25\%)元. \newline

本问题中,设盈利25\%的那件衣服的进价是x元,它的商品利润就是0.25x元. 根据进价与利润的和等于售价,列出方程\newline

x+0.25x=60.\newline

由此得\newline

x=48.\newline

类似地,可以设另一件衣服的进价为y元,它的商品利润是-0.25y元,列出方程\newline

y<font face=symbol>-</font>0.25y=60.\newline

由此得\newline

y=80.\newline

两件衣服的进价是x+y=128元,而两件衣服的售价是60+60=120元,进价大于售价由此可知卖这两件衣服总共亏损8元. \newline

列、解方程后得出的结论与你先前估算一致吗?通过对本题的探究,你对方程在实际问题中的应用有什么新的认识?\newline

探究2\newline

球赛积分表问题\newline

某次篮球联赛积分榜\newline

(1)用式子表示总积分与胜、负场数之间的数量关系;\newline

(2)某队的胜场总积分能等于它的负场总积分吗?\newline

分析:观察积分榜,从最下面一行数据可以看出:负一场积一分. \newline

设胜一场积x分,从表中其他任何一行可以列方程,求出x的值. 例如,从第一行得方程\newline

10x+1&times;4=24.\newline

由此得\newline

x=2.\newline

用积分榜中其他行可以验证,得出结论:负一场积1分,胜一场积2分.\newline

(1)如果一个队胜m场,则负(14<font face=symbol>-</font>m)场,胜场积分为2m,负场积分为14<font face=symbol>-</font>m,总积分为\newline

2m+(14<font face=symbol>-</font>m)=m+14. \newline

(2)设一个队胜了x场,则负了(14<font face=symbol>-</font>x)场. 如果这个队的胜场总积分等于负场总积分,则得方程\newline

2x=14<font face=symbol>-</font>x. \newline

由此得\newline

x=(14)/(3). \newline

想一想,x表示什么量?它可以是分数吗?山此你能得出什么结论?

解决实际问题时,要考虑得到的结果是不是符合实际。x(所胜的场数)的值必须是整数,所以x=(14)/(3)不符合实际,由此可以判定没有哪个队的胜场总积分等于负场总积分。

这个问题说明:利用方程不仅能求具体数值,而且可以进行推理判断。

上面的问题说明,用方程解决实际问题时,不仅要注意解方程的过程是否正确,还要检验方程的解是否符合问题的实际意义。



考虑下列问题:

探究3

电话计费问题

下表中有两种移动电话计费方式。

(1) 设一个月内用移动电话主叫为t min(t是正整数).根据上表,列表说明:当t在不同时间范围内取值时,按方式一和方式二如何计费。

(2) 观察你的列表,你能从中发现如何根据主叫时间选择省钱的计费方式吗?通过计算验证你的看法。

分析: (1)由上表可知,计费与主叫时间相关,计费时首先要看主叫是否超过限定时间。因此,考虑t的取值时,两个主叫限定时间150 min和350 min是不同时间范围的划分点。

当t在不同时间范国内取值时,方式一和方式二的计费如下页表:

想一想,x表示什么量?它可以是分数吗?山此你能得出什么结论?

解决实际问题时,要考虑得到的结果是不是符合实际。x(所胜的场数)的值必须是整数,所以x=(14)/(3)不符合实际,由此可以判定没有哪个队的胜场总积分等于负场总积分。

这个问题说明:利用方程不仅能求具体数值,而且可以进行推理判断。

上面的问题说明,用方程解决实际问题时,不仅要注意解方程的过程是否正确,还要检验方程的解是否符合问题的实际意义。

考虑下列问题:

探究3

电话计费问题

下表中有两种移动电话计费方式。

(1) 设一个月内用移动电话主叫为t min(t是正整数).根据上表,列表说明:当t在不同时间范围内取值时,按方式一和方式二如何计费。

(2) 观察你的列表,你能从中发现如何根据主叫时间选择省钱的计费方式吗?通过计算验证你的看法。

分析: (1)由上表可知,计费与主叫时间相关,计费时首先要看主叫是否超过限定时间。因此,考虑t的取值时,两个主叫限定时间150 min和350 min是不同时间范围的划分点。

当t在不同时间范国内取值时,方式一和方式二的计费如下页表:
\endexample

\enddocument
</body>
</html>
