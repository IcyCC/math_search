<html>
<head>
<title>LaTeX4Web 1.4 OUTPUT</title>
<style type="text/css">
<!--
 body {color: black;  background:"#FFCC99";  }
 div.p { margin-top: 7pt;}
 td div.comp { margin-top: -0.6ex; margin-bottom: -1ex;}
 td div.comb { margin-top: -0.6ex; margin-bottom: -.6ex;}
 td div.norm {line-height:normal;}
 td div.hrcomp { line-height: 0.9; margin-top: -0.8ex; margin-bottom: -1ex;}
 td.sqrt {border-top:2 solid black;
          border-left:2 solid black;
          border-bottom:none;
          border-right:none;}
 table.sqrt {border-top:2 solid black;
             border-left:2 solid black;
             border-bottom:none;
             border-right:none;}
-->
</style>
</head>
<body>
\documentclass[UTF8]report
\usepackageCTEX
\usepackage[english]babel
\usepackageamsmath,amsthm
\usepackageamsfonts
\usepackageexercise
\newtheoremthmTheorem[chapter]
\newtheoremcor[thm]Corollary
\newtheoremlem[thm]Lemma
\newtheoremprop[thm]Proposition
\theoremstyledefinition
\newtheoremdefn[thm]Definition
\theoremstyleremark
\newtheoremrem[thm]Remark
\begindocument
<br>agraph数字1和字母X的对话<br>
1:数学是由数产生的,数才是数学王国的真正主人。<br>
X:我是字母,我虽然不是具体的数,但是可以表示各种各样的数,我可以代表你1,也可以代表其他数,<br>
1:由我们数组成的式子有确切的大小例如,人们一见到1+2就知道是1与2的和,即3. 你们字母能这样做吗?<br>
X:有我们字母的式子进行运算和推理时具有一般性。例如,x+y可以表示任何两个数的和,包括1+2. x+y=y+工能表示任何两数相加时都可以交换顺序,即加法交换律.<br>
1:人们解决实际问题时,必须根据已知的具体数字进行计算,而字母有什么用呢?X;在解决实际问题时,用字母表示来知数,把字母列入算式(方程),能更方便地表示数量关系。数和字母一起运算会使问题的解法更简单.<br>
1:数是人们经过长期实践创造出来的,并建立了专门研究数及其运算的学科一 算术,你们字母行吗?<br>
X:随着实践的发展,人们发现只有算术还不够,用字母表示数会起到更大的作用,于是产生了代数这门学科。它首要研究的就是用字母表示的式子的运算法则和方程的解法,从算术发展到代数是数学的一大进步。<br>
1:算术几乎是伴随着人类社会活动的产生和发展而逐渐形成的,它有着非常悠久的历史,代数有怎样的历史呢?<br>
X:代数的历史可以追意到约3800年首的古埃及和古巴比伦时期,那时就有了代数的萌芽。到了公元3世纪,代数在希腊获得显著的发展,其代表人物是被誉为代数学鼻祖的丢香图。之后,印度的代数发展很快。同时,阿拉位地区的代数研究取得很大进展,其中著名的代表作是数学家阿尔花拉子米于公元820年左右发表的《代数学》(这本书的拉丁文译本取名为《对消与还原》).这本书第一次提出了这门学科的名称.
<p><a name="toc.1"><h1>1&nbsp;整式的加减</h1>
<br>agraph我们来看本章引言中的问题(2)。<br>
在西宁到拉萨路段,如果列车通过冻土路段的时间是th,那么它通过非冻土地段的时间是2.12.1th,这段铁路的全长(单位:km)是<br>
100t+120&times;2.1t,<br>
 即<br>
100t+252t,<br>
 类比数的运算,我们应如何为化简式子100t+252t 呢?

<br>agraph在(1)中,我们知道,根据分配律可得<br>
100&times;2+252&times;2=(100+252)&times;2=352&times;2.<br>
100&times;(<font face=symbol>-</font>2)+252&times;(<font face=symbol>-</font>2)=(100+252)&times;(<font face=symbol>-</font>2)=352&times;(<font face=symbol>-</font>2).<br>
 在(2)中,式子100t+252t表示100t与252t两项的和。式子<br>
100t+252t<br>
和<br>
100&times;(<font face=symbol>-</font>2)+252&times;(<font face=symbol>-</font>2)<br>
有相同的结构,并且字母t代表的是一个因(乘)数,因此根据分配律也应该有<br>
100&times;t+252&times;t=(100+252)&times;t=352t.

<br>agraph
对于上面的(1)(2)(3),利用分配律可得
<br>
100t<font face=symbol>-</font>252t=(100<font face=symbol>-</font>252)t=<font face=symbol>-</font>152t,
<br>
3x<sup>2</sup>
</td>
<td nowrap align=center>
  +2x<sup>2</sup>
</td>
<td nowrap align=center>
  =(3+2)x<sup>2</sup>
</td>
<td nowrap align=center>
  =5x<sup>2</sup>
</td>
<td nowrap align=center>
  ,
<br>
3ab<sup>2</sup>
</td>
<td nowrap align=center>
  <font face=symbol>-</font>4ab<sup>2</sup>
</td>
<td nowrap align=center>
  (3<font face=symbol>-</font>4)ab<sup>2</sup>
</td>
<td nowrap align=center>
  =<font face=symbol>-</font>ab<sup>2</sup>
</td>
<td nowrap align=center>
  ,
<br>
  观察(1)中的多项式的项100t和<font face=symbol>-</font>252t, 它们含有相同的字母t,并且t的指数都是1; (2)中的多项式的项3x<sup>2</sup>
</td>
<td nowrap align=center>
   和2x<sup>2</sup>
</td>
<td nowrap align=center>
  , 含有相同的字母x, 并且x的指数都是2; (3)中的多项式的项3ab<sup>2</sup>
</td>
<td nowrap align=center>
  与<font face=symbol>-</font>4ab<sup>2</sup>
</td>
<td nowrap align=center>
  ,都含有字母a, b,并且a的指数都是1次,b 的指数都是2次。像100t与<font face=symbol>-</font>252t, 3x<sup>2</sup>
</td>
<td nowrap align=center>
  与2x<sup>2</sup>
</td>
<td nowrap align=center>
  ,3ab与<font face=symbol>-</font>4ab<sup>2</sup>
</td>
<td nowrap align=center>
  这样,所含字母相同,并且相同字母的指数也相同的项叫做同类项,几个常数项也是同类项
<br>
  因为多项式中的字母表示的是数,所以我们也可以运用交换律、结合律、分配律把多项式中的同类项进行合并,例如,
<br>
4x<sup>2</sup>
</td>
<td nowrap align=center>
  +2x+7+3x<font face=symbol>-</font>8x<sup>2</sup>
</td>
<td nowrap align=center>
  <font face=symbol>-</font>2=4x<sup>2</sup>
</td>
<td nowrap align=center>
  <font face=symbol>-</font>8x<sup>2</sup>
</td>
<td nowrap align=center>
  +2x=3x+7<font face=symbol>-</font>2=(4x<sup>2</sup>
</td>
<td nowrap align=center>
  <font face=symbol>-</font>8x2<sup>2</sup>
</td>
<td nowrap align=center>
  +(2x+3x)+(7<font face=symbol>-</font>2)=(4<font face=symbol>-</font>8)x<sup>2</sup>
</td>
<td nowrap align=center>
  +(2+3)x+(7<font face=symbol>-</font>2)=<font face=symbol>-</font>4x<sup>2</sup>
</td>
<td nowrap align=center>
  +(2+3)x+(7<font face=symbol>-</font>2)=<font face=symbol>-</font>4x<sup>2</sup>
</td>
<td nowrap align=center>
  +5x+5.

\defn把多项式中的同类项合并成一项,叫做合并同类项。合并同类项后,所得项的系数是合并前各同类项的系数的和,且字母连同它的指数不变。



\beginExercise合并下列各式的同类项:
 <br>
(1)xy<sup>2</sup>
</td>
<td nowrap align=center>
  <font face=symbol>-</font>(1)/(5)xy<sup>2</sup>
</td>
<td nowrap align=center>
  ;
 <br>
(2)<font face=symbol>-</font>3x<sup>2</sup>
</td>
<td nowrap align=center>
  y+2x<sup>2</sup>
</td>
<td nowrap align=center>
  y<font face=symbol>-</font>3xy<sup>2</sup>
</td>
<td nowrap align=center>
  <font face=symbol>-</font>2xy ;
 <br>
(3)4a<sup>2</sup>
</td>
<td nowrap align=center>
  +3b<sup>2</sup>
</td>
<td nowrap align=center>
  +2ab<font face=symbol>-</font>4a<sup>2</sup>
</td>
<td nowrap align=center>
  <font face=symbol>-</font>4b<sup>2</sup>
</td>
<td nowrap align=center>
  .
 \endExercise
 \beginAnswer
 解:(1)xy<sup>2</sup>
</td>
<td nowrap align=center>
  <font face=symbol>-</font>(1)/(5)xy<sup>2</sup>
</td>
<td nowrap align=center>
  =(1<font face=symbol>-</font>(1)/(5))xy<sup>2</sup>
</td>
<td nowrap align=center>
  =(4)/(5)xy<sup>2</sup>
</td>
<td nowrap align=center>
  ;
 <br>
(2)<font face=symbol>-</font>3x<sup>2</sup>
</td>
<td nowrap align=center>
  y+2x<sup>2</sup>
</td>
<td nowrap align=center>
  y<font face=symbol>-</font>3xy<sup>2</sup>
</td>
<td nowrap align=center>
  <font face=symbol>-</font>2xy
 <br>
=(<font face=symbol>-</font>3+2)x<sup>2</sup>
</td>
<td nowrap align=center>
  y+(3<font face=symbol>-</font>2)xy<sup>2</sup>
</td>
<td nowrap align=center>

 <br>
=<font face=symbol>-</font>x<sup>2</sup>
</td>
<td nowrap align=center>
  y+xy<sup>2</sup>
</td>
<td nowrap align=center>
  ;
 <br>
(3)4a<sup>2</sup>
</td>
<td nowrap align=center>
  +3b<sup>2</sup>
</td>
<td nowrap align=center>
  +2ab<font face=symbol>-</font>4a<sup>2</sup>
</td>
<td nowrap align=center>
  <font face=symbol>-</font>4b<sup>2</sup>
</td>
<td nowrap align=center>

 <br>
=(4a<sup>2</sup>
</td>
<td nowrap align=center>
  <font face=symbol>-</font>4a<sup>2</sup>
</td>
<td nowrap align=center>
  )+(3<font face=symbol>-</font>4)b<sup>2</sup>
</td>
<td nowrap align=center>
  +2ab
 <br>
<font face=symbol>-</font>b<sup>2</sup>
</td>
<td nowrap align=center>
  +2ab.
 \endAnswer

\beginExercise
(1) 求多项式2x<sup>2</sup>
</td>
<td nowrap align=center>
  <font face=symbol>-</font>5x+x<sup>2</sup>
</td>
<td nowrap align=center>
  +4x<font face=symbol>-</font>3x<sup>2</sup>
</td>
<td nowrap align=center>
  <font face=symbol>-</font>2的值,其中x=(1)/(2);
 <br>
(2)求多项式3a<font face=symbol>-</font>abc<font face=symbol>-</font>(1)/(3)c<sup>2</sup>
</td>
<td nowrap align=center>
  <font face=symbol>-</font>3a+(1)/(3)c<sup>2</sup>
</td>
<td nowrap align=center>
  的値,其中a=<font face=symbol>-</font>(1)/(6),b=2,c=-3.
\endExercise
 \beginAnswer
分析:在求多项式的值时,可以先将多项式中的同类项合并,然后再求值,这样做往往可以简化计算.
<br>
 解: (1) 2x<sup>2</sup>
</td>
<td nowrap align=center>
  <font face=symbol>-</font>5x+x<sup>2</sup>
</td>
<td nowrap align=center>
  +4x<font face=symbol>-</font>3x<sup>2</sup>
</td>
<td nowrap align=center>
  <font face=symbol>-</font>2
      <br>
=(2+1<font face=symbol>-</font>3)x<sup>2</sup>
</td>
<td nowrap align=center>
  +(<font face=symbol>-</font>5+4)x<font face=symbol>-</font>2
      <br>
=<font face=symbol>-</font>x<font face=symbol>-</font>2.
<br>
 当x=(1)/(2)时,原式=<font face=symbol>-</font>(1)/(2)<font face=symbol>-</font>2=<font face=symbol>-</font>(2)/(5).
(2)3a+abc<font face=symbol>-</font>(1)/(3)c<sup>2</sup>
</td>
<td nowrap align=center>
  <font face=symbol>-</font>3a+(1)/(3)c<sup>2</sup>
</td>
<td nowrap align=center>

<br>
=(3<font face=symbol>-</font>3)a+abc+(<font face=symbol>-</font>(1)/(3))c<sup>2</sup>
</td>
<td nowrap align=center>

<br>
=abc.
<br>
 当a=<font face=symbol>-</font>(1)/(6),b=2,c=-3时,原式=(<font face=symbol>-</font>(1)/(6)&times;2&times;(<font face=symbol>-</font>3)=1).
\endAnswer

\beginExercise
1.计算:
      <br>
(1)12x<font face=symbol>-</font>20x;  (2)x+7x<font face=symbol>-</font>5x;
      <br>
(3)<font face=symbol>-</font>5a+0.3a<font face=symbol>-</font>2.7a  (4)(1)/(3)y<font face=symbol>-</font>(2)/(3)y+2y;
      <br>
(5)<font face=symbol>-</font>6ab<font face=symbol>-</font>ba+8ab;  (6)10y<sup>2</sup>
</td>
<td nowrap align=center>
  <font face=symbol>-</font>0.5y<sup>2</sup>
</td>
<td nowrap align=center>
  ;
\endExercise

\beginExercise2.求下列各式的値:
<br>
(1) 3a+2b<font face=symbol>-</font>5a<font face=symbol>-</font>b,其中a=-2,b=l;<br>
(2) 3x<font face=symbol>-</font>4x<sup>2</sup>
</td>
<td nowrap align=center>
  +7<font face=symbol>-</font>3x+2x<sup>2</sup>
</td>
<td nowrap align=center>
  +1,其中x=-3.
<br>
3. (1) x的4倍与x的5倍的和是多少?
<br>
(2) x的3倍比x的一半大多少?
<br>
4.如图,大圆的半径是R,小圆的半径是大圆面积的(4)/(9),求阴影部分的面积。
\endExercise

\beginExercise
(1) 水库水位第一天连续下降了ah,每小时平均下降2cm;第二天连续上升了ah,每小时平均上升0.5cm,这两天水位总的变化情况如何?
<br>
(2)某商店原有5袋大米,每袋大米为xkg.上午卖出3袋, 下午又购进同样包装的大米4袋。进货后这个商店有大米多少千克?
<br>
 解: (1) 把下降的水位变化量记为负,上升的水位变化量记为正.第一天水位的变化量是<font face=symbol>-</font>2a cm,第二天水位的变化量是0.5a cm,
<br>
 两天水位的总变化量(单位: cm)是
<br>
  <font face=symbol>-</font>2a+0.5a=(<font face=symbol>-</font>2+0.5)a= <font face=symbol>-</font>1.5a.
<br>
 这两天水位总的变化情况为下降了1.5a cm.
<br>
 (2)把进货的数量记为正,售出的数量记为负进货后这个商店共有大米(单位: kg)
<br>
  5x<font face=symbol>-</font>3x+4x=(5<font face=symbol>-</font>3+4)x=6x.
\endExercise
\enddocument

<hr>
<p><h1>Table Of Contents</h1>
<p><a href="#toc.1"><h1>1&nbsp;整式的加减</h1></a>
</body>
</html>
