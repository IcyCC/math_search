<html>
<head>
<title>LaTeX4Web 1.4 OUTPUT</title>
<style type="text/css">
<!--
 body {color: black;  background:"#FFCC99";  }
 div.p { margin-top: 7pt;}
 td div.comp { margin-top: -0.6ex; margin-bottom: -1ex;}
 td div.comb { margin-top: -0.6ex; margin-bottom: -.6ex;}
 td div.norm {line-height:normal;}
 td div.hrcomp { line-height: 0.9; margin-top: -0.8ex; margin-bottom: -1ex;}
 td.sqrt {border-top:2 solid black;
          border-left:2 solid black;
          border-bottom:none;
          border-right:none;}
 table.sqrt {border-top:2 solid black;
             border-left:2 solid black;
             border-bottom:none;
             border-right:none;}
-->
</style>
</head>
<body>
\documentclass[UTF8]article
\usepackageCTEX
\usepackagetextcomp
\usepackagegraphicx
\begindocument
实验与探究
填幻方
有人建议向火星发射如图1的困案。它叫做幻方,其中9个格中的点数分别是1, 2.3,4.5.6, 7, 8.9.每一横行、每一整列以及两条斜对角线上的点数的和都是15.如果火星上有智能生物,那么他们可以从这种“数学语言”了解到地球上也有智能生物(人).

你能将4, -3, -2. -1, 0.1, 2, 3, 4这9个数分别填入图2的幻方的9个空格中,使得处于同一横行、同一整列、同一斜对角线上的3个数相加都得0吗?
你是将0填入中央的格中吗?与同学交流一下,你们镇这个幻方的方法相同吗?

1.3.2 有理数的减法
实际问题中有时还要涉及有理数的减法。例如,本章引言中,北京某天的气温是一-3\textcelsius &nbsp;3\textcelsius,这天的温差(最高气温减最低气温,单位:\textcelsius )就是3-(-3).这里遇到正数与负数的减法
减法是加法的逆运算,计算3-(- 3), 就是:要求出一个数x,使得x与-3相加得3.因为6与-3相加得3,所以工应该是6,即
3-(-3)=6.①另方面,我们知道
3+(+3)=6,②由①②,有
3-(- 3)=3+(+3).③
探究

从③式能看出减3相当于加哪个数吗?把3换成0,-1,-5,用上面的方法考虑

0-(-3),(-1)-(-3),(-5)-(-3).这些数减一3的结果与它们加十3的结果相

同吗?  换几个数再试一试

计算

9 -8, 9+(-8); 15-7,15+(-7).从中又有什么新发现?

可以发现,有理数的减法可以转化为加法来进行有理数减法法则:

减去一个数。等于加这个数的相反数。有理数减法法则也可以表示成

a-b=a+(-b).

例4计算:

(1) (-3)-(-5);  (2) 0-7;

(3) 7.2-(- 4.8);  (4)<font face=symbol>-</font>3(1)/(2)<font face=symbol>-</font>5(1)/(4);
解: 
(1) (--3)-(- 5)=(- -3)+5=2;(2) 0-7=0+(-7)= -7;

(3) 7.2-(-4.8)=7.2+4.8=12;

(4)(<font face=symbol>-</font>3(1)/(2))-5(1)/(4)=(<font face=symbol>-</font>3(1)/(2))+(<font face=symbol>-</font>5(1)/(4)) =(<font face=symbol>-</font>8(3)/(4))

思考

在小学,只有当a大于或等于b时,我们才会做a-b (例如2- 1.1-1).现在,当a小于b时,你会做a-b (例如1-2,(-1)-1)吗?

一般地,较小的数减去较大的数,所得的差的符号是什么?

1. 计算:
(1)6<font face=symbol>-</font>9
(2)(+4)<font face=symbol>-</font>(<font face=symbol>-</font>7)
(3)(<font face=symbol>-</font>5)<font face=symbol>-</font>(<font face=symbol>-</font>8)
(4)0<font face=symbol>-</font>(<font face=symbol>-</font>5)
(5)(<font face=symbol>-</font>2.5)<font face=symbol>-</font>5.9
(6)1.9<font face=symbol>-</font>(<font face=symbol>-</font>0.6)

2.计算:
(1)比2<sup><font face=symbol>°</font></sup>
</td>
<td nowrap align=center>
  C <br> 低8<sup><font face=symbol>°</font></sup>
</td>
<td nowrap align=center>
  C <br> 的温度;
(2)比<font face=symbol>-</font>3<sup><font face=symbol>°</font></sup>
</td>
<td nowrap align=center>
  C <br> 低6<sup><font face=symbol>°</font></sup>
</td>
<td nowrap align=center>
  C <br> 的温度。

下面我们研究怎样进行有理数的加减混合运算.

例5 计算(- 20)+(+3)-(-5)- (+7).
分析:这个算式中有加法,也有减法。可以根据有理数减法法则,把它改写为
	(<font face=symbol>-</font> 20)+(+3)+&lt;+5&gt;+&lt;<font face=symbol>-</font>7). 
使问题转化为几个有理数的加法,
解:  (一20)+&lt;+3)<font face=symbol>-</font>&lt;<font face=symbol>-</font>5&gt;<font face=symbol>-</font>(+7&gt;
	= ( 20&gt;+&lt;+3)+(+5)+(<font face=symbol>-</font> 7)
	=[(<font face=symbol>-</font> <font face=symbol>-</font>20)+(<font face=symbol>-</font>7)]+[(+5)+(+3&gt;]=(<font face=symbol>-</font>27)+&lt;+8)= <font face=symbol>-</font>19.

这里使用了哪些运算律?

归纳
引入相反数后,加减混合运算可以统一为加法运算。
	a+b<font face=symbol>-</font>c=a+b+(<font face=symbol>-</font>c).	

算式
	(<font face=symbol>-</font> 20)+&lt;+3&gt;+&lt;+5&gt;+&lt;<font face=symbol>-</font>7)
是一20, 3, 5, <font face=symbol>-</font>7这四个数的和,为书写简单,可以省略算式中的括号和加号,把它写为

	一20+3+5<font face=symbol>-</font>7.
这个算式可以读作“负20、正3、正5、负7的和",或读作“负20加3

加5减7”.例5的运算过程也可以简单地写为
	( 20)+(+3)<font face=symbol>-</font>(<font face=symbol>-</font>5)<font face=symbol>-</font> (+7)
	= <font face=symbol>-</font> 20+3+5<font face=symbol>-</font>7
	=<font face=symbol>-</font>20<font face=symbol>-</font>7+3+5
	=<font face=symbol>-</font>27+8
	=<font face=symbol>-</font> 19.

探究
在数轴上,点A, B分别表示数a, b.利用有理数减法,分别计算下列情况下点A, B之间的距离:
	a=2, b=6; a=0. b=6; a=2. b=<font face=symbol>-</font>6; a=<font face=symbol>-</font>2, b=<font face=symbol>-</font>6.
你能发现点A. B之间的距离与数a. b之间的关系吗?

练习
计算:
(1) 1<font face=symbol>-</font>4+3<font face=symbol>-</font>0.5;
(2) <font face=symbol>-</font>2.4+3.5<font face=symbol>-</font> 4.6+3.5;
(3) (<font face=symbol>-</font>7)<font face=symbol>-</font>(+5)+(<font face=symbol>-</font>4)<font face=symbol>-</font>(<font face=symbol>-</font>10);
(4) (3)/(4) <font face=symbol>-</font> (7)/(2)+(<font face=symbol>-</font>(1)/(6))<font face=symbol>-</font>(<font face=symbol>-</font>(2)/(3))<font face=symbol>-</font>1


习题1.3
复习巩固
1.计算:
	(1) (<font face=symbol>-</font>10)+(+6);
	(2) (+12)+(<font face=symbol>-</font>4);
	(3) (<font face=symbol>-</font>5)+(<font face=symbol>-</font>7);
	(4) (+6)+(<font face=symbol>-</font>9);
	(5) (<font face=symbol>-</font>0.9)+(<font face=symbol>-</font>2.7);
	(6) (2)/(5)+(<font face=symbol>-</font>(3)/(5));
	(7) (<font face=symbol>-</font>(1)/(3))+((2)/(5));
	(8)	(<font face=symbol>-</font>3(1)/(4))+(<font face=symbol>-</font>1(1)/(12)).
2.计算:
(1) (<font face=symbol>-</font>8)+10+2+(<font face=symbol>-</font>1);
(2) 5+(<font face=symbol>-</font>6)+3+9+(<font face=symbol>-</font>4)+(<font face=symbol>-</font>7);

(3) (-0.8)+1.2+(-0.7)+(-2.1)+0.8+3.5;

(4) (1)/(2)+(-(2)/(3))+(4)/(5)+(-(1)/(2))+(-(1)/(3)).

3. 计算

(1) (-8)-8;

(2) (-8)-(-8);

(3) 8-(-8);

(4) 8-8;

(5) 0-6;

(6) 0-(-6);

(7) 16-47;

(8) 28-(-74);

(9) (-3.8)-(+7);

(10) (-5.9)-(-6.1).

4. 计算

(1) (+(2)/(5)-(-(3)/(5)));

(2) (-(2)/(5))-(-(3)/(5));

(3) (1)/(2)-(1)/(3);

(4) (-(1)/(2))-(1)/(3);

(5) -(2)/(3)-(-(1)/(6));

(6) 0-(-(3)/(4));

(7) (-2)-(+(2)/(3));

(8) -16(3)/(4)-(-10(1)/(4))-(+1(1)/(2)).

5. 计算

(1) -4.2+5.7-8.4+10

(2) -(1)/(4)+(5)/(6)+(2)/(3)-(1)/(2);

(3) 12-(-18)+(-7)-15;

(4) 4.7-(-8.9)-7.5+(-6);

(5) (-4(7)/(8))-(-5(1)/(2))+(-4(1)/(4))-(+3(1)/(8));

(6) -(2)/(3)+
</td>
<td style="border-left:1 solid black;">&nbsp;
</td>
<td>
  0-5(1)/(6)
</td>
<td style="border-right:1 solid black;">&nbsp;
</td>
<td>
  +
</td>
<td style="border-left:1 solid black;">&nbsp;
</td>
<td>
  -4(5)/(6)
</td>
<td style="border-right:1 solid black;">&nbsp;
</td>
<td>
  +(-9(1)/(3)).

综合运用

6. 如图,陆上最高处是珠穆朗玛峰的峰顶(8844.43m), 最低处位于亚洲西部名为死海的湖(-415m),两处高度相差多少?



\enddocument</body>
</html>
