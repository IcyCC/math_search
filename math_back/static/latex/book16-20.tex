<html>
<head>
<title>LaTeX4Web 1.4 OUTPUT</title>
<style type="text/css">
<!--
 body {color: black;  background:"#FFCC99";  }
 div.p { margin-top: 7pt;}
 td div.comp { margin-top: -0.6ex; margin-bottom: -1ex;}
 td div.comb { margin-top: -0.6ex; margin-bottom: -.6ex;}
 td div.norm {line-height:normal;}
 td div.hrcomp { line-height: 0.9; margin-top: -0.8ex; margin-bottom: -1ex;}
 td.sqrt {border-top:2 solid black;
          border-left:2 solid black;
          border-bottom:none;
          border-right:none;}
 table.sqrt {border-top:2 solid black;
             border-left:2 solid black;
             border-bottom:none;
             border-right:none;}
-->
</style>
</head>
<body>
\documentclassarticle
\usepackage[utf8]ctex

\begindocument

\maketitle

<p><a name="toc.1"><h1>1&nbsp;1.3 有理数的加减法</h1>

<h3>1.3.1 有理数的加法</h3>

在小学,我们学过正数及0的加法运算引入负数后,怎样进行加法运算呢?实际问题中,有时也会遇到与负数有关的加法运算例如,在本章引言中, 把收入记作正数,支出记作负数,在求“结余”时,需要计算8.5+(<font face=symbol>-</font>4.5),4+(<font face=symbol>-</font>5.2)等.<br>



小学学过的加法是正数与正数相加、正数与0相加。引入负数后,加法有哪几种情况?<br>


引人负数后,除已有的正数与正数相加、正数与0相加外,还有负数与负;数相加、负数与正数相加、负数与0相加等。下面借助具体情境和数轴来讨论有理数的加法,<br>

看下面的问题<br>

一个物体作左右方向的运动,我们规定向左为负,向右为正,向有运动5m记作5m,向左运动5m记作<font face=symbol>-</font>5m.<br>



思考<br>


如果物体先向右运动5, 再向右运动3m, 那么两次运动的最后结果是什么?可以用怎样的算式表示?<br>


两次运动后物体从起点向右运动了8m.写成算式就是5+3=8.<br>


将物体的运动起点放在原点,则这个算式可用数轴表示为图1.3-1.<br>




思考<br>


如果物体先向左运动5m,再向左运动3m,那么两次运动的最后结果是什么?可以用怎样的算式表示? <br>


两次运动后物体从起点向左运动了8m.写成算式就是(<font face=symbol>-</font>5)+(<font face=symbol>-</font>3)=<font face=symbol>-</font> 8.<br>


这个运算也可以用数轴表示,其中假设原点O为运动起点(图1.3-2).<br>


从算武①②可以看出:符号相同的两个数相加,结果的符号不变,绝对值相加<br>


探究<br>


(1)如果物体先向左运动3m,再向右运动5m,那么两次运动的最后结果怎样?如何用算式表示?<br>
 
(2)如果物体先向右运动3m,再向左运动5m.那么两次运动的最后结果怎样?如何用算式表示?<br>


(1)结果是物体从起点向右运动了2m.写成算式就是(<font face=symbol>-</font>3)+5=2.<br>

(2)结果是物体从起点向左运动了2m.写成算式就是3+(<font face=symbol>-</font>5)=<font face=symbol>-</font>2.<br>


从算式③①可以看出:符号相反的两个数相加,结果的符号与绝对值较大的加数的符号相同,并用较大的绝对值减去较小的绝对值.<br>


探究<br>


如果物体先向右运动5m.再向左运动5m,那么两次运动的最后结果如何?<br>


结果是仍在起点处,写成算式就是5+(<font face=symbol>-</font>5)=0.<br>


算式⑤表明,互为相反数的两个数相加,结果为0.<br>


如果物体第1s向右(或左)运动5m,第2s原地不动,2s后物体从起点向右(或左)运动了5m.写成算式就是<br>


5+0=5 (或(<font face=symbol>-</font>5)+0= <font face=symbol>-</font>5).  ⑥<br>


思考<br>


从算式⑥可以得出什么结论?<br>


从算式①&nbsp;⑥可知,有理数加法运算中,既要考虑符号,又要考虑绝对值你能从这些算式中归纳出有理数加法的运算法则吗?<br>


\beginpropertory<br>

有理数加法法则:<br>


1.同号两数相加,取相同的符号,并把绝对值相加<br>


2.绝对值不相等的异号两数相加,取绝对值较大的加数的符号,并用较大的绝对值减去较小的绝对值互为相反数的两个数相加得0.<br>


3.一个数同0相加,仍得这个数<br>

\endpropertory<br>


\beginexample
    
    例1计算:<br>

    
    (1) (<font face=symbol>-</font>3)+(<font face=symbol>-</font>9);<br>

    
    (2) (<font face=symbol>-</font>4.7)+3.9.<br>

    
    解: <br>

    (1) (<font face=symbol>-</font>3)+(<font face=symbol>-</font>9)=<font face=symbol>-</font>(3+9)=<font face=symbol>-</font>12;<br>


    (2) (<font face=symbol>-</font>4.7)+3.9=<font face=symbol>-</font>(4.7<font face=symbol>-</font>3.9)=<font face=symbol>-</font>0.8.<br>


\endexample
    
\beginexercise
    
    练习<br>

    
    1.用算式表示下面的结果:<br>

    
    (1)温度由一4C上升7C;<br>

    (2)收入7元,又支出5元,<br>

    
    2.口算:<br>


    (1) (<font face=symbol>-</font>4)+(<font face=symbol>-</font>6);<br>

    
    (2) 4+(<font face=symbol>-</font> 6);<br>

    
    (3) (<font face=symbol>-</font>4)+6;<br>


    (4) (<font face=symbol>-</font>4)+4;<br>

    
    (5) (<font face=symbol>-</font>4)+14;<br>

    
    (6) (<font face=symbol>-</font>14)+4;<br>

    
    (7) 6+(<font face=symbol>-</font>6);<br>


    (8) 0+(<font face=symbol>-</font>6).<br>

    
    
    3.计算:<br>


    (1) 15+(<font face=symbol>-</font>22);<br>

    
    (2) (<font face=symbol>-</font>13)+(<font face=symbol>-</font>8);<br>

    
    (3) (<font face=symbol>-</font>0.9)+1.5;<br>


    (4) (1)/(2)+(<font face=symbol>-</font>(2)/(3))<br>


    4.请你用生活实例解释5+(<font face=symbol>-</font>3)=2, (<font face=symbol>-</font>5)+(<font face=symbol>-</font>3)=<font face=symbol>-</font>8的意义,<br>

    
\endexercise

    我们以前学过加法交换律、结合律,在有理数的加法中它们还适用吗?<br>

    
    探究<br>

    
    计算<br>

    
    30+(<font face=symbol>-</font>20), (<font face=symbol>-</font>20)+30.<br>

    
    两次所得的和相同吗?换几个加数再试一试。<br>

    
    
    从上述计算中,你能得出什么结论? <br>

    
\beginproperty
    
有理数的加法中,两个数相加,交换加数的位置,和不变<br>


加法交换律: a+b=b+a.<br>


\endproperty
    
    探究<br>

    
    计算<br>


[8+(<font face=symbol>-</font>5)]+(<font face=symbol>-</font>4), 8+[(<font face=symbol>-</font>5)+(<font face=symbol>-</font>4)].两次所得的和相同吗?换几个加数再试一试.<br>


从上述计算中,你能得出什么结论? <br>


\beginproperty
    
有理数的加法中,三个数相加,先把前两个数相加,或者先把后两个数相加,和不变<br>

加法结合律: (a+b)+c=a+(b+c).<br>


\endproperty

\beginexample
    
    例2
    
    计算 16+(<font face=symbol>-</font>25)+24+(<font face=symbol>-</font>35).<br>

    
    解: 16+(<font face=symbol>-</font>25)+24+(<font face=symbol>-</font>35)=16+24+[(<font face=symbol>-</font>25)+(<font face=symbol>-</font>35)]=40+(<font face=symbol>-</font>60)=<font face=symbol>-</font>20.<br>

    
    例2中是怎祥使计算简化的?根据是什么?<br>

    
    利用加法交换律、结合律,可以使运算简化认识运算律对于理解运算有很重要的意义。<br>

    
    例3<br>

    
    10袋小麦称后记录如图1.3-3所示(单位: kg). 10袋小麦一共多少千克?如果每袋小麦以90 kg为标准。10 袋小麦总计超过多少千克或不足多少千克?<br>

    
    解法1:<br>

    
    先计算10袋小麦-共多少千克:<br>


    91+91+91.5+89+91.2+91.3+88.7+88.8+91.8+91.1=905.4.<br>

    
    再计算总计超过多少千克:<br>

    
    905.4<font face=symbol>-</font>90&times;10=5.4.<br>


    解法2:<br>

    每袋小麦超过90 kg的千克数记作正数,不足的千克数记作负数.10袋小麦对应的数分别为+1, +1, +1.5, <font face=symbol>-</font>1, +1.2, +1.3, <font face=symbol>-</font>1.3, <font face=symbol>-</font> 1.2, +1.8, +1.1.<br>


    1+1+1.5+(<font face=symbol>-</font>1)+1.2+1.3+(<font face=symbol>-</font>1.3)+(<font face=symbol>-</font>1.2)+1.8+1.1=[1+(<font face=symbol>-</font>1)]+[1.2+(<font face=symbol>-</font>1.2)]+[1.3+(<font face=symbol>-</font> 1.3)]+(1+1.5+1.8+1.1)=5.4.<br>

    
    90X10+5.4=905.4.<br>


    答: 10袋小麦一共905.4kg, 总计超过5.4kg.<br>

    
\endexample

\beginexercise
    
    练习<br>


    1.计算:<br>


    (1) 23+(<font face=symbol>-</font>17)+6+(<font face=symbol>-</font>22);<br>

    
    (2) (<font face=symbol>-</font>2)+3+1+(<font face=symbol>-</font>3)+2+(<font face=symbol>-</font>4).<br>


    2.计算:<br>


    (1) 1+(<font face=symbol>-</font>(1)/(2))+(1)/(3)+(<font face=symbol>-</font>(1)/(6))<br>

    
    (2) 3(1)/(4)+(<font face=symbol>-</font>2(3)/(5))+5(3)/(4)<font face=symbol>-</font>8(2)/(5).<br>

    
\endexercise
    
    
    
    \enddocument
<hr>
<p><h1>Table Of Contents</h1>
<p><a href="#toc.1"><h1>1&nbsp;1.3 有理数的加减法</h1></a>
</body>
</html>
