<html>
<head>
<title>LaTeX4Web 1.4 OUTPUT</title>
<style type="text/css">
<!--
 body {color: black;  background:"#FFCC99";  }
 div.p { margin-top: 7pt;}
 td div.comp { margin-top: -0.6ex; margin-bottom: -1ex;}
 td div.comb { margin-top: -0.6ex; margin-bottom: -.6ex;}
 td div.norm {line-height:normal;}
 td div.hrcomp { line-height: 0.9; margin-top: -0.8ex; margin-bottom: -1ex;}
 td.sqrt {border-top:2 solid black;
          border-left:2 solid black;
          border-bottom:none;
          border-right:none;}
 table.sqrt {border-top:2 solid black;
             border-left:2 solid black;
             border-bottom:none;
             border-right:none;}
-->
</style>
</head>
<body>
\documentclassctexart
\title56-60
\begindocument
\maketitle
\beginarticle
\beginex
1.某种商品每袋4.8元,在一个月内的销售量是m 袋,用式子表示在这个月内销售这种商品的收入。<br>

2.圆柱体的底面半径,高分别是r,h,用式子表示圆柱体的体积。<br>

3.有两片棉田,一片有hm<sup>2</sup>
</td>
<td nowrap align=center>
  (公顷,1hm<sup>2</sup>
</td>
<td nowrap align=center>
  =10<sup>4</sup>
</td>
<td nowrap align=center>
  m<sup>2</sup>
</td>
<td nowrap align=center>
  ),平均每公顷产棉花akg,另一片有nhm<sup>2</sup>
</td>
<td nowrap align=center>
  ,平均每公顷产棉花bkg,用式子表示两片棉花田上棉花的总产量。<br>

4.在一个大正方形铁片中挖去一个小正方形铁片,大正方形的边长是amm,小正方形的边长是bmm,用式子表示剩余部分的面积。<br>

\endex
思考<br>

我们来看引言与例一中的式子<br>

100t,0.8p,mn,a<sup>2</sup>
</td>
<td nowrap align=center>
  h,-n,<br>

这些式子有什么特点?<br>

\beginconcept
这些式子都是数或字母的积,像这样的式子叫做单项式(monomial),单独的一个数或一个字母也是单项式。<br>

单项式中的数字因数叫做这个单项式的系数(coeffcient).例如,单项式100t,a<sup>2</sup>
</td>
<td nowrap align=center>
  h,-n的系数分别是100,1,-1,单项式表示数与字母相乘时,通常把数写在前面。<br>

一个单项式中,所有字母的指数的和叫做这个单项式的次数(degree of a monomial).例如,在单项式100t中,字母t的指数是1,100t的次数是1;在单项式a<sup>2</sup>
</td>
<td nowrap align=center>
  h中,字母a与h 的指数的和是3,a<sup>2</sup>
</td>
<td nowrap align=center>
  h的次数是3.<br>

\endconcept
\beginexample
例3 用单项式填空,并指出他们的系数和次数:<br>

(1)每包书有12册,n包书有\_册;<br>

(2)底边长为acm,高为hcm的三角形的面积是\_cm<sup>2</sup>
</td>
<td nowrap align=center>
  ;<br>

(3)棱长为acm的正方体的体积是\_cm<sup>2</sup>
</td>
<td nowrap align=center>
  ;<br>

(4)一台电视机原价b元,现按原价的9折出售,这台电视机现在的售价是\_元;<br>

\endexample
\endarticle

\beginarticle

\beginexample
(5)一个长方形的长是0.9m,宽是bm,这个长方形的面积是_m<sup>2</sup>
</td>
<td nowrap align=center>
  .<br>

解:(1)12n,他的系数是12,次数是1;<br>

(2)
<table cellspacing=0  border=0 align=center>
<tr>
  <td nowrap align=center>
    1/2ah
  </td>
</tr>
</table>
,他的系数是
<table cellspacing=0  border=0 align=center>
<tr>
  <td nowrap align=center>
    1/2
  </td>
</tr>
</table>
,次数是2;<br>

(3)a<sup>3</sup>
</td>
<td nowrap align=center>
  ,他的系数是1,次数是3;<br>

(4)0.9b,他的系数是0.9,次数是1;<br>

(5)0.9b,他的系数是0.9,次数是1;<br>


\endexample
用字母表示数后,同一个式子可以表示不同的含义。例如,在例三的第(4),(5)小题中,0.9b既可以表示电视机的售价,又可以表示长方形的面积,当然它还可以表示更多的含义,你能赋予0.9b一个含义吗?<br>

\beginex
1填表<br>

\begintabular|c|c|c|c|c|c|
\hline  单项式&2a<sup>2</sup>
</td>
<td nowrap align=center>
  &-1.2h& xy<sup>2</sup>
</td>
<td nowrap align=center>
  & <font face=symbol>-</font>t<sup>2</sup>
</td>
<td nowrap align=center>
  &(<font face=symbol>-</font>2vt)/(3)<br>

\hline
系数&&&&&<br>

\hline
次数&&&&&<br>

\hline
  \endtabular
2填空:<br>

(1)全校学生总数是x,其中女生占总数的48\%,则女生人数是\_,男生人数是\_;<br>

(2)一辆长途汽车从杨柳村出发,3h后到达距出发地skm的溪河镇,这辆长途汽车的平均速度是\_km/h;<br>

(3)产量由mkg增长10\%,就达到\_kg.<br>


\endex
思考<br>

我们来看例二中的式子<br>

v+2.5,v-2.5,3x+5y+2z,1/2ab-<font face=symbol>p</font>r<sup>2</sup>
</td>
<td nowrap align=center>
  ,x<sup>2</sup>
</td>
<td nowrap align=center>
  +2x+18,<br>

这些式子有什么特点?<br>

\endarticle

\beginarticle
\usepackagegraphicx
这些式子都可以看做几个单项式的和。例如,v-2.5可以看作是v与-2.5的和;x<sup>2</sup>
</td>
<td nowrap align=center>
  +2x+18 可以看作单项式x<sup>2</sup>
</td>
<td nowrap align=center>
  ,2x与18的和。<br>

\beginconcept
像这样,几个单项式的和叫做多项式(polynomial).其中,每个单项式叫做多项式的项(term),不含字母的项叫作常数项(constant term).例如,多项式v-2.5的项是v与-2.5,其中-2.5是常数项;多项式x<sup>2</sup>
</td>
<td nowrap align=center>
  +2x+18的项是x<sup>2</sup>
</td>
<td nowrap align=center>
  ,2x和18,其中18是常数项。<br>

多项式里,次数最高的项数,叫做这个多项式的次数(degree of a polynomial).例如,多项式v-2.5中次数最高项是一次项v,这个多项式的次数是1;多项式x<sup>2</sup>
</td>
<td nowrap align=center>
  +2x+18中次数最高项是二次项x<sup>2</sup>
</td>
<td nowrap align=center>
  ,这个多项式的次数是2.<br>

单项式与多项式统称整式(intergral expression).例如,上面见到的单项式100t,0.8p,mn,a<sup>2</sup>
</td>
<td nowrap align=center>
  h,-n,以及多项式v+2.5,v-2.5,3x+5y+2z,1/2ab-<font face=symbol>p</font>r<sup>2</sup>
</td>
<td nowrap align=center>
  ,x<sup>2</sup>
</td>
<td nowrap align=center>
  +2x+18等都是整式。
\endconcept
\beginexample
例4 如图2.1-3,用式子表示圆环的面积,当R=15cm,r=10cm时,求圆环的面积(<font face=symbol>p</font>取3.14).<br>

 <font face=symbol>Î</font> cludegraphics1.png
解:外圆的面积减去内圆的面积就是圆环的面积,所以圆环的面积是<font face=symbol>p</font>R<sup>2</sup>
</td>
<td nowrap align=center>
  <font face=symbol>-</font><font face=symbol>p</font>r<sup>2</sup>
</td>
<td nowrap align=center>
  .<br>

当R=15cm,r=10cm时,圆环的面积(单位:cm<sup>2</sup>
</td>
<td nowrap align=center>
  )是<font face=symbol>p</font>R<sup>2</sup>
</td>
<td nowrap align=center>
  -<font face=symbol>p</font>r<sup>2</sup>
</td>
<td nowrap align=center>
  =3.14&times;15<sup>2</sup>
</td>
<td nowrap align=center>
  -3.14&times;10<sup>2</sup>
</td>
<td nowrap align=center>
  =392.5.<br>

这个圆环的面积是392.5cm<sup>2</sup>
</td>
<td nowrap align=center>
  .<br>

\endexample
\beginex
1填空:
(1)a,b分别表示长方形的长和宽,则长方形的周长l=\_,面积S=\_,当a=2cm,b=3cm时,l=\_cm,S=\_cm<sup>2</sup>
</td>
<td nowrap align=center>
  ;<br>

(2)a,b分别表示梯形的上底和下底,h表示梯形的高,则梯形的面积S=\_,当a=2cm,b=4cm,h=5cm时,S=\_cm<sup>2</sup>.

\endex
\endarticle
\beginarticle
\usepackagegraphicx
\beginex
2.用整形填空,指出单项式的次数以及多项式的次数和项:<br>

(1)每袋大米5kg,x袋大米()kg;
(2)如图(图中长度单位:m),阴影部分的面积时()m<sup>2</sup>;
 <font face=symbol>Î</font> cludegraphics2.png<br>

(3)体重由xkg增加2kg后时()kg.<br>

\endex
\beginex
复习巩固<br>

1列示表示:
(1)棱长为acm的正方体的表面积。<br>

(2)每件a元的上衣,降价20\%后的售价是多少元?<br>

(3)一辆汽车的行驶速度时vkm/h,th行驶多少千米?<br>

(4)长方形绿地的长,宽分别时am,bm,如果长增加xm,新增的绿地面积时多少平方米?<br>

2列式表示<br>

(1)温度由t<sup><font face=symbol>°</font></sup>
</td>
<td nowrap align=center>
  C上升5<sup><font face=symbol>°</font></sup>
</td>
<td nowrap align=center>
  C后是多少?<br>

(2)两车同时,同地,同向出发,快车行驶速度时xkm/h,慢车行驶速度时ykm/h,3h后辆车相距多少千米?<br>

(3)某种苹果的售价时每千克x元(x<10),用50 元买5kg这种苹果,应找回多少钱?<br>

(4)如图(途中长度单位:cm),钢管的体积是多少?
 <font face=symbol>Î</font> cludegraphics3.png<br>

3填表:
\begintabular|c|c|c|c|c|c|
\hline  整式&-15ab&4a<sup>2</sup>
</td>
<td nowrap align=center>
  b<sup>2</sup>
</td>
<td nowrap align=center>
  & (3x<sup>2</sup>
</td>
<td nowrap align=center>
  y)/(5)& 4x<sup>2</sup>
</td>
<td nowrap align=center>
  <font face=symbol>-</font>3&a<sup>4</sup>
</td>
<td nowrap align=center>
  <font face=symbol>-</font>2a<sup>2</sup>
</td>
<td nowrap align=center>
  b<sup>2</sup>
</td>
<td nowrap align=center>
  +b<sup>4</sup>
</td>
<td nowrap align=center>
  <br>

\hline
系数&&&&&<br>

\hline
次数&&&&&<br>

\hline
项&&&&&<br>

\hline
  \endtabular
综合运用
4.测得一种树苗的高度与树苗生长的年数的有关数据如下页表(树苗原高100cm);
\endex
\endarticle
\beginarticle
\usepackagegraphicx
\beginex
\begintabular|c|c|
\hline  年数&高度/cm<br>

\hline
1&100+5<br>

\hline
2&100+10<br>

\hline
3&100+15<br>

\hline
4&100+20<br>

\hline
.....&...<br>

\hline
  \endtabular
前四年树苗高度的变化与年数有什么关系?假设以后各年树苗高度的变化与年数保持上述关系,用式子表示生长了n 年的树苗的高度.<br>

5.礼堂第一排有a个座位,后面每排都比前一排多一个座位,前2排有多少个座位?第三排呢?用式子表示第n排的座位数。如果第一排有20个座位,计算第19排的座位数。<br>

6.一块三角尺的形状和尺寸如图所示。如果圆孔的半径是r,三角尺的厚度是h,用式子表示折块三角尺的体积V。若a=6cm,r=0.5cm,h=0.2cm,求V的值。(<font face=symbol>p</font> 取3)。
 <font face=symbol>Î</font> cludegraphics4.png<br>

拓广探索<br>

7.设n表示任意一个整数,用含n的式子表示<br>

(1)任意一个偶数;(2)任意一个奇数<br>

8.3个球队进行单循环比赛(参加比赛的每一队都与其他所有的队各赛一场),总的比赛场数是多少?4个队呢?5个队呢?n个队呢?<br>

9.对于密码L dp vwxghaw,你能看出它代表什么意思吗?如果给你一把破译它的钥匙x-3,联想英语字母表中字母的顺序,你再试试能不能解读他,英语字母表中字母是按一下顺序排列的:<br>

abcdefghiklmnopqrstuvwxyz<br>

如果规定a又接在z的后面,使26个字母排成圈,并能联想到x-3可以代表“把一个字母换成字母表中从他向前移动3
位的字母”,按这个规律就有<br>

L dp vwxghaw <font face=symbol>®</font> I am a student<br>

这样你就能解读它的意思了。
为了保密,许多情况下都要采用密码,这时候就需要有破译密码的“钥匙".上面的例子中,如果写和读密码的双方事先约定了作为”钥匙“的式子x-3的含义,那么他们就可以用一种保密方式通信了,你和同伴不妨也利用数学式子来制定一种类似的”钥匙",并互相合作,通过游戏试试如何保密通信。
 <font face=symbol>Î</font> cludegraphics5.png<br>

\endex
\endarticle
\enddocument
</body>
</html>
