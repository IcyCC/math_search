<html>
<head>
<title>LaTeX4Web 1.4 OUTPUT</title>
<style type="text/css">
<!--
 body {color: black;  background:"#FFCC99";  }
 div.p { margin-top: 7pt;}
 td div.comp { margin-top: -0.6ex; margin-bottom: -1ex;}
 td div.comb { margin-top: -0.6ex; margin-bottom: -.6ex;}
 td div.norm {line-height:normal;}
 td div.hrcomp { line-height: 0.9; margin-top: -0.8ex; margin-bottom: -1ex;}
 td.sqrt {border-top:2 solid black;
          border-left:2 solid black;
          border-bottom:none;
          border-right:none;}
 table.sqrt {border-top:2 solid black;
             border-left:2 solid black;
             border-bottom:none;
             border-right:none;}
-->
</style>
</head>
<body>
\documentclassarticle
\usepackage[utf8]ctex

\begindocument


观察与猜想

翻牌游戏中的数学道理

桌上有9米正西向上的扑克牌,每次自动其中任意2张(包括已翻过的牌)。使它们从一面向上变为另一面向上,这样一直做下去,观察能否使所有的牌都反面向上?

你不妨动手试一试,看看会不会出现所有牌都反面向上。

事实上,不论你翻多少次,都不能使9张牌都反面向上,从这个结果,你能想到其中的数学道理吗?

如果在每张牌的正西都写1.反面都写-1.考虑所有牌朝上一面的数的积开始9张牌都正面向上,上面的数的积是1.每次翻动2来。就是说有2张牌同时改变符号,这能改变朝上一面的数的积是1这一结果吗? 9张牌都反面向上时,上面的数的积是什么数?这种现象为什么不能出现?

你能解释为什么不会使9张牌都反面向上了吗?如果桌上有任意奇数张牌,猜想结果会是怎样?



<p><a name="toc.0.1"><h2>0.1&nbsp;1.5有理数的乘方</h2>
<p><a name="toc.0.1.1"><h3>0.1.1&nbsp;1.5.1 乘方</h3>

前面学了有理数的乘法,下面研究各个乘数都相同时的乘法运算,

我们知道,边长为2cm的正方形的面积是2&times;2=4(cm<sup>2</sup>); 棱长为2cm的正方体的体积是2&times;2&times;2=8(cm<sup>2</sup>).

2&times;2, 2&times;2&times;2都是相同因数的乘法.

为了简便,我们将它们分别记作2<sup>2</sup>, 2<sup>3</sup>. 2<sup>2</sup>读作“2的平方”(或“2的二次方"),2<sup>3</sup>读作“2的立方”(或“2的三次方").

同样:
(<font face=symbol>-</font>2)&times;(<font face=symbol>-</font>2)&times;(<font face=symbol>-</font>2)&times;(<font face=symbol>-</font>2)记作(<font face=symbol>-</font>2)<sup>4</sup>,读作“-2的四次方";(<font face=symbol>-</font>(2)/(5))&times;(<font face=symbol>-</font>(2)/(5))&times;(<font face=symbol>-</font>(2)/(5))&times;(<font face=symbol>-</font>(2)/(5))&times;(<font face=symbol>-</font>(2)/(5))记作(<font face=symbol>-</font>(2)/(5))<sup>5</sup>,读作“<font face=symbol>-</font>(2)/(5)的五次方”。
一般地,n个相同的因数a相乘,即a<font face=symbol>·</font> a<font face=symbol>·</font>&#183;s<font face=symbol>·</font> a<sub>\textn</sub> 记作a<sup>n</sup>",读作“a的n次方”。

\begindefinition
求n个相同因数的积的运算,叫做乘方,乘方  指数的结果叫做幂(power). 在a<sup>n</sup>中,a叫做底数(base number), n叫做指数(exponent), 当a<sup>n</sup>看作a的n次方的结果时,也可读作“a的n次幂"。
\enddefinition

例如,在9<sup>4</sup>中,底数是9,指数是4, 9<sup>4</sup>读作“9的4次方",或“9的4次幂".

一个数可以看作这个数本身的一次方例如,5就是5<sup>1</sup>.指数1通常省略不写。

因为a<sup>n</sup>就是n个a相乘,所以可以利用有理数的乘法运算来进行有理数的乘方运算.

\beginexample
<br>agraph例1计算:
(1) (<font face=symbol>-</font>4)<sup>3</sup>;  (2) (<font face=symbol>-</font>2)<sup>4</sup>;  (3) (<font face=symbol>-</font>(2)/(3))<sup>3</sup>.
<br>agraph解:
(1) (<font face=symbol>-</font>4)<sup>3</sup>=(<font face=symbol>-</font>4)&times;(<font face=symbol>-</font>4)&times;(<font face=symbol>-</font>4)=<font face=symbol>-</font>64;
(2) (<font face=symbol>-</font>2)<sup>4</sup>=(<font face=symbol>-</font>2)&times;(<font face=symbol>-</font>2)&times;(<font face=symbol>-</font>2)&times;(<font face=symbol>-</font>2)=16;
(3) (<font face=symbol>-</font>(2)/(3))<sup>3</sup>=(<font face=symbol>-</font>(2)/(3))&times;(<font face=symbol>-</font>(2)/(3))&times;(<font face=symbol>-</font>(2)/(3))=<font face=symbol>-</font>(8)/(27).
\endexample

\beginexercise
<br>agraph思考
从例1,你发现负数的暴的正负有什么规律?

当指数是___数时,负数的幂是___数;

当指数是___数时, 负数的幂是__数.
\endexercise

\beginprop
根据有理数的乘法法则可以得出:

负数的奇次幂是负数,负数的偶次幕是正数

显然,正数的任何次幂都是正数,0的任何正整数次幂都是0.
\endprop

\beginexample
例2 用计算器计算(<font face=symbol>-</font>8)<sup>5</sup>和(<font face=symbol>-</font>3)<sup>4</sup>.

解:用带符号键"-"的计算器.

((-)8)<sup>5</sup>=

显示:(-8)<sup>5</sup>

-32768

((-)3)<sup>6</sup>=

显示:(-3)<sup>6</sup>

729.

所以(<font face=symbol>-</font>8)<sup>5</sup>=<font face=symbol>-</font> 32768, (<font face=symbol>-</font>3)<sup>4</sup>=729.

\endexample

\beginexercise
练习

1. (1) (<font face=symbol>-</font>7)<sup>8</sup>中,底数、指数各是什么?

(2) (<font face=symbol>-</font>10)<sup>8</sup>中-10叫做什么数? 8叫做什么数? (<font face=symbol>-</font>10)<sup>4</sup>是正数还是负数?

2.计算:

(1) (<font face=symbol>-</font>1)<sup>1</sup>0;  (2) (<font face=symbol>-</font>1)<sup>7</sup>:  (3) 8<sup>3</sup>;  (4) (<font face=symbol>-</font>5)<sup>3</sup>;

(5) 0.1<sup>3</sup>;    (6) (<font face=symbol>-</font>(1)/(2))<sup>6</sup>; (7) (<font face=symbol>-</font>10)<sup>4</sup>; (8) (<font face=symbol>-</font>10)<sup>5</sup>;
\endexercise

做有理数的混合运算时,应注意以下运算顺序:

1.先乘方,再乘除,最后加减:

2.同级运算,从左到右进行

3.如有括号,先做括号内的运算,按小括号、中话号、大括号依次进行.

\beginexample

例3 计算:

(1) 2&times;(<font face=symbol>-</font>3)<sup>3</sup><font face=symbol>-</font>4&times;(<font face=symbol>-</font>3)+15;

(2) (<font face=symbol>-</font>2)<sup>3</sup>+(<font face=symbol>-</font>3)&times;[(<font face=symbol>-</font>4)<sup>2</sup>+2]<font face=symbol>-</font>(<font face=symbol>-</font>3)<sup>2</sup><font face=symbol>¸</font>(<font face=symbol>-</font>2).

解: (1)原式\beginalign
        &=2&times;(-27)-(-12)+15 <br>

        &=-54+12+15     <br>

        &=-27;
\endalign

(2)原式\beginalign
    &=-8+(-3)&times;(16+2)-9+(-2) <br>

    &=-8+(-3)&times;18-(-4.5) <br>

    &=-8-54+4.5 <br>

    &=-57.5.
    \endalign

例4 观察下面三行数:

-2, 4, -8, 16, -32, 64, \dots ①

0, 6, -6, 18, -30, 66, \dots  ②

-1, 2, -4, 8,-16, 32, \dots  ③

(1)第①行数按什么规律排列?

(2)第②③行数与第①行数分别有什么关系?

(3)取每行数的第10个数,计算这三个数的和

分析:观察①,发现各数均为2的倍数.联系数的乘方,从符号和绝对值两方面考虑,可发现持列的规律.

解: (1)第①行数是

-2, (<font face=symbol>-</font>2)<sup>2</sup>, (<font face=symbol>-</font>2)<sup>3</sup>, (<font face=symbol>-</font>2)<sup>4</sup>, \dots.

(2)对比①②两行中位置对应的数,可以发现:

第②行数是第①行相应的数加2,即

-2+2, (<font face=symbol>-</font>2)<sup>2</sup>+2,  (<font face=symbol>-</font>2)<sup>2</sup>+2, (<font face=symbol>-</font>2)<sup>4</sup>+2, \dots;

对比①③两行中位置对应的数,可以发现:

第③行数是第①行相应的数的0.5倍,即

<font face=symbol>-</font>2&times;0.5, (<font face=symbol>-</font>2)<sup>2</sup>&times;0.5, (<font face=symbol>-</font>2)<sup>3</sup>&times;0.5, (<font face=symbol>-</font>2)<sup>4</sup>&times;0.5, \dots.

(3)每行数中的第10个数的和是

\beginalign
    &(-2)<sup>10</sup>
</td>
<td nowrap align=center>
  +[(-2)<sup>10</sup>
</td>
<td nowrap align=center>
  +2]+(-2)<sup>10</sup>
</td>
<td nowrap align=center>
  &times;0.5 <br>

    &=1024+(1024+2)+1024&times;0.5 <br>

    &=1024+1026+512 <br>

    &=2562.
\endalign

\beginexercise
    计算:

    (1) (<font face=symbol>-</font>1)<sup>20</sup>
</td>
<td nowrap align=center>
  &times;2+(<font face=symbol>-</font>2)<sup>3</sup><font face=symbol>¸</font>4;

    (2)  (<font face=symbol>-</font>5)<sup>3</sup><font face=symbol>-</font>3&times;(<font face=symbol>-</font>(1)/(2))<sup>4</sup>;

    (3)  (11)/(5)&times;((1)/(3)<font face=symbol>-</font>(1)/(2))&times;(3)/(11)<font face=symbol>¸</font>(5)/(4);
    
    (4)  (<font face=symbol>-</font>10)<sup>4</sup>+[(<font face=symbol>-</font>4)<sup>2</sup><font face=symbol>-</font>(3+3<sup>2</sup>)&times;2] .
\endexercise

<p><a name="toc.0.1.2"><h3>0.1.2&nbsp;1.5.2 科学记数法</h3>

现实中,我们会遇到一些比较大的数。 例如,太阳的半径、光的速度、目前世界人口等。读、写这样大的数有一定困难。
观察10的乘方有如下特点:

10<sup>2</sup>=100,  10<sup>3</sup>=1000 ,  10<sup>4</sup>=10000 , &#183;s.

一般地,10的n次幂等于10\dots 0(在1的后面有n个0),所以可以利用10的乘方表示一些大数,例如

567000000=5.67&times;100000000=5.67&times;10<sup>8</sup>
</td>
<td nowrap align=center>
  

读作“5.67乘10的8次方(幂)”。

这样不仅可以使书写简短,同时还便于读数。

\begindefinition

像上面这样,把一个大于10的数表示成a&times;10<sup>n</sup>
</td>
<td nowrap align=center>
  的形式(其中a大于或等于1且小于10. n是正整数),使用的是科学记数法.

\enddefinition   

对于小于-10的数也可以类似表示,例如:

 <font face=symbol>-</font>567000000=<font face=symbol>-</font>5.67&times;10<sup>8</sup>
</td>
<td nowrap align=center>
   .

\beginexample

例5 用科学记数法表示下列各数:

1000000, 57000000, -123000000000.

解:

1000000=10<sup>6</sup>
</td>
<td nowrap align=center>
  ,

 57000000=5.7&times;10<sup>7</sup>
</td>
<td nowrap align=center>
   ,

 <font face=symbol>-</font>123000000000=<font face=symbol>-</font>1.23&times;10<sup>11</sup>
</td>
<td nowrap align=center>
   .

\endexample

\beginexercise
思考:

上面的式子中,等号左边整数的位数与右边10的指数有什么关系?用科学记数法表示一个n位整数,其中10的指数是__.
\endexercise

\beginexercise
1.用科学记数法写出下列各数:

10000, 800000, 56 000 000, -7400 000.

2.下列用科学记数法写出的数,原来分别是什么数?

1&times;10<sup>7</sup>
</td>
<td nowrap align=center>
  , 4&times;10<sup>9</sup>
</td>
<td nowrap align=center>
  , 8.5&times;10<sup>6</sup>
</td>
<td nowrap align=center>
  , 7.04&times;10<sup>5</sup>
</td>
<td nowrap align=center>
  , <font face=symbol>-</font>3.96&times;10<sup>4</sup>
</td>
<td nowrap align=center>
  .

3.中国的陆地面积约为9600000km<sup>2</sup>
</td>
<td nowrap align=center>
  ,领水面积约为370000km<sup>2</sup>
</td>
<td nowrap align=center>
  ,用科学记数法表示上述两个数字.

\endexercise

<p><a name="toc.0.1.3"><h3>0.1.3&nbsp;1.5.3 近似数</h3>

先看一个例子,对于参加同一个会议的人数,有两个报道。一个报道说:“会议秘书处宣布,参加今天会议的有513人”这里数字513确切地反映了实际人数,它是一个准确数。另一报道说: 

\enddocument
<hr>
<p><h1>Table Of Contents</h1>
<p><a href="#toc.0.1"><h2>0.1&nbsp;1.5有理数的乘方</h2></a>
<p><a href="#toc.0.1.1"><h3>0.1.1&nbsp;1.5.1 乘方</h3></a>
<p><a href="#toc.0.1.2"><h3>0.1.2&nbsp;1.5.2 科学记数法</h3></a>
<p><a href="#toc.0.1.3"><h3>0.1.3&nbsp;1.5.3 近似数</h3></a>
</body>
</html>
