<html>
<head>
<title>LaTeX4Web 1.4 OUTPUT</title>
<style type="text/css">
<!--
 body {color: black;  background:"#FFCC99";  }
 div.p { margin-top: 7pt;}
 td div.comp { margin-top: -0.6ex; margin-bottom: -1ex;}
 td div.comb { margin-top: -0.6ex; margin-bottom: -.6ex;}
 td div.norm {line-height:normal;}
 td div.hrcomp { line-height: 0.9; margin-top: -0.8ex; margin-bottom: -1ex;}
 td.sqrt {border-top:2 solid black;
          border-left:2 solid black;
          border-bottom:none;
          border-right:none;}
 table.sqrt {border-top:2 solid black;
             border-left:2 solid black;
             border-bottom:none;
             border-right:none;}
-->
</style>
</head>
<body>
\documentclassctexart
\newtheoremexampleExample[section]\newtheoremexerciseExercise[section]

\begindocument
\beginexample
即
[28x+21x+6x+42x=1386]
<br>
合并同类项,得
[97x=1386]
<br>
系数化为1,得
[x=(97)/(1386)]
<br>
为更全面地讨论问题,我们再以方程(3x+1)/(2)<font face=symbol>-</font>2=(3x<font face=symbol>-</font>2)/(10)<font face=symbol>-</font>(2x+3)/(5)为例,看看解有分数的一元一次方程的步骤.
这个方程中各分母的最小公倍数是10,方程两边乘10,于是方程左边变为
[10&times;
</td>
<td style="border-left:1 solid black;border-top:1 solid black;border-bottom:1 solid black;">&nbsp;
</td>
<td>
  (3x+1)/(2)-2
</td>
<td style="border-right:1 solid black;border-top:1 solid black;border-bottom:1 solid black;">&nbsp;
</td>
<td>
  =10&times;(3x+1)/(2)-10&times;2=5(3x+1)-10&times;2]
<br>
去了分母,方程右边变为什么?你具体算算.
下面的框图表示了解这个方程的流程.
[(3x+1)/(2)-2=(3x-2)/(10)-(2x+3)/(5)]
<center>去分母(方程两边乘各分母的最小公倍数)</center>
[5(3x+1)-10&times;2=(3x-2)-2(2x+3)]
<center>去括号</center>
[15x+5-20=3x-2-4x-6]
<center>移项</center>
[15x-3x+4x=-2-6-5+20]
<center>合并同类项</center>
[16x=7]
<center>系数化为1</center>
[x=(7)/(16)]
\endexample

\beginexample
归纳
<br>
解一元一次方程的一般步骤包括:去分母、去括号、移项、合并同类项、系数化为1等.通过这些步骤可以使以x为未知数的方程逐步向着x=a的形式转化,这个过程主要依据等式的基本性质和运算律等.

<br>
例3  解下列方程:

(1)(x+1)/(2)<font face=symbol>-</font>1=2+(2<font face=symbol>-</font>x)/(4);
(2)3x+(x<font face=symbol>-</font>1)/(2)=3<font face=symbol>-</font>(2x<font face=symbol>-</font>1)/(3)
<br>
解:(1)去分母(方程两边乘4),得
[2(x+1)-4=8+(2-x)]
去括号,得
[2x+2-4=8+2-x]
移项,得
[2x+x=8+2-2+4]
合并同类项,得
[3x=12]
系数化为1,得
[x=4]
(2)去分母(方程两边乘6),得
[18x+3(x-1)=18-2(2x-1)]
去括号,得
[18x+3x-3=18-4x+2]
移项
[18x+3x+4x=18+2+3]
合并同类项,得
[25x=23]
系数化为1,得
[x=(23)/(25)]
<br>
在本章第一个问题中,我们根据路程、速度和时间三者的关系,列出方程(x)/(60)<font face=symbol>-</font>(x)/(70)=1.现在你一定会了解它了.去分母(方程两边乘420),得7x<font face=symbol>-</font>6x=420,x=420,于是得出两地间的路程为420km.
\endexample

\beginexercise
练习

解下列方程:

(1)(19)/(100)x=(21)/(100)(x<font face=symbol>-</font>2);
(2)(x+1)/(2)<font face=symbol>-</font>2=(x)/(4);

(3)(5x<font face=symbol>-</font>1)/(4)=(3x+1)/(2)<font face=symbol>-</font>(2<font face=symbol>-</font>x)/(3);
(4)(3x+2)/(2)<font face=symbol>-</font>1=(2x<font face=symbol>-</font>1)/(4)<font face=symbol>-</font>(2x+1)/(5);

习题3.3

复习巩固

1.解下列方程:
(1)5a+(2<font face=symbol>-</font>4a)=0;
(2)25b<font face=symbol>-</font>(b<font face=symbol>-</font>5)=29;

(3)7x+2(3x<font face=symbol>-</font>3)=20
(4)8y<font face=symbol>-</font>3(3y+2)=6

2.解下列方程:
(1)2(x+8)=3(x<font face=symbol>-</font>1);
(2)8x=<font face=symbol>-</font>2(x+4);
(3)2x<font face=symbol>-</font>(2)/(3)(x+3)=<font face=symbol>-</font>x+3;
(4)2(10<font face=symbol>-</font>0.5y)=<font face=symbol>-</font>(1.5y+2)

3.解下列方程:
(1)(3x+5)/(2)=(2x<font face=symbol>-</font>1)/(3)
(2)(x<font face=symbol>-</font>3)/(<font face=symbol>-</font>5)=(3x+4)/(15)
(3)(3y<font face=symbol>-</font>1)/(4)<font face=symbol>-</font>1=(5y<font face=symbol>-</font>7)/(6)
(4)(5y+4)/(3)+(y<font face=symbol>-</font>1)/(4)=2<font face=symbol>-</font>(5y<font face=symbol>-</font>5)/(12)

4.用方程解答下列问题:
(1)x与4之和的1.2倍等于x与14之差的3.6倍,求x;
(2)y的3倍与1.5之和的二分之一等于y与1之差的四分之一,求y.

综合运用
5.张华和李明登一座山,张华每分登高10 m,并且先出发30 min(分),李明每分登高15 m,两人同时登上山顶,设张华登山用了x min,如何用含x的式子表示李明登山所用时间?试用方程求x 的值,由x的值能求出山高吗?如果能,山高多少米?

6.两辆汽车从相距298 km的两地同时出发相向而行,甲车的速度比乙车的速度的2倍还快20 km/h, 半小时后两车相遇,两车的速度是多少?

7.在风速为24 km/h的条件下,一架飞机顺丰从A机场飞到B机场要用2.8 h,它逆风飞行同样的航线要用3 h,求:(1)无风时这架飞机在这一航线的平均航速;(2)两机场之间的航程.

8.买两种布料共138m,花了540元,其中蓝布料每米3元,黑布料每米5元,两种布料各买了多少米?

拓广探索


9.有一些相同的房间需要粉刷墙面,一天3名一级技工去粉刷8个房间,结果其中有50m<sup>2</sup>
</td>
<td nowrap align=center>
   墙面未来得及粉刷;同样时间内5名二级技工粉刷了10个房间之外,还多粉刷了另外的40m<sup>2</sup>
</td>
<td nowrap align=center>
  墙面.每名一级技工比二级技工一天多粉刷10m<sup>2</sup>
</td>
<td nowrap align=center>
  墙面,求每个房间需要粉刷的墙面面积.

10.王力骑自行车从A地到B地,陈平骑自行车从B地到A地,两人都沿同一公路匀速前进,已知两人在上车8时同时出发,到上午10时,两人还相距36 km,到中午12时,两人又相距36km,求A,B两地间的路程.

11.一列火车匀速行驶,经过一条长300 m的隧道需要20 s的时间.隧道的顶上有一盏灯,垂直向下发光,灯光照在火车上的时间是10 s.
(1)设火车的长度为x m,用含x的式子表示:从车头经过灯下到车尾经过灯下火车所走的路程和这段时间内火车的平均速度;
(2)设火车的长度为x m,用含x的式子表示:从车头进入隧道到车尾离开隧道火车所走的路程和这段时间内火车的平均逵度
(3)上述问题中火车的平均速度发生了变化吗?
(4)求这列火车的长度
\endexercise


3.4  实际问题与一元一次方程
<br>
从前面的讨论中已经可以看出,方程是分析和解决问题的一种很有用的数学工具.本节我们重点讨论如何用一元一次方程解决实际问题.
\beginexample
<br>
例1  某车间有22名工人,每人每天可以生产1200个螺钉或2000个螺母.1个螺钉需要配2个螺母,为使每天生产的螺钉和螺母刚好配套,应安排生产螺钉和螺母的工人各多少名?
<br>
分析:每天生产的螺母数量是螺钉数量的2倍时,它们好配套
<br>
解:设应安排x名工人生产螺钉,(22-x)名工人生产螺母.
如果设x名工人生产螺母,怎样列方程?
<br>
根据螺母数量应是螺钉数量的2倍,列出方程
[2000(22-x)=2&times;1200x]
<br>
解方程,得
[5(22-x)=6x]
[110-5x=6x]
[11x=110]
[x=10]
[22-x=12]
<br>
答:应安排10名工人生产螺钉,12名工人生产螺母
<br>
例2整理一批图书,由一个人做要40h完成,现计划由一部分人先做1h,然后增加2人与他们一起做8h,完成这项工作.假设这些人的工作效率相同,具体应先安排多少人工作?
<br>
分析:如果把总工作量设为1,则人均效率(一个人1h完成的工作量)
为(1)/(4),x人先做4h完成的工作量为(4x)/(40),增加2人后再做8h完成的工作量为(8(x+2))/(40),这两个工作量之和应等于总工作量.
\endexample

\enddocument

</body>
</html>
