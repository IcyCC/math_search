<html>
<head>
<title>LaTeX4Web 1.4 OUTPUT</title>
<style type="text/css">
<!--
 body {color: black;  background:"#FFCC99";  }
 div.p { margin-top: 7pt;}
 td div.comp { margin-top: -0.6ex; margin-bottom: -1ex;}
 td div.comb { margin-top: -0.6ex; margin-bottom: -.6ex;}
 td div.norm {line-height:normal;}
 td div.hrcomp { line-height: 0.9; margin-top: -0.8ex; margin-bottom: -1ex;}
 td.sqrt {border-top:2 solid black;
          border-left:2 solid black;
          border-bottom:none;
          border-right:none;}
 table.sqrt {border-top:2 solid black;
             border-left:2 solid black;
             border-bottom:none;
             border-right:none;}
-->
</style>
</head>
<body>
\documentclassarticle
\usepackage[utf8]ctex
\usepackageunderscore

\begindocument

\beginexercise
7.一天早晨的气温是 <font face=symbol>-</font>7<sup><font face=symbol>°</font></sup>
</td>
<td nowrap align=center>
  C,中午上升了11<sup><font face=symbol>°</font></sup>
</td>
<td nowrap align=center>
  C,半夜又下降了9<sup><font face=symbol>°</font></sup>
</td>
<td nowrap align=center>
  C,半夜的气温是多少摄氏度?

8.食品店一周中各天的盈亏状况如下(盈余为正):
	132元,<font face=symbol>-</font>12.5元,<font face=symbol>-</font>10.5元,127元,<font face=symbol>-</font>87元,136.5元,98元.
一周总的盈亏状况如何?

9.有8筐白菜,以每筐25kg为准,超过的千克数记作正数,不足的千克数记作负数,称后的记录如下:

1.5,-3,2,-0.5,1,-2,-2,-2.5,

这8筐白菜一共多少千克?

10.某地一周内每天的最高气温与最低气温记录如下表,哪天的昼夜温差最大?哪天的昼夜温差最小?

\begintabular|c|c|c|c|c|c|c|c|
     \hline
     星期&一&二&三&四&五&六&日<br>

     \hline
     最高气温&10<sup><font face=symbol>°</font></sup>
</td>
<td nowrap align=center>
  C&12<sup><font face=symbol>°</font></sup>
</td>
<td nowrap align=center>
  C&11<sup><font face=symbol>°</font></sup>
</td>
<td nowrap align=center>
  C&9<sup><font face=symbol>°</font></sup>
</td>
<td nowrap align=center>
  C&7<sup><font face=symbol>°</font></sup>
</td>
<td nowrap align=center>
  C&5<sup><font face=symbol>°</font></sup>
</td>
<td nowrap align=center>
  C&7<sup><font face=symbol>°</font></sup>
</td>
<td nowrap align=center>
  C<br>

     \hline
     最低气温&2<sup><font face=symbol>°</font></sup>
</td>
<td nowrap align=center>
  C&1<sup><font face=symbol>°</font></sup>
</td>
<td nowrap align=center>
  C&0<sup><font face=symbol>°</font></sup>
</td>
<td nowrap align=center>
  C&-1<sup><font face=symbol>°</font></sup>
</td>
<td nowrap align=center>
  C&-4<sup><font face=symbol>°</font></sup>
</td>
<td nowrap align=center>
  C&-5<sup><font face=symbol>°</font></sup>
</td>
<td nowrap align=center>
  C&-5<sup><font face=symbol>°</font></sup>
</td>
<td nowrap align=center>
  <br>

     \hline
\endtabular

拓广探索

11.填空:

    (1)__+11=27          (2)7+__=4

    (3)(<font face=symbol>-</font>9)+__=9       (4)12+__=0

    (5)(<font face=symbol>-</font>8)+__=<font face=symbol>-</font>15     (6)__+(<font face=symbol>-</font>13)=<font face=symbol>-</font>6

12.计算下列各式的值:

    (<font face=symbol>-</font>2)+(<font face=symbol>-</font>2)                 (<font face=symbol>-</font>2)+(<font face=symbol>-</font>2)+(<font face=symbol>-</font>2)

    (<font face=symbol>-</font>2)+(<font face=symbol>-</font>2)+(<font face=symbol>-</font>2)+(<font face=symbol>-</font>2)       (<font face=symbol>-</font>2)+(<font face=symbol>-</font>2)+(<font face=symbol>-</font>2)+(<font face=symbol>-</font>2)+(<font face=symbol>-</font>2)

猜想下列各式的值:

    (<font face=symbol>-</font>2)&times; 2, (<font face=symbol>-</font>2)&times; 3, (<font face=symbol>-</font>2)&times; 2, (<font face=symbol>-</font>2)&times; 5

你能进一步猜出负数乘正数的法则吗?

13.一种股票第一天的最高价比开盘价高0.3元,最低价比开盘价低0.2元;

第二天的最高价比开盘价高0.2元,最低价比开盘价低0.2元;

第三天的最高价等于开盘价,最低价比开盘价低0.13元,

计算每天最高价和最低价的差,以及这些差的平均值。

股票交易是市场经济中的一种金融活动,它可以促进投资和资金流通。
\endexercise

\beginarticle

\begincenter
    中国人最先使用负数
\endcenter

中国人很早就开始使用负数。著名的中国古代数学著作《九章算术》的“方程”一章,
在世界数学史上首次正式引入负数及其加减乘除运算法则,并给出名为“正负数”的算法,
魏晋时期的数学家刘徽在其著作《九章算术注》中用不同颜色的算筹(小棍形状的计数工具)
分别表示正数和负数(红色为正,黑色为负),

“正负术”是正负数加减法则,其中有一段话是“同名相除,异名相异,正无入负之,负无入正之。”
你知道它的意思吗?其实它就是减法法则,以现代算式为例,可以将这段话解释如下:

“同名相除”,即同号两数相减时,括号前为被减数的符号,括号内为被减数的绝对值加减数的绝对值,例如

(<font face=symbol>-</font>5)<font face=symbol>-</font>(+3)=+(5<font face=symbol>-</font>3),

(<font face=symbol>-</font>5)<font face=symbol>-</font>(<font face=symbol>-</font>3)=<font face=symbol>-</font>(5<font face=symbol>-</font>3),

"异名相异",即异号两数相减时,括号前为被减数的符号,括号内为被减数的绝对值加减数的绝对值,例如

(+5)<font face=symbol>-</font>(<font face=symbol>-</font>3)=+(5+3)

(<font face=symbol>-</font>5)<font face=symbol>-</font>(<font face=symbol>-</font>3)=<font face=symbol>-</font>(5+3)

"正无入负之,负无入正之",即0减正得负,0减负得正,例如

0<font face=symbol>-</font>(+3)=(<font face=symbol>-</font>3)

0<font face=symbol>-</font>(<font face=symbol>-</font>3)=(+3)

史料证明:追溯到两千多年前,中国人已经开始使用负数,并应用到生产和生活中,例如,在古代商业活动中,以收入为正,支出为负;以盈余为正,亏损为负,在古代农业活动中,
以增产为正,减产为负,中国人使用负数在世界上是首创。

\endarticle

\beginarticle

<p><a name="toc.1"><h1>1&nbsp;1.4 有理数的乘除法</h1>

<p><a name="toc.1.1"><h2>1.1&nbsp;1.4.1 有理数的乘法</h2>

我们已经熟悉正数及0的乘法运算。与加法类似,引入负数后,将出现3&times;(<font face=symbol>-</font>3),
(<font face=symbol>-</font>3)&times;3,(<font face=symbol>-</font>3)&times;3这样的乘法,该怎样进行这一类的运算呢?

\beginexample

<p><a name="toc.1.1.1"><h3>1.1.1&nbsp;思考</h3>

观察下面的乘法算式,你能发现什么规律吗?

3&times;3=9,

3&times;2=6,

3&times;1=3,

3&times;0=0,

\endexample

可以发现,上述算式有如下规律:随着后一乘数的逐次递减1,积逐次递减3.

要使这个规律在引入负数后仍然成立,那么应有:

3&times;(<font face=symbol>-</font>1)=<font face=symbol>-</font>3

3&times;(<font face=symbol>-</font>2)=__

3&times;(<font face=symbol>-</font>3)=__

\beginexample

观察下面的算式,你又能发现什么规律?

3&times;3=9

2&times;3=6

1&times;3=3

0&times;3=0

\endexample

可以发现,上述算式有如下规律,随着前一乘数逐次递减1,积逐次递减3.

要使上述规律在引入负数后仍然成立,那么你认为下面的空格应该填写什么数?

(<font face=symbol>-</font>1)&times;3=__,

(<font face=symbol>-</font>2)&times;3=__,

(<font face=symbol>-</font>3)&times;3=__,

从符号和绝对值两个角度观察上述所有算式,可以归纳如下:

正数乘正数,积为正数;正数乘负数,积为负数;负数乘正数,积也是负数,
积的绝对值等于各乘数绝对值的积.

\beginexample

利用上面归纳的结论计算下面的公式,你发现有什么规律?

(<font face=symbol>-</font>3)&times;3=__,

(<font face=symbol>-</font>3)&times;2=__,

(<font face=symbol>-</font>3)&times;1=__,

(<font face=symbol>-</font>3)&times;0=__,

\endexample

可以发现,上述算式有如下规律:随着后一乘数逐次递减1,积逐次增加3.

按照上述规律,下面的空格可以各填什么数?

(<font face=symbol>-</font>3)&times;(<font face=symbol>-</font>1)=__

(<font face=symbol>-</font>3)&times;(<font face=symbol>-</font>2)=__

(<font face=symbol>-</font>3)&times;(<font face=symbol>-</font>3)=__

可以归纳出如下结论:

负数乘负数,积为正数,乘积的绝对值等于各乘数绝对值的积

一般地,我们有有理数乘法法则:

两数相乘,同号得正,异号得负,并把绝对值相乘。

任何数与0相乘,都得0.

例如,  (<font face=symbol>-</font>5)&times;(<font face=symbol>-</font>3), ............................同号两数相乘

        (<font face=symbol>-</font>5)&times;(<font face=symbol>-</font>3)=+( ),................................得正

所以    5&times;3=15

又如    (<font face=symbol>-</font>7)&times;4,................................____________

        (<font face=symbol>-</font>7)&times;4=<font face=symbol>-</font>(),............................____________

        7&times;4=28,................................____________

所以    (<font face=symbol>-</font>7)&times;4=................................____________

也就是:有理数相乘,可以先确定积的符号,再确定积的绝对值。

\beginexample

例1 计算:

(1)(<font face=symbol>-</font>3)&times;9;

(2)8&times;(<font face=symbol>-</font>1);

(3)(<font face=symbol>-</font>1/2)&times;(<font face=symbol>-</font>2)

解:(1)(<font face=symbol>-</font>3)&times;9=<font face=symbol>-</font>27

(2)8&times;(<font face=symbol>-</font>1)=8

(3)(<font face=symbol>-</font>(1)/(2))&times;(<font face=symbol>-</font>2)=1

例1(3)中,(<font face=symbol>-</font>(1)/(2))&times;(<font face=symbol>-</font>2)=1,我们说(<font face=symbol>-</font>(1)/(2))和(-2)互为倒数,一般地,
在有理数中仍然有:

乘积是1的两个数互为倒数。

例2 用正负数去表示气温的变化量,上升为正,下降为负,登山队攀登一座山峰,每登高1km气温的变化量为
(<font face=symbol>-</font>6)<sup><font face=symbol>°</font></sup>
</td>
<td nowrap align=center>
  C,攀登3km后,气温有什么变化?

解:(<font face=symbol>-</font>6)&times;3=<font face=symbol>-</font>18

答:气温下降(18)<sup><font face=symbol>°</font></sup>
</td>
<td nowrap align=center>
  C

\endexample

\beginexercise

<p><a name="toc.1.1.2"><h3>1.1.2&nbsp;练习</h3>

1.计算:

(1)6&times;(<font face=symbol>-</font>9)

(2)(<font face=symbol>-</font>4)&times;6

(3)(<font face=symbol>-</font>6)&times;(<font face=symbol>-</font>1)

(4)(<font face=symbol>-</font>6)&times;0

(5)(2)/(3)&times;(<font face=symbol>-</font>(9)/(4))

(6)(<font face=symbol>-</font>(1)/(3))&times;(1)/(4)

2.商店降价销售某种商品,每件降5元,售出60件后,与按原价销售同样数量的商品相比,销售额有什么变化?

3.写出下列各数的倒数

1,-1,(1)/(3),<font face=symbol>-</font>(1)/(3),5,-5,(2)/(3),<font face=symbol>-</font>(2)/(3)

\endexercise

多个有理数相乘,可以把它们按顺序依次相乘

<p><a name="toc.1.1.3"><h3>1.1.3&nbsp;思考</h3>

观察下列各式,它们的积是正的还是负的?

2&times;3&times;4&times;(<font face=symbol>-</font>5)

2&times;3&times;(<font face=symbol>-</font>4)&times;(<font face=symbol>-</font>5)

2&times;(<font face=symbol>-</font>3)&times;(<font face=symbol>-</font>4)&times;(<font face=symbol>-</font>5)

(<font face=symbol>-</font>2)&times;(<font face=symbol>-</font>3)&times;(<font face=symbol>-</font>4)&times;(<font face=symbol>-</font>5)

几个不是0的数相乘,积的符号与负因数的个数之间有什么关系?

<p><a name="toc.1.1.4"><h3>1.1.4&nbsp;归纳</h3>

几个不是0的数相乘,负因数的个数是偶数时,积是正数;负因数的个数是奇数时,积是负数。

\beginexample

例3 计算:

(1)(<font face=symbol>-</font>3)&times;(5)/(6)&times;(<font face=symbol>-</font>(9)/(5))&times;(<font face=symbol>-</font>(1)/(4));

(2)(<font face=symbol>-</font>5)&times;6&times;(<font face=symbol>-</font>(4)/(5))&times;(1)/(4),

解:(1)(<font face=symbol>-</font>3)&times;(5)/(6)&times;(<font face=symbol>-</font>(9)/(5))&times;(<font face=symbol>-</font>(1)/(4)) =
<font face=symbol>-</font>3&times;(5)/(6)&times;(9)/(5)&times;(<font face=symbol>-</font>(1)/(4))=<font face=symbol>-</font>(9)/(8)

(2)(<font face=symbol>-</font>5)&times;6&times;(<font face=symbol>-</font>(4)/(5))&times;(1)/(4)=5&times;6&times;(4)/(5)&times;(1)/(4)=6

多个不是0的数相乘,先做哪一步,再做哪一步?

\endexample

<p><a name="toc.1.1.5"><h3>1.1.5&nbsp;思考</h3>

你能看出下式的结果吗?如果能,请说明理由:

7.8&times;(<font face=symbol>-</font>8.1)&times;0&times;(<font face=symbol>-</font>19.6)

几个数相乘,如果其中有因数为0,那么积等于0.

\endarticle

\enddocument

<hr>
<p><h1>Table Of Contents</h1>
<p><a href="#toc.1"><h1>1&nbsp;1.4 有理数的乘除法</h1></a>
<p><a href="#toc.1.1"><h2>1.1&nbsp;1.4.1 有理数的乘法</h2></a>
<p><a href="#toc.1.1.1"><h3>1.1.1&nbsp;思考</h3></a>
<p><a href="#toc.1.1.2"><h3>1.1.2&nbsp;练习</h3></a>
<p><a href="#toc.1.1.3"><h3>1.1.3&nbsp;思考</h3></a>
<p><a href="#toc.1.1.4"><h3>1.1.4&nbsp;归纳</h3></a>
<p><a href="#toc.1.1.5"><h3>1.1.5&nbsp;思考</h3></a>
</body>
</html>
