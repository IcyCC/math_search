<html>
<head>
<title>LaTeX4Web 1.4 OUTPUT</title>
<style type="text/css">
<!--
 body {color: black;  background:"#FFCC99";  }
 div.p { margin-top: 7pt;}
 td div.comp { margin-top: -0.6ex; margin-bottom: -1ex;}
 td div.comb { margin-top: -0.6ex; margin-bottom: -.6ex;}
 td div.norm {line-height:normal;}
 td div.hrcomp { line-height: 0.9; margin-top: -0.8ex; margin-bottom: -1ex;}
 td.sqrt {border-top:2 solid black;
          border-left:2 solid black;
          border-bottom:none;
          border-right:none;}
 table.sqrt {border-top:2 solid black;
             border-left:2 solid black;
             border-bottom:none;
             border-right:none;}
-->
</style>
</head>
<body>
\documentclassarticle
\usepackage[utf8]ctex

\begindocument

\maketitle

<h3>1.5.3 近似数</h3>

\beginarticle

先看一个例子。对于参加同一个会议的人数,有两个报道。一个报道说:“会议秘书处宣布,参加今天会议的有513人。”这里数字513确切地反映了实际人数,它是一个准确数。另一报道说:“约有五百人参加了今天的会议。”五百这个数只是接近实际人数,但与实际人数还有差别,它是一个\begindefinition近似数\enddefinition(approximate number)。

在许多情况下,很难取得准确数,或者不必使用准确数,而可以使用近似数。例如,宇宙现在的年龄约为200亿年,长江长约6300km,圆周率<font face=symbol>p</font>约为3.13,这里的数都是近似数。

近似数与准确数的接近程度,可以用精确度表示。例如,前面的五百是精确到百位数的近似数,它与准确数513的误差为13。

按四舍五入对圆周率<font face=symbol>p</font>取近似数时,有

<font face=symbol>p</font>  <font face=symbol>»</font>  3(精确到个位),

<font face=symbol>p</font>  <font face=symbol>»</font>  3.1(精确到0.1,或叫做精确到十分位),

<font face=symbol>p</font>  <font face=symbol>»</font>  3.14(精确到0.01,或叫做精确到百分位),

<font face=symbol>p</font>  <font face=symbol>»</font>  3.142(精确到\underline\hbox to 20mm,或叫做精确到\underline\hbox to 20mm),

<font face=symbol>p</font>  <font face=symbol>»</font>  3.1416(精确到\underline\hbox to 20mm),或叫做精确到\underline\hbox to 20mm)。

······

\beginexample
例6 按括号内的要求,用四舍五入法对下列各数取近似数:

(1) 0.0158(精确到0.001);

(2) 304.35(精确到个位);

(3) 1.804(精确到0.1);

(4) 1.804(精确到0.01)。

解:(1) 0.0158  <font face=symbol>»</font>  0.016;

(2) 304.35  <font face=symbol>»</font>  304;

(3) 1.804  <font face=symbol>»</font>  1.8;

(4) 1.804  <font face=symbol>»</font>  1.80;

\endexample

\beginexeicise

用四舍五入法对下列各数取近似数:

(1) 0.00356(精确到万分位);    (2) 61.235(精确到个位);

(3) 1.8935(精确到0.001);      (4) 0.0571(精确到0.1)。

习题1.5

复习巩固

1. 计算:

    (1) (<font face=symbol>-</font>3)<sup>3</sup>;    (2) (<font face=symbol>-</font>2)<sup>4</sup>;

    (3) (<font face=symbol>-</font>1.7)<sup>2</sup>;  (4) <font face=symbol>-</font>(4)/(3)<sup>3</sup>;

    (5) <font face=symbol>-</font>(<font face=symbol>-</font>2)<sup>3</sup>;   (6) (<font face=symbol>-</font>2)<sup>2</sup> &times; (<font face=symbol>-</font>3)<sup>2</sup>。

2. 用计算器计算:

    (1) (<font face=symbol>-</font>12)<sup>8</sup>;    (2) 103<sup>4</sup>;

    (3) 7.12<sup>3</sup>;       (4) <font face=symbol>-</font>45.7<sup>3</sup>;

3. 计算:

    (1) (<font face=symbol>-</font>1)<sup>1</sup>00 &times; 5 + (<font face=symbol>-</font>2)<sup>4</sup> <font face=symbol>¸</font> 4;

    (2) (<font face=symbol>-</font>3)<sup>2</sup><font face=symbol>-</font>3&times; (<font face=symbol>-</font>(1)/(3))<sup>4</sup>;

    (3) (7)/(6) &times; ((1)/(6)<font face=symbol>-</font>(1)/(3))&times; (3)/(14)<font face=symbol>¸</font> (3)/(5);

    (4) (<font face=symbol>-</font>10)<sup>3</sup>+[(<font face=symbol>-</font>4)<sup>2</sup><font face=symbol>-</font>(1<font face=symbol>-</font>3<sup>2</sup>)&times; 2];

    (5) <font face=symbol>-</font>2<sup>3</sup><font face=symbol>¸</font> (4)/(9) &times; (<font face=symbol>-</font>(2)/(3))<sup>2</sup>;

    (6) 4+(<font face=symbol>-</font>2)<sup>3</sup>&times; 5<font face=symbol>-</font>(<font face=symbol>-</font>0.28)<font face=symbol>¸</font> 4。

4. 用科学计数法表示下列各数:

    (1) 235 000 000;       (2) 188 520 000;

    (3) 701 000 000 000;   (4) -38 000 000;

5. 下列用科学计数法表示的数,原来各是什么树?

    3&times; 10<sup>2</sup>,1.3&times; 10<sup>3</sup>,8.05&times; 10<sup>6</sup>,2.004&times; 10<sup>5</sup>,<font face=symbol>-</font>1.96&times; 10<sup>4</sup>。

6. 用四舍五入法对下列各数取近似值:

    (1) 0.00356(精确到0.0001);

    (2) 566.1235(精确到个位);

    (3) 3.8963(精确到0.01);

    (4) 0.0571(精确到千分位);

综合运用

7. 平方等于9的数是几?立方等于27的数是几?

8. 一个长方体的长、宽都是a,高是b,它的体积和表面积怎样计算?当a=2cm,b=5cm时,它的体积和表面积是多少?

9. 地球绕太阳公转的速度是1.1&times; 10<sup>5</sup>km/h,声音在空气中的传播速度是340m/s,试比较两个速度的大小。

10. 一天有8.64 &times; 10<sup>4</sup>s,一年按365天计算,一年有多少秒(用科学计数法表示)?

拓展探索

11. (1) 计算0.1<sup>2</sup>,1<sup>2</sup>,10<sup>2</sup>,100<sup>2</sup>,观察这些结果,底数的小数点向左(右)移动一位时,平凡数小数点有什么移动规律?

    (2) 计算0.1<sup>3</sup>,1<sup>3</sup>,10<sup>3</sup>,100<sup>3</sup>,观察这些结果,底数的小数点向左(右)移动一位时,平凡数小数点有什么移动规律?

    (3) 计算0.1<sup>4</sup>,1<sup>4</sup>,10<sup>4</sup>,100<sup>4</sup>,观察这些结果,底数的小数点向左(右)移动一位时,平凡数小数点有什么移动规律?

12。 计算(<font face=symbol>-</font>2)<sup>2</sup>,2<sup>2</sup>,(<font face=symbol>-</font>2)<sup>3</sup>,2<sup>3</sup>。联系这类具体的数的乘方,你认为当a<0时下列各式是否成立?

    (1) a<sup>2</sup>&gt;0;       (2) a<sup>2</sup>=(<font face=symbol>-</font>a)<sup>2</sup>;

    (3) a<sup>2</sup>=<font face=symbol>-</font>a<sup>2</sup>;    (4) a<sup>3</sup>=<font face=symbol>-</font>a<sup>3</sup>;

\endexeicise

数学活动

活动1

帮助家庭记录一个月(或一周)的生活收支账目,收入记为正数,支出记为负数,计算当月(周)的总收入、总支出、总节余以及每日平均支出等数据。

妥善保存账目,作为日后家庭理财的参考资料。

活动2

熟悉你所用的计算器有关有理数运算的功能和操作方法,对于包含乘方、乘除与加减运算的算式,考虑怎样操作计算器最简便,实习这样的操作,并与同学进行交流。

活动3

收集现实生活中你认为非常大的数据的实例,体会科学记数法和近似数等在实际中的应用。

小结

一、本章知识结构图

 <font face=symbol>Î</font> cludegraphics[width=3in]51-55小结.png

二、回顾与思考

本章我们在小学学习的基础上,进一步认识了负数,使数的范围扩充到有理数,引入负数不仅可以表示具有相反意义的量,而且还扩展了减法运算的范围,由此,类似于x+2=1的方程就可以解了。

我们知道,有理数是整数与分数的统称,由于整数可以看成是分母为1的分数,因此有理数可以写成(p)/(q)(p, q是整数, q <font face=symbol>¹</font>  0)的形式;另一方面,形如(p)/(q)(p,q是整数,q <font face=symbol>¹</font> 0)的数都是有理数。所以,有理数可用(p)/(q)(p,q是整数,q  <font face=symbol>¹</font>  0)表示。

本章我们研究了有理数的加、减、乘、除和乘方运算。实际上,与负数有关的运算,我们都借助绝对值,将它们转化为正数之间的运算。数轴不仅能直观表示数,而且还能帮助我们理解数的运算。在运算的过程中,数形结合、转化是很重要的思想方法。

我们从具体数的加法和乘法中,扫纳出了交换律、结合律和分配律等运算律,运算律不仅能给数的运算带来方便,而且还是今后研究代数问题(如解方程、不等式等)的基础。

    请你带着下面的问题,复习一下全章的内容吧。
    1.你能举出一些实例,说明正数、负数在表示相反意义的量时的作用吗?
    2.你能用一个图表示有理数的分类吗?引入负数后,减法中哪些原来不能进行的运算可以进行了?
    3.怎样用数轴表示有理数?数轴与普通直线有什么不同?怎祥利用数轴

\endarticle
\enddocument
</body>
</html>
