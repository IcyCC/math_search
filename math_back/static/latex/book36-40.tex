<html>
<head>
<title>LaTeX4Web 1.4 OUTPUT</title>
<style type="text/css">
<!--
 body {color: black;  background:"#FFCC99";  }
 div.p { margin-top: 7pt;}
 td div.comp { margin-top: -0.6ex; margin-bottom: -1ex;}
 td div.comb { margin-top: -0.6ex; margin-bottom: -.6ex;}
 td div.norm {line-height:normal;}
 td div.hrcomp { line-height: 0.9; margin-top: -0.8ex; margin-bottom: -1ex;}
 td.sqrt {border-top:2 solid black;
          border-left:2 solid black;
          border-bottom:none;
          border-right:none;}
 table.sqrt {border-top:2 solid black;
             border-left:2 solid black;
             border-bottom:none;
             border-right:none;}
-->
</style>
</head>
<body>
\documentclass[11pt]article
\usepackageamsmath

\usepackagectex
\newtheoremexercise
\newtheoremarticle
\newtheoremnature
\newtheoremtip
\begindocument

\beginexercise
练习<br>

1.化简<br>

(1)
(<font face=symbol>-</font>72)/(9),<br>

(2)
(<font face=symbol>-</font>30)/(<font face=symbol>-</font>45),<br>

(3)
(0)/(<font face=symbol>-</font>75),<br>

2.计算:
(1)
(<font face=symbol>-</font>36(9)/(11)<font face=symbol>¸</font> 9.<br>

(2)
(<font face=symbol>-</font>12)<font face=symbol>¸</font>(<font face=symbol>-</font>4)<font face=symbol>¸</font>(<font face=symbol>-</font>1(1)/(5))<br>

(3)
(<font face=symbol>-</font>(2)/(3))&times;(<font face=symbol>-</font>(8)/(5))<font face=symbol>¸</font>(<font face=symbol>-</font>0.25)<br>

\endexercise
\beginarticle
	有理数的加减乘除混合运算,如无括号指出先做什么运算,则与小学所学的混合运算一样,按照“先乘除,后加减”的顺序执行.<br>

\endarticle
\beginexercise
例3 计算:<br>

(1)
<font face=symbol>-</font>8+4<font face=symbol>¸</font>(<font face=symbol>-</font>2)<br>

(2)
(<font face=symbol>-</font>7)&times;(<font face=symbol>-</font>5)<font face=symbol>-</font>90<font face=symbol>¸</font>(<font face=symbol>-</font>15)<br>

解:(1)<br>

(<font face=symbol>-</font>8)+4<font face=symbol>¸</font>(<font face=symbol>-</font>2)<br>

=<font face=symbol>-</font>8+(<font face=symbol>-</font>2)<br>

=<font face=symbol>-</font>10<br>

(2)<br>

(<font face=symbol>-</font>7)&times;(<font face=symbol>-</font>5)<font face=symbol>-</font>90<font face=symbol>¸</font>(<font face=symbol>-</font>15)<br>

=35<font face=symbol>-</font>(<font face=symbol>-</font>6)<br>

35+6<br>

41<br>

\endexercise

\beginexercise
练习<br>

计算:<br>

(1)
6<font face=symbol>-</font>(<font face=symbol>-</font>12)<font face=symbol>¸</font>(<font face=symbol>-</font>3),<br>

(2)
3&times;(<font face=symbol>-</font>4)+(<font face=symbol>-</font>28)<font face=symbol>¸</font>7<br>

(3)
(<font face=symbol>-</font>48)<font face=symbol>¸</font>8<font face=symbol>-</font>(<font face=symbol>-</font>25)&times;(<font face=symbol>-</font>6)<br>

(4)
42&times;(<font face=symbol>-</font>frac23)+(<font face=symbol>-</font>frac34)<font face=symbol>¸</font>(<font face=symbol>-</font>0.25)
\endexercise

\beginexercise
例9 某公司去年1&nbsp;3月平均每月亏损1.5万元,4&nbsp;6月平均每月盈利<br>

2万元,7&nbsp;10月平均每月盈利1.7万元,11&nbsp;12月平均每月亏损2.3万元.<br>

这个公司去年总的盈亏情况如何?<br>

解:<br>

记盈利额为正数,亏损额为负数,公司去年全年盈亏额(单位:万元)为<br>

(<font face=symbol>-</font>1.5)&times;3+2&times;3+1.7&times;4+(<font face=symbol>-</font>2.3)&times;2<br>

=<font face=symbol>-</font>4.5+6+6.8<font face=symbol>-</font>4.6=3.7<br>

答:这个公司去年全年盈利为3.7万元<br>

\endexercise

\beginarticle
    计算器是一种方便实用的计算工具,用计算器进行比较复杂的数的计算,比笔算要快捷的多.<br>

    例如,可以用计算器计算例9中的<br>

    (<font face=symbol>-</font>1.5)&times;3+2&times;3+1.7&times;4+(<font face=symbol>-</font>2.3)&times;2<br>

    不同品牌的计算器的操作方法可能有所不同,具体参见计算器的使用说明.<br>

\endarticle

\beginexercise
练习<br>

357+(<font face=symbol>-</font>154)+26+(<font face=symbol>-</font>212).<br>

<font face=symbol>-</font>5.13+4.62+(<font face=symbol>-</font>8.47)<font face=symbol>-</font>(<font face=symbol>-</font>2.3)<br>

26&times;(<font face=symbol>-</font>41)+(<font face=symbol>-</font>35)&times;(<font face=symbol>-</font>17).<br>

1.252<font face=symbol>¸</font>(<font face=symbol>-</font>44)<font face=symbol>-</font>(<font face=symbol>-</font>356)<font face=symbol>¸</font>(<font face=symbol>-</font>0.196).<br>

\endexercise

\beginexercise
习题1.4<br>

1.计算:<br>

(1)
(<font face=symbol>-</font>8)&times;(<font face=symbol>-</font>7)<br>

(2)
12&times;(<font face=symbol>-</font>5)<br>

(3)
2.9&times;(<font face=symbol>-</font>0.4)<br>

(4)
<font face=symbol>-</font>30.5&times;0.2<br>

(5)
100&times;(<font face=symbol>-</font>0.001)<br>

(6)
<font face=symbol>-</font>4.8&times;(<font face=symbol>-</font>1.25)<br>


2.计算:<br>

(1)
(1)/(4)&times;(<font face=symbol>-</font>frac89)<br>

(<font face=symbol>-</font>frac56)&times;(<font face=symbol>-</font>frac310)<br>

(<font face=symbol>-</font>frac3415)&times;25<br>

(<font face=symbol>-</font>0.3)&times;(<font face=symbol>-</font>frac107)<br>


3.写出下列各数的倒数<br>

(1)<font face=symbol>-</font>15<br>

(2)<font face=symbol>-</font>(5)/(9)<br>

(3)<font face=symbol>-</font>0.25<br>

(4)0.17<br>

(5)4(1)/(4)<br>

(6)<font face=symbol>-</font>5(2)/(5)<br>


4.计算<br>

(1)<font face=symbol>-</font>91<font face=symbol>¸</font>13<br>

(2)<font face=symbol>-</font>56<font face=symbol>¸</font>(<font face=symbol>-</font>14)<br>

(3)16<font face=symbol>¸</font>(<font face=symbol>-</font>3)<br>

(4)(<font face=symbol>-</font>48)<font face=symbol>¸</font>(<font face=symbol>-</font>16)<br>

(5)(4)/(5)<font face=symbol>¸</font>(<font face=symbol>-</font>1)<br>

(6)<font face=symbol>-</font>0.25<font face=symbol>¸</font>(3)/(8)<br>


5.填空<br>

1&times;(<font face=symbol>-</font>5)=<br>

1<font face=symbol>¸</font>(<font face=symbol>-</font>5)<br>

1+(<font face=symbol>-</font>5)<br>

1<font face=symbol>-</font>(<font face=symbol>-</font>5)<br>

<font face=symbol>-</font>1&times;(<font face=symbol>-</font>5)<br>

<font face=symbol>-</font>1<font face=symbol>¸</font>(<font face=symbol>-</font>5)<br>

<font face=symbol>-</font>1+(<font face=symbol>-</font>5)<br>

<font face=symbol>-</font>1<font face=symbol>-</font>(<font face=symbol>-</font>5)<br>


6.化简下列分数<br>

(1)(<font face=symbol>-</font>21)/(7)<br>

(2)(3)/(<font face=symbol>-</font>36)<br>

(3)(<font face=symbol>-</font>54)/(<font face=symbol>-</font>8)<br>

(4)(<font face=symbol>-</font>6)/(<font face=symbol>-</font>0.3)<br>


7.计算<br>

(1)<font face=symbol>-</font>2&times;3&times;(<font face=symbol>-</font>4)<br>

(2)<font face=symbol>-</font>6&times;(<font face=symbol>-</font>5)&times;(<font face=symbol>-</font>7)<br>

(3)(<font face=symbol>-</font>(8)/(25))&times;1.25&times;(<font face=symbol>-</font>8)<br>

(4)0.1<font face=symbol>¸</font>(<font face=symbol>-</font>0.001)<font face=symbol>¸</font>(<font face=symbol>-</font>1)<br>

(5)(<font face=symbol>-</font>(3)/(4))&times;(<font face=symbol>-</font>1(1)/(2))<font face=symbol>¸</font>(<font face=symbol>-</font>2(1)/(4))<br>

(6)<font face=symbol>-</font>6&times;(0.25)&times;(11)/(14)<br>

(7)(<font face=symbol>-</font>7)&times;(<font face=symbol>-</font>56)&times;0<font face=symbol>¸</font>(<font face=symbol>-</font>13)<br>

(8)<font face=symbol>-</font>9&times;(<font face=symbol>-</font>11)<font face=symbol>¸</font>(3)<font face=symbol>¸</font>(<font face=symbol>-</font>3)<br>


综合运用<br>

8.计算:<br>

(1)23&times;(<font face=symbol>-</font>5)<font face=symbol>-</font>(<font face=symbol>-</font>3)<font face=symbol>¸</font>(3)/(128)<br>

(2)<font face=symbol>-</font>7&times;(<font face=symbol>-</font>3)&times;(<font face=symbol>-</font>0.5)+(<font face=symbol>-</font>12)&times;(<font face=symbol>-</font>2.6)<br>

(3)(1(3)/(4)<font face=symbol>-</font>(7)/(8)<font face=symbol>-</font>(7)/(12))<font face=symbol>¸</font>(<font face=symbol>-</font>(7)/(8))<font face=symbol>¸</font>(1(3)/(4)<font face=symbol>-</font>(7)/(8)<font face=symbol>-</font>(7)/(12))<br>

(4)<font face=symbol>-</font> | <font face=symbol>-</font>(2)/(3) | <font face=symbol>-</font> | <font face=symbol>-</font>(1)/(2)&times;(2)/(3) | <font face=symbol>-</font> | (1)/(3)<font face=symbol>-</font>(1)/(4) | <font face=symbol>-</font> | <font face=symbol>-</font>3 | <br>


9.用计算器计算.<br>

(1)(<font face=symbol>-</font>36)&times;128<font face=symbol>¸</font>(<font face=symbol>-</font>74)<br>

(2)<font face=symbol>-</font>6.23<font face=symbol>¸</font>(<font face=symbol>-</font>0.25)&times;940<br>

(3)<font face=symbol>-</font>4.325&times;(<font face=symbol>-</font>0.012)<font face=symbol>-</font>2.31<font face=symbol>¸</font>(<font face=symbol>-</font>5.315)<br>

(4)180.65<font face=symbol>-</font>(<font face=symbol>-</font>32)&times;47.8<font face=symbol>¸</font>(<font face=symbol>-</font>15.5)<br>


10.用正数和负数填空.<br>

(1)小商店平均每天可盈利250元,一个月的利润是<br>

(2)小商店每天亏损20元,一周的利润是<br>

(3)小商店一周的利润是1400元,平均每天的利润是<br>

(4)小商店一周一共亏损840元,平均每天的利润是<br>


11.一架直升机从高度为450m的位置开始,先以20m/s的速度上升60s,后以12m/s的速度下降120s,这是直升机所在的高低是多少?

拓展探索<br>

12.计算2&times;1,2&times;(1)/(2,2&times;(<font face=symbol>-</font>1),2&times;(<font face=symbol>-</font>(1)/(2).<br>

联系这类具体的数的乘法,你认为一个非0的有理数一定小于它的2倍吗?为什么?<br>


13.利用分配律可以得到<font face=symbol>-</font>2&times;6+3&times;6=(<font face=symbol>-</font>2+3)&times;6.如果用a表示任意一个数,那么利用分配律可以得到<font face=symbol>-</font>2a+3a等于什么?

14.计算(<font face=symbol>-</font>4)<font face=symbol>¸</font>2,4<font face=symbol>¸</font>(<font face=symbol>-</font>2),(<font face=symbol>-</font>4)<font face=symbol>¸</font>(<font face=symbol>-</font>2).<br>

联系这类具体的数的除法,你认为下列式子是否成立?<br>

(1)(<font face=symbol>-</font>a)/(b)=(a)/(<font face=symbol>-</font>b)=<font face=symbol>-</font>(a)/(b)<br>

(2)(<font face=symbol>-</font>a)/(<font face=symbol>-</font>b)=(a)/(b)<br>

\endexercise

\beginarticle
翻牌游戏中的数学道理<br>

桌上有9张正面向上的扑克牌,每次翻动其中的任意两张(包括已翻过的牌)使他们从一面向上变为另一面向上,这样一直下去,观察能否使所有的牌都反面向上?<br>

你不妨动手试一试,看看会不会出现所有的牌都反面向上<br>

事实上,不论你翻多少次,都不能是9张牌都反面向上,从这个结果,你能想到其中的数学道理吗?<br>

如果在每张牌的正面都写1,反面都写-1,考虑所有牌朝上一面数的积,开始9张牌都正面向上,上面数的积是1,每次翻动两张,即有两张牌同时改变符号,这能改变超上一面数的积是一这个结果吗?<br>

你能解释为什么不会使9张牌都反面向上了吗?<br>

如果桌上有任意奇数张牌,你猜想结果是怎样的?<br>

\endarticle

\enddocument
</body>
</html>
