<html>
<head>
<title>LaTeX4Web 1.4 OUTPUT</title>
<style type="text/css">
<!--
 body {color: black;  background:"#FFCC99";  }
 div.p { margin-top: 7pt;}
 td div.comp { margin-top: -0.6ex; margin-bottom: -1ex;}
 td div.comb { margin-top: -0.6ex; margin-bottom: -.6ex;}
 td div.norm {line-height:normal;}
 td div.hrcomp { line-height: 0.9; margin-top: -0.8ex; margin-bottom: -1ex;}
 td.sqrt {border-top:2 solid black;
          border-left:2 solid black;
          border-bottom:none;
          border-right:none;}
 table.sqrt {border-top:2 solid black;
             border-left:2 solid black;
             border-bottom:none;
             border-right:none;}
-->
</style>
</head>
<body>
\documentclassarticle
\usepackage[utf8]inputenc
\usepackage[utf8]ctex
\usepackagegraphicx

\date
\begindocument
\title
\maketitle
<h3> 3.2解一元一次方程(一)——合并同类项与移项</h3>

\beginarticle
    <p>&nbsp;&nbsp;&nbsp;&nbsp; 我们已经知道,直接利用等式的基本性质可以解简单的方程,本节重点讨论如何利用\begindefinition“合并同类项”和“移项”\enddefinition解一元一次方程.

    <p>&nbsp;&nbsp;&nbsp;&nbsp; 约公元820年,中亚细亚数学家阿尔-花拉子米写了一本代数书,重点论述怎样解方程.这本书的拉丁译本取名为《对消与还原》.“对消”与“还原”是什么意思呢?我们先讨论下面的内容,然后再回答这个问题.

    \beginexample
    <p>&nbsp;&nbsp;&nbsp;&nbsp;  问题1 某校三年共购买计算机140台,去年购买数量是前年的2倍,今年购买数量又是去年的2倍前年这个学校购买了多少台计算机?

    <p>&nbsp;&nbsp;&nbsp;&nbsp; 设前年购买计算机x台。可以表示出:去年购买计算机2x台,今年购买计算机4.x台.根据问题中的相等关系:前年购买量+去年购买量+今年购买量=140台,列得方程
    \begincenterx+2x+4x=140.\endcenter

    <p>&nbsp;&nbsp;&nbsp;&nbsp; 把含有x的项合并同类项,得
    \begincenter7x= 140.\endcenter

     <p>&nbsp;&nbsp;&nbsp;&nbsp; 下面的框图表示了解这个方程的流程:
     \beginfigure
         \centering
          <font face=symbol>Î</font> cludegraphicsp1.PNG
     \endfigure

     <p>&nbsp;&nbsp;&nbsp;&nbsp; 由下可知,前年这个学校购买了20台计算机
     <p><hr>
      例1解下列方程:

      (1) 2x<font face=symbol>-</font>(5)/(2)x=6<font face=symbol>-</font>8;  (2) 7x<font face=symbol>-</font>2.5x+3x<font face=symbol>-</font>1.5x=<font face=symbol>-</font>15&times;4<font face=symbol>-</font>6&times;3.

      解: (1) 合并同类项,得

      <font face=symbol>-</font>(1)/(2)x=<font face=symbol>-</font>2

      系数化为1,得
      x=4.

      (2)合并同类项,得
     6x=<font face=symbol>-</font>78.
      系数化为1,得
      x=<font face=symbol>-</font>13.

      例2有一列数.按一定规律排列成1, -3, 9,一27, 81. -243. .其中某三个相邻数的和是一-1 701.这三个数各是多少?

      分析:从符号和绝对值两方面观察,可发现这列数的排列规律:后面的数是它前面的数与- 3的乘积,如果三个相邻数中的第1个记为x,则后两个数分别是-3x, 9x.

      解:设所求三个数分别是x,- 3x, 9x.由三个数的和是一1 701.得

      x<font face=symbol>-</font>3x+9r=<font face=symbol>-</font>1701.

      合并同类项,得

      7x=<font face=symbol>-</font>1 701.

      系数化为1,得

      x=<font face=symbol>-</font>243.

      所以
      一3x=729,9x=<font face=symbol>-</font>2187.
      答:这三个数是-243,729,- -2 187.
      \endexample

      \beginexercise
      (1) 5x<font face=symbol>-</font>2x=9;  (2)(x)/(2)+(3x)/(2)=7

      (3)<font face=symbol>-</font>3x+0.5x=10 (4)7x<font face=symbol>-</font>4.5x=2.5&times;3<font face=symbol>-</font>5.

      2.某工厂的产值连续增长,去年是前年的1.5倍,今年是去年的2倍。这三年的总产值为550万元,前年的产值是多少?
      \endexercise

      \beginexample
      问题2把一些图书分给某班学生阅读,如果每人分3本,则剩余20本;如果每人分4本,则还缺25本.这个班有多少学生?

      设这个班有x名学生

      每人分3本,共分出3x本,加上剩余的20本,这批书共(3x+20)本

      每人分4本,需要4x:本,减去缺的25本,这批书共(4x-25)本

      这批书的总数是一个定值,表示它的两个式子应相等,根据这一相等关 系列得方程

      3x+20=4x<font face=symbol>-</font>25.

     思考
     方程3x+20=4x- 25的两边都有含x的项(3x与4r)和不含字母的常数项(20与25),怎样才能使它向x=a(常数)的形式转化呢?

     为了使方程的右边没有含的项,等号两边减4.x:为了使左边没有常数项,等号两边减20.利用等式的性质1,得

     3x<font face=symbol>-</font><font face=symbol>-</font>4x=<font face=symbol>-</font>25<font face=symbol>-</font>20.

     上面方程的变形,相当于把原方程左边的20变为一20移到右边,把右边的4.x变为-4x移到左边把某项从等式边移到另一边时有什么变化?

     像上面那样把等式边的某项变号后移到另-边,叫做\begindefination移项\enddefination.
    \endexample

    解方程时经常要“合并同类项”和“移项”,前面提到的古老的代数书中的“对消”和“还原”,指的就是“合并同类项"和“移项".早在一千多年前,数学家阿尔-花拉子米就已经对“合并同类项”和“移项”非常重视了,
    \beginexample

    例3解下列方程

    (1) 3x+7=32<font face=symbol>-</font>2x;(2) x<font face=symbol>-</font>3= (3)/(2)x+1.

    解: (1)移项,得

    3x+2x=32<font face=symbol>-</font> 7.

    合并同类项,得

    5x=25.

    系数化为1,得

    x=5.

    (2)移项,得

    x<font face=symbol>-</font>(3)/(2)x=1+3.

    合并同类项,得

    -(1)/(2)x=4.

    系数化为1

    x=-8.

    例4某制药厂制造批药品,如用旧工艺,则废水排量要比环保限制的最大量还多200t;如用新工艺,则废水排量比环保限制的最大量少100t.新、旧工艺的废水排量之比为2:5,两种工艺的废水排量各是多少?

    分析:因为新、旧工艺的废水推量之比为2:5.所以可设它们分别为2xt和5xt,再根据它们与环保限制的最大量之间的关系列方程。

    解:设新、旧工艺的废水排量分别为2xt和5xt。

    根据废水排量与环保限制最大量之间的关系,得

    5x-200=2x+100.

    移项,得

    5x-2x=100+200.

    合并同类项,得

    3x=300.

    系数化为1,得

    x=100.

    所以

    2x=200,

    5x=500.

    答:新、旧工艺产生的废水排量分别为200t和500t.
    \endexample
    \beginexercise

   练习

    1.解下列方程:

    (1) 6x-7=4x-5;

    (2)(1)/(2)x-6=(3)/(4)x.

    2.王芳和李丽同时采摘樱桃,王芳平均每小时采摘8kg,李丽平均每小时采摘7kg.采捕结束后王芳从她采摘的樱桃中取出0.25 kg给了李丽,这时两人的樱桃一样多.她们采摘用了多少时间?
    \endexercise
\endarticle
\enddocument
</body>
</html>
