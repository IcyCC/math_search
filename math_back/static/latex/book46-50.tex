<html>
<head>
<title>LaTeX4Web 1.4 OUTPUT</title>
<style type="text/css">
<!--
 body {color: black;  background:"#FFCC99";  }
 div.p { margin-top: 7pt;}
 td div.comp { margin-top: -0.6ex; margin-bottom: -1ex;}
 td div.comb { margin-top: -0.6ex; margin-bottom: -.6ex;}
 td div.norm {line-height:normal;}
 td div.hrcomp { line-height: 0.9; margin-top: -0.8ex; margin-bottom: -1ex;}
 td.sqrt {border-top:2 solid black;
          border-left:2 solid black;
          border-bottom:none;
          border-right:none;}
 table.sqrt {border-top:2 solid black;
             border-left:2 solid black;
             border-bottom:none;
             border-right:none;}
-->
</style>
</head>
<body>
\documentclassarticle

\setcountersecnumdepth2

\title\LaTeX.js Showcase
\authormade with \varheartsuit by Michael Brade
\date2017--2018


\begindocument

\maketitle

This document will show most of the supported features of \LaTeX.js.


<p><a name="toc.1"><h1>1&nbsp;Characters</h1>

It is possible to input any UTF-8 character either directly or by character code
using one of the following:

<ul>
    
<li> \texttt\textbackslash symbol{"00A9}: \symbol"00A9
    
<li> \verb|\char"A9|: \char"A9
    
<li> \verb|<sup>^</sup>A9 or <sup>^</sup><sup>^</sup>00A9|: <sup>^</sup>A9 or <sup>^</sup><sup>^</sup>00A9
</ul>

\bigskip

<p>
Special characters, like those:
\begincenter
\ \& \% \# \<sub> </sub>{ } &#126; \<sup></sup>
</td>
<td nowrap align=center>
   \textbackslash \endcenter
have to be escaped.

More than 200 symbols are accessible through macros. For instance: 30&nbsp;\textcelsius is
86&nbsp;\textdegreeF. See also Section&nbsp;<a href="#refsec:symbols">(0)</a>
.



<p><a name="toc.2"><h1>2&nbsp;Spaces and Comments</h1>

Spaces and comments, of course, work just as they do in \LaTeX.
This is an            example: Supercal              ifragilist    icexpialidocious

It does not matter whether you enter one or several     spaces after a word, it
will always be typeset as one space<font face=symbol>-</font><font face=symbol>-</font><font face=symbol>-</font>unless you force several spaces, like  now.

New \TeX users may miss whitespaces after a command. Experienced \TeX users are \TeX perts, and know how to use whitespaces. 
Longer comments can be embedded in the \textttcomment environment:
This is another  \begin  comment
rather stupid,
but helpful
\end
comment
example for embedding comments in your document.



<p><a name="toc.3"><h1>3&nbsp;Dashes and Hyphens</h1>

\LaTeX knows four kinds of dashes. Access three of them with different numbers
of consecutive dashes. The fourth sign is actually not a dash at all<font face=symbol>-</font><font face=symbol>-</font><font face=symbol>-</font>it is the
mathematical minus sign:

\beginquote
  daughter<font face=symbol>-</font>in<font face=symbol>-</font>law, X<font face=symbol>-</font>rated<br>

  pages 13<font face=symbol>-</font><font face=symbol>-</font>67<br>

  yes<font face=symbol>-</font><font face=symbol>-</font><font face=symbol>-</font>or no? <br>

  0, 1 and -1
\endquote
The names for these dashes are: ‘<font face=symbol>-</font>’ hyphen, ‘<font face=symbol>-</font><font face=symbol>-</font>’ en<font face=symbol>-</font>dash, ‘<font face=symbol>-</font><font face=symbol>-</font><font face=symbol>-</font>’ em<font face=symbol>-</font>dash,
and ‘-’ minus sign. \LaTeX.js outputs the actual true unicode character for those
instead of using the hypen<font face=symbol>-</font>minus.



<p><a name="toc.4"><h1>4&nbsp;Text and Paragraphs, Ligatures</h1>

An empty line starts a new paragraph, and so does \verb|<br>|.
<br> Like this. A new line usually starts automatically when the previous one is
full. However, using \verb+\newline+ or \verb|<br>
|,\newline one can force <br>
 to start a new line.

Ligatures are supported as well, for instance:

\begincenter
fi, fl, ff, ffi, ffl \dots instead of f\/i, f\/l, f\/f\/l \dots
\endcenter

Use \texttt\textbackslash<span style="position:relative;left:-7pt;"><b><font face=courier size=+2>/</font></b></span>sh to prevent a ligature.



\beginmulticols2[<p><a name="toc.4.1"><h2>4.1&nbsp;Multicolumns</h2>]

The multi<font face=symbol>-</font>column layout, using the \textttmulticols environment, allows easy
definition of multiple columns of text<font face=symbol>-</font><font face=symbol>-</font><font face=symbol>-</font>just like in newspapers. The first
and mandatoriy argument specifies the number of columns the text should be divided into.

It is often convenient to spread some text over all columns, just before the multicolumn
output. In \LaTeX, this was needed to prevent any page break in between. To achieve this,
the \textttmulticols environment has an optional second argument which can be used for
this purpose.

For instance, this text you are reading now was started with the argument
\texttt\textbackslash subsection{Multicolumns}.

\endmulticols



<p><a name="toc.5"><h1>5&nbsp;Boxes</h1>

\LaTeX.js supports most of the standard \LaTeX boxes.

\medbreak

&lt;p&gt;\fbox\verb|\mbox|\emphtext\verb||

<font size="-1">break

We already know one of them: it<font face=symbol>¢</font>s called \verb|\mbox|. It simply packs up a series of boxes into another one, and
can be used to prevent \LaTeX.js from breaking two words. As boxes can be put inside boxes, these horizontal box
packers give you ultimate flexibility.

\bigbreak

&lt;p&gt;\fbox
\verb|\makebox[|\emphwidth\verb|][|\emphpos\verb|]|\emphtext\verb||


&lt;p&gt;\fbox
\verb|\framebox[|\emphwidth\verb|][|\emphpos\verb|]|\emphtext\verb||


<font size="-1">break

&lt;p&gt;
\emphwidth defines the width of the resulting box as seen from the outside.
The \emphpos parameter takes a one letter value: \textbfcenter,
flush\textbfleft, or flush\textbfright. \textbfspread is not really working
in HTML.


The command \verb|framebox| works exactly the same as \verb|makebox|, but it
draws a box around the text.

The following example shows you some things you could do with
the \verb|makebox| and \verb|framebox| commands.

\beginquote
  \fbox\makebox[10cm][c]\textbfc e \textbfn t r a l<br>

  \frameboxGuess I<font face=symbol>¢</font>m framed now! <br>
  \framebox[2cm][r]Bummer, I am too wide <br>
  \framebox[1cm][l]never mind, so am I
  Can you read this?
\endquote


\bigbreak

&lt;p&gt;\fbox
\verb|<br>box[|\emphpos\verb|][|\emphheight\verb|][|\emphinner<font face=symbol>-</font>pos\verb|]|\emphwidth\verb||\emphtext\verb||


<font size="-1">break

&lt;p&gt;
The \verb|<br>box| command produces a box the contents of which are created in paragraph mode. However, only small
pieces of text should be used, paragraph<font face=symbol>-</font>making environments shouldn<font face=symbol>¢</font>t be used inside a \verb|<br>box| argument. For
larger pieces of text, including ones containing a paragraph<font face=symbol>-</font>making environment, you should use a \verb|minipage|
environment.

By default LaTeX will position vertically a parbox so its center lines up with the center of the surrounding text line.
When the optional position argument is present and equal either to ‘\verb|t|’ or ‘\verb|b|’, this allows you respectively
to align either the top or bottom line in the parbox with the baseline of the surrounding text. You may also specify
‘\verb|m|’ for position to get the default behaviour.

The optional \emphheight argument overrides the natural height of the box.

The \emphinner<font face=symbol>-</font>pos argument controls the placement of the text inside the box, as follows; if it is not specified, \emphpos is used.

\verb|t| text is placed at the top of the box.

\verb|c| text is centered in the box.

\verb|b| text is placed at the bottom of the box.

\verb|s| is not supported in HTML

<font size="-1">break&lt;p&gt;
The following examples demonstrate simple positioning:

&lt;ul&gt;
&lt;li&gt; simple alignment:

Some text
\fbox<br>box2cmparbox default alignment, parbox test text
some text
\fbox<br>box[t]2cmparbox top alignment, text parbox test text
some text
\fbox<br>box[b]2cmparbox bottom alignment, text parbox test text
some text.


&lt;li&gt; alignment with a given height:

Some text
\fbox<br>box[c][3cm]2cmparbox default alignment, parbox test text
some text
\fbox<br>box[t][3cm]2cmparbox top alignment, text parbox test text
some text
\fbox<br>box[b][3cm]2cmparbox bottom alignment, text parbox test text
some text.
&lt;/ul&gt;

&lt;p&gt;
The following examples demonstrate all explicit \emphpos/\emphinner<font face=symbol>-</font>pos combinations:
&lt;ul&gt;
&lt;li&gt; center alignment:

&lt;p&gt;
Some text
\fbox<br>box[c][3cm][t]2cmparbox default alignment, parbox test text
some text
\fbox<br>box[c][3cm][c]2cmparbox top alignment, text parbox test text
some text
\fbox<br>box[c][3cm][b]2cmparbox bottom alignment, text parbox test text
some text.


&lt;li&gt; top alignment:

&lt;p&gt;
Some text
\fbox<br>box[t][3cm][t]2cmparbox default alignment, parbox test text
some text
\fbox<br>box[t][3cm][c]2cmparbox top alignment, text parbox test text
some text
\fbox<br>box[t][3cm][b]2cmparbox bottom alignment, text parbox test text
some text.

&lt;li&gt; bottom alignment:

&lt;p&gt;
Some text
\fbox<br>box[b][3cm][t]2cmparbox default alignment, parbox test text
some text
\fbox<br>box[b][3cm][c]2cmparbox top alignment, text parbox test text
some text
\fbox<br>box[b][3cm][b]2cmparbox bottom alignment, text parbox test text
some text.


&lt;li&gt; top/bottom in one line:

&lt;p&gt;
Some text
\fbox<br>box[b][3cm][t]2cmparbox default alignment, parbox test text
some text
\fbox<br>box[t][3cm][c]2cmparbox top alignment, text parbox test text
some text.
&lt;/ul&gt;



<p><a name="toc.5.1"><h2>5.1&nbsp;Low<font face=symbol>-</font>level box<font face=symbol>-</font>interface</h2>

\LaTeX.js supports the following low<font face=symbol>-</font>level \TeX commands:
  \verb|&lt;&lt;ap|\emphtext\verb||,
  \verb|\rlap|\emphtext\verb||, and
  \verb|\smash|\emphtext\verb||, as well as
  \verb|\hphantom|\emphtext\verb||,
  \verb|<font face=symbol>j</font>hantom|\emphtext\verb||, and
  \verb|\phantom|\emphtext\verb||.

A phantom looks like this: \phantomphantom yes, now the phantom is gone.

&lt;p&gt;
&lt;&lt;ap\verb|&lt;&lt;ap |could be used to put something in the margin. However, there are better alternatives for that.
See \verb|\marginpar|. \marginpar<font size="+3"> test</font> in margin here


<p><a name="toc.6"><h1>6&nbsp;Spacing</h1>

The following horizontal spaces are supported:
<br>
[8pt]
Negative thin space: |<font face=symbol>Ø</font>thinspace| <br>

No space (natural): || <br>

Thin space: |&nbsp;| or |\thinspace| <br>

Normal space: | | <br>

Normal space: | | <br>

Non<font face=symbol>-</font>break space: |&nbsp;| <br>

en<font face=symbol>-</font>space: |\enspace| <br>

em<font face=symbol>-</font>space: |&nbsp;&nbsp;&nbsp;| <br>

2x em<font face=symbol>-</font>space: |&nbsp;&nbsp;&nbsp;&nbsp;&nbsp;&nbsp;|<br>

3cm horizontal space: |\hspace3cm| <br>




<p><a name="toc.7"><h1>7&nbsp;Environments</h1>

<p><a name="toc.7.1"><h2>7.1&nbsp;Lists: Itemize, Enumerate, and Description</h2>

The \textttitemize environment is suitable for simple lists, the \textttenumerate environment for
enumerated lists, and the \textttdescription environment for descriptions.

&lt;ol&gt;
    &lt;li&gt; You can nest the list environments to your taste:
        &lt;ul&gt;
            &lt;li&gt; But it might start to look silly.
            &lt;br&gt;<font face=symbol>-</font>&nbsp; With a dash.
        &lt;/ul&gt;
    &lt;li&gt; Therefore remember: <a name="refremember">

        \begindescription
            &lt;br&gt;Stupid&nbsp; things will not become smart because they are in a list.
            &lt;br&gt;Smart&nbsp; things, though, can be presented beautifully in a list.
        \enddescription
    &lt;br&gt;important&nbsp; Technical note: Viewing this in Chrome, however, will show too much vertical space
        at the end of a nested environment (see above). On top of that, margin collapsing for inline<font face=symbol>-</font>block
        boxes is not allowed. Maybe using \textttdl elements is too complicated for this and a simple nested
        \textttdiv should be used instead.
&lt;/ol&gt;
Lists can be deeply nested:
&lt;ul&gt;
  &lt;li&gt; list text, level one
    &lt;ul&gt;
      &lt;li&gt; list text, level two
        &lt;ul&gt;
          &lt;li&gt; list text, level three

            And a new paragraph can be started, too.
            &lt;ul&gt;
              &lt;li&gt; list text, level four

                And a new paragraph can be started, too.
                This is the maximum level.

              &lt;li&gt; list text, level four
            &lt;/ul&gt;

          &lt;li&gt; list text, level three
        &lt;/ul&gt;
      &lt;li&gt; list text, level two
    &lt;/ul&gt;
  &lt;li&gt; list text, level one
  &lt;li&gt; list text, level one
&lt;/ul&gt;


<p><a name="toc.7.2"><h2>7.2&nbsp;Flushleft, Flushright, and Center</h2>

The \textttflushleft environment:
\beginflushleft
This text is<br>
 left<font face=symbol>-</font>aligned.
\LaTeX is not trying to make
each line the same length.
\endflushleft
The \textttflushright environment:
\beginflushright
This text is right<font face=symbol>-</font><br>
aligned.
\LaTeX is not trying to make
each line the same length.
\endflushright
And the \textttcenter environment:
\begincenter
At the centre<br>
of the earth
\endcenter



<p><a name="toc.7.3"><h2>7.3&nbsp;Quote, Quotation, and Verse</h2>

The \textttquote environment is useful for quotes, important phrases and examples.
A typographical rule of thumb for the line length is:
\beginquote
On average, no line should be longer than 66 characters.
\endquote

There are two similar environments: the \textttquotation and the \textttverse environments.
The \textttquotation environment is useful for longer quotes going over several paragraphs,
because it indents the first line of each paragraph.

The \textttverse environment is useful for poems where the line breaks are important.
The lines are separated by issuing a \texttt\textbackslash\textbackslash at the end of a line
and an empty line after each verse.

\beginverse
Humpty Dumpty sat on a wall:<br>

Humpty Dumpty had a great fall.<br>

All the King’s horses and all the King’s men<br>

Couldn’t put Humpty together again.

<font face=symbol>®</font>ggedleft <font face=symbol>-</font><font face=symbol>-</font><font face=symbol>-</font>J.W. Elliott<br>
\endverse


<p><a name="toc.7.4"><h2>7.4&nbsp;Picture</h2>

\frame\setlength\unitlength20.4mm
\beginpicture(3,2.1)(<font face=symbol>-</font>1.2,<font face=symbol>-</font>0.05)
  \put(0,1)<span style="position:relative;top:-9pt;left:6pt;"><font face="symbol" size="-1">&#174;</font></span><span style="position:relative;left:-5pt;">t</span>or(1,0)1
  \put(0,1)<font face=symbol>°</font>le2
  \thicklines
  \put(0,0)\line(1,0)1
  \put(0,0.01)xxxxxxxxxxx
  \put(0,0.1)XXXX
\endpicture
\frame\setlength\unitlength1mm
\beginpicture(60, 50)
  \put(20,30)<font face=symbol>°</font>le1
  \put(20,30)<font face=symbol>°</font>le2
  \put(20,30)<font face=symbol>°</font>le4
  \put(20,30)<font face=symbol>°</font>le8
  \put(20,30)<font face=symbol>°</font>le16
  \put(20,30)<font face=symbol>°</font>le32
  \put(40,30)<font face=symbol>°</font>le1
  \put(40,30)<font face=symbol>°</font>le2
  \put(40,30)<font face=symbol>°</font>le3
  \put(40,30)<font face=symbol>°</font>le4
  \put(40,30)<font face=symbol>°</font>le5
  \put(40,30)<font face=symbol>°</font>le6
  \put(40,30)<font face=symbol>°</font>le7
  \put(40,30)<font face=symbol>°</font>le8
  \put(40,30)<font face=symbol>°</font>le9
  \put(40,30)<font face=symbol>°</font>le10
  \put(40,30)<font face=symbol>°</font>le11
  \put(40,30)<font face=symbol>°</font>le12
  \put(40,30)<font face=symbol>°</font>le13
  \put(40,30)<font face=symbol>°</font>le14
  \put(15,10)<font face=symbol>°</font>le*1
  \put(20,10)<font face=symbol>°</font>le*2
  \put(25,10)<font face=symbol>°</font>le*3
  \put(30,10)<font face=symbol>°</font>le*4
  \put(35,10)<font face=symbol>°</font>le*5
\endpicture

\frame\setlength\unitlength0.75mm
\beginpicture(60,40)
  \put(30,20)<span style="position:relative;top:-9pt;left:6pt;"><font face="symbol" size="-1">&#174;</font></span><span style="position:relative;left:-5pt;">t</span>or(1,0)30
  \put(30,20)<span style="position:relative;top:-9pt;left:6pt;"><font face="symbol" size="-1">&#174;</font></span><span style="position:relative;left:-5pt;">t</span>or(4,1)20
  \put(30,20)<span style="position:relative;top:-9pt;left:6pt;"><font face="symbol" size="-1">&#174;</font></span><span style="position:relative;left:-5pt;">t</span>or(3,1)25
  \put(30,20)<span style="position:relative;top:-9pt;left:6pt;"><font face="symbol" size="-1">&#174;</font></span><span style="position:relative;left:-5pt;">t</span>or(2,1)30
  \put(30,20)<span style="position:relative;top:-9pt;left:6pt;"><font face="symbol" size="-1">&#174;</font></span><span style="position:relative;left:-5pt;">t</span>or(1,2)10
  \thicklines
  \put(30,20)<span style="position:relative;top:-9pt;left:6pt;"><font face="symbol" size="-1">&#174;</font></span><span style="position:relative;left:-5pt;">t</span>or(<font face=symbol>-</font>4,1)30
  \put(30,20)<span style="position:relative;top:-9pt;left:6pt;"><font face="symbol" size="-1">&#174;</font></span><span style="position:relative;left:-5pt;">t</span>or(<font face=symbol>-</font>1,4)5
  \thinlines
  \put(30,20)<span style="position:relative;top:-9pt;left:6pt;"><font face="symbol" size="-1">&#174;</font></span><span style="position:relative;left:-5pt;">t</span>or(<font face=symbol>-</font>1,<font face=symbol>-</font>1)5
  \put(30,20)<span style="position:relative;top:-9pt;left:6pt;"><font face="symbol" size="-1">&#174;</font></span><span style="position:relative;left:-5pt;">t</span>or(<font face=symbol>-</font>1,<font face=symbol>-</font>4)5
\endpicture
\setlength\unitlength5cm
\beginpicture(1,1)
  \put(0,0)\line(0,1)1
  \put(0,0)\line(1,0)1
  \put(0,0)\line(1,1)1
  \put(0,0)\line(1,2).5
  \put(0,0)\line(1,3).3333
  \put(0,0)\line(1,4).25
  \put(0,0)\line(1,5).2
  \put(0,0)\line(1,6).1667
  \put(0,0)\line(2,1)1
  \put(0,0)\line(2,3).6667
  \put(0,0)\line(2,5).4
  \put(0,0)\line(3,1)1
  \put(0,0)\line(3,2)1
  \put(0,0)\line(3,4).75
  \put(0,0)\line(3,5).6
  \put(0,0)\line(4,1)1
  \put(0,0)\line(4,3)1
  \put(0,0)\line(4,5).8
  \put(0,0)\line(5,1)1
  \put(0,0)\line(5,2)1
  \put(0,0)\line(5,3)1
  \put(0,0)\line(5,4)1
  \put(0,0)\line(5,6).8333
  \put(0,0)\line(6,1)1
  \put(0,0)\line(6,5)1
\endpicture


\frame
  \setlength\unitlength1cm
  \beginpicture(6,5)
    \thicklines
    \put(1,0.5)\line(2,1)3
    \put(4,2)\line(<font face=symbol>-</font>2,1)2
    \put(2,3)\line(<font face=symbol>-</font>2,<font face=symbol>-</font>5)1
    \put(0.7,0.3)A
    \put(4.05,1.9)B
    \put(1.7,2.9)C
    \put(3.1,2.5)a
    \put(1.3,1.7)b
    \put(2.5,1)c
    \put(0.3,4)F=<font face=symbol>Ö</font><span style="border-top:1 solid black;">s(s-a)(s-b)(s-c)</span>
    \put(3.5,0.4)\displaystyle s:=(a+b+c)/(2)
  \endpicture


\setlength\unitlength2mm
\beginpicture(30,20)
  \linethickness0.075mm
  <font face=symbol>m</font>ltiput(0,0)(1,0)26\line(0,1)20
  <font face=symbol>m</font>ltiput(0,0)(0,1)21\line(1,0)25
  \linethickness0.15mm
  <font face=symbol>m</font>ltiput(0,0)(5,0)6\line(0,1)20
  <font face=symbol>m</font>ltiput(0,0)(0,5)5\line(1,0)25
  \linethickness0.3mm
  <font face=symbol>m</font>ltiput(5,0)(10,0)2\line(0,1)20
  <font face=symbol>m</font>ltiput(0,5)(0,10)2\line(1,0)25
\endpicture



<p><a name="toc.8"><h1>8&nbsp;Labels and References</h1>

Labels can be attached to parts, chapters, sections, items of enumerations, footnotes, tables and figures.
For instance: item&nbsp;<a href="#refremember">(0)</a>
 was important, and regarding fonts, read Section&nbsp;<a href="#refsec:advice">(0)</a>
. And
below, we can reference item&nbsp;<a href="#refkey-1">(0)</a>
 and <a href="#refkey-2">(0)</a>
.

&lt;ol&gt;
  &lt;li&gt; list text, level one
    &lt;ol&gt;
      &lt;li&gt; list text, level two
        &lt;ol&gt;
          &lt;li&gt; list text, level three

            And a new paragraph can be started, too.
            &lt;ol&gt;
              &lt;li&gt; list text, level four

                And a new paragraph can be started, too.
                This is the maximum level.

              &lt;li&gt; list text, level four <a name="refkey-1">

            &lt;/ol&gt;

          &lt;li&gt; list text, level three
        &lt;/ol&gt;
      &lt;li&gt;<a name="refkey-2">
 list text, level two
    &lt;/ol&gt;
  &lt;li&gt; list text, level one
  &lt;li&gt; list text, level one
&lt;/ol&gt;


<p><a name="toc.9"><h1>9&nbsp;Mathematical Formulae</h1>

Math is typeset using KaTeX. Inline math:

f(x) = <font face=symbol>ò</font>_-<font face=symbol>¥</font><sup><font face=symbol>¥</font></sup>
</td>
<td nowrap align=center>
   <span style="position:relative;top:-7pt;left:4pt;">^</span><span style="position:relative;left:-4pt;">f</span>(<font face=symbol>x</font>)&nbsp;e<sup>2 <font face=symbol>p</font> i <font face=symbol>x</font> x</sup>
</td>
<td nowrap align=center>
   &nbsp; d<font face=symbol>x</font>

as well as display math is supported:

  </td>
</tr>
</table>

f(n) = \begincases (n)/(2), & \textif  n\text is even <br>
 3n+1, & \textif  n\text is odd \endcases

<table cellspacing=0  border=0 align=center>
<tr>
  <td nowrap align=center>
    


<p><a name="toc.10"><h1>10&nbsp;Groups</h1>


Today is \today.

Actually, what about  some groups?  They&nbsp;are     nice.


<p><a name="toc.11"><h1>11&nbsp;Symbols</h1>
<a name="refsec:symbols">
    

&lt;p&gt;
lowercase greek letters:
\textalpha \textbeta \textgamma \textdelta \textepsilon \textzeta \texteta \texttheta \textiota \textkappa
\textlambda \textmu \textnu \textxi \textomikron \textpi \textrho \textsigma \texttau \textupsilon \textphi \textchi
\textpsi \textomega

&lt;p&gt;
uppercase greek letters:
\textAlpha \textBeta \textGamma \textDelta \textEpsilon \textZeta \textEta \textTheta \textIota \textKappa
\textLambda \textMu \textNu \textXi \textOmikron \textPi \textRho \textSigma \textTau \textUpsilon \textPhi \textChi
\textPsi \textOmega

&lt;p&gt;
currencies:
\texteuro \textcent \textsterling \pounds \textbaht \textcolonmonetary \textcurrency \textdong \textflorin \textlira
\textnaira \textpeso \textwon \textyen

&lt;p&gt;
old<font face=symbol>-</font>style numerals:
\textzerooldstyle \textoneoldstyle \texttwooldstyle \textthreeoldstyle \textfouroldstyle \textfiveoldstyle
\textsixoldstyle \textsevenoldstyle \texteightoldstyle \textnineoldstyle

&lt;p&gt;
math:
\textperthousand \perthousand \textpertenthousand \textonehalf \textthreequarters \textonequarter
\textfractionsolidus \textdiv \texttimes \textminus \textpm \textsurd \textlnot \textasteriskcentered
\textonesuperior \texttwosuperior \textthreesuperior

&lt;p&gt;
arrows:
\textleftarrow \textuparrow \textrightarrow \textdownarrow

&lt;p&gt;
misc:
\checkmark \textreferencemark \textordfeminine \textordmasculine \textmarried \textdivorced \textbar \textbardbl
\textbrokenbar \textbigcircle \textcopyright \copyright \textcircledP \textregistered \textservicemark
\texttrademark \textnumero \textrecipe \textestimated \textmusicalnote \textdiscount

&lt;p&gt;
non<font face=symbol>-</font>ASCII:
\AE \ae \IJ <font face=symbol>i</font>j \OE \oe \TH \th \SS \ss \DH \dh \O \o \DJ \dj \L \l <font face=symbol>i</font> j \NG \ng


<p><a name="toc.12"><h1>12&nbsp;Fonts</h1>

Usually, \LaTeX.js chooses the right font<font face=symbol>-</font><font face=symbol>-</font><font face=symbol>-</font>just like \LaTeX.  In some cases,
one might like to change fonts and sizes by hand. To do this, use the standard
commands. The actual size of each font is a design issue and depends
on the document class (in this case on the CSS file).

<font size="-1"> The small and
  \textbfbold Romans ruled</font>
  <font size="+2"> all of great big
  \textitItaly.</font>

\textitYou can also
  \emphemphasize text if
  it is set in italics,
  \textsfin a
  \emphsans<font face=symbol>-</font>serif font,
  \textttor in
  \emphtypewriter style.

The environment form of the font commands is available, too:

\begincenter
\beginitshape
This whole paragraph is emphasized, for instance.
\enditshape
\endcenter


<p><a name="toc.12.1"><h2>12.1&nbsp;An advice</h2>
<a name="refsec:advice">
    
\begincenter
  \underline\textbfRemember\Huge! \textitThe
  \textsfM\textbf\LARGE O\textttR\textslE fonts \Huge you
  \tiny use \footnotesize \textbfin a <font size="-1"> \textttdocument,
  <font size="+1"> \textitthe <font size="+0"> more \textscreadable and
  \textsl\textsfbeautiful it bec<font size="+1"> o<font size="+2"> m\LARGE e<font size="+3"> s</font></font></font>.
\endcenter


\appendix

<p><a name="toc.13"><h1>13&nbsp;Source</h1>

The source of \LaTeX.js is here on GitHub: \urlhttps://github.com/michael<font face=symbol>-</font>brade/LaTeX.js

\enddocument

  
<hr>
<p><h1>Table Of Contents</h1>
<p><a href="#toc.1"><h1>1&nbsp;Characters</h1></a>
<p><a href="#toc.2"><h1>2&nbsp;Spaces and Comments</h1></a>
<p><a href="#toc.3"><h1>3&nbsp;Dashes and Hyphens</h1></a>
<p><a href="#toc.4"><h1>4&nbsp;Text and Paragraphs, Ligatures</h1></a>
<p><a href="#toc.4.1"><h2>4.1&nbsp;Multicolumns</h2></a>
<p><a href="#toc.5"><h1>5&nbsp;Boxes</h1></a>
<p><a href="#toc.5.1"><h2>5.1&nbsp;Low<font face=symbol>-</font>level box<font face=symbol>-</font>interface</h2></a>
<p><a href="#toc.6"><h1>6&nbsp;Spacing</h1></a>
<p><a href="#toc.7"><h1>7&nbsp;Environments</h1></a>
<p><a href="#toc.7.1"><h2>7.1&nbsp;Lists: Itemize, Enumerate, and Description</h2></a>
<p><a href="#toc.7.2"><h2>7.2&nbsp;Flushleft, Flushright, and Center</h2></a>
<p><a href="#toc.7.3"><h2>7.3&nbsp;Quote, Quotation, and Verse</h2></a>
<p><a href="#toc.7.4"><h2>7.4&nbsp;Picture</h2></a>
<p><a href="#toc.8"><h1>8&nbsp;Labels and References</h1></a>
<p><a href="#toc.9"><h1>9&nbsp;Mathematical Formulae</h1></a>
<p><a href="#toc.10"><h1>10&nbsp;Groups</h1></a>
<p><a href="#toc.11"><h1>11&nbsp;Symbols</h1></a>
<p><a href="#toc.12"><h1>12&nbsp;Fonts</h1></a>
<p><a href="#toc.12.1"><h2>12.1&nbsp;An advice</h2></a>
<p><a href="#toc.13"><h1>13&nbsp;Source</h1></a>
</body>
</html>
