<html>
<head>
<title>LaTeX4Web 1.4 OUTPUT</title>
<style type="text/css">
<!--
 body {color: black;  background:"#FFCC99";  }
 div.p { margin-top: 7pt;}
 td div.comp { margin-top: -0.6ex; margin-bottom: -1ex;}
 td div.comb { margin-top: -0.6ex; margin-bottom: -.6ex;}
 td div.norm {line-height:normal;}
 td div.hrcomp { line-height: 0.9; margin-top: -0.8ex; margin-bottom: -1ex;}
 td.sqrt {border-top:2 solid black;
          border-left:2 solid black;
          border-bottom:none;
          border-right:none;}
 table.sqrt {border-top:2 solid black;
             border-left:2 solid black;
             border-bottom:none;
             border-right:none;}
-->
</style>
</head>
<body>
\documentclassarticle
\usepackage[utf8]ctex

\begindocument
	
	\maketitle
	
	<h3>2.1 整式</h3>
	
	\beginarticle
		                     
<table cellspacing=0  border=0 align=center>
<tr>
  <td nowrap align=center>
    100u+120(u<font face=symbol>-</font>0.5)
  </td>
</tr>
</table>
,
		      冻土地段与非冻土地段相差(单位:km)
		                     100u<font face=symbol>-</font>12a0(u<font face=symbol>-</font>0.5),
		      上面的式子都带有括号,类比数的运算,他们应如何化简?
		           利用分配律,可以去括号,再合并同类项,得,
		                     100u+120(u<font face=symbol>-</font>0.5)=100u+120u<font face=symbol>-</font>60=220u<font face=symbol>-</font>60,
		                     100u<font face=symbol>-</font>120(u<font face=symbol>-</font>0.5)=100u<font face=symbol>-</font>120u+60=<font face=symbol>-</font>20u+60,
		           上面两式中
		                     +120(u<font face=symbol>-</font>0.5)=+120u<font face=symbol>-</font>60,
		                     <font face=symbol>-</font>120(u<font face=symbol>-</font>0.5)=<font face=symbol>-</font>120u+60,
		           比较上面两式,你能发现去括号时符号变化的规律吗?
		           如果括号外的因数是正数,去括号后原括号内各项的符号与原来的符号相同;
		           如果括号外的因数是负数,去括号后原括号内各项的符号与原来的符号相反。
		           特别的,+(x<font face=symbol>-</font>3)与<font face=symbol>-</font>(x<font face=symbol>-</font>3)可以分别看做1与-1分别乘(x<font face=symbol>-</font>3),利用分配律,可以将式子中的括号去掉,得
		                     
<table cellspacing=0  border=0 align=center>
<tr>
  <td nowrap align=center>
    +(x<font face=symbol>-</font>3)=x<font face=symbol>-</font>3
  </td>
</tr>
</table>

		                     
<table cellspacing=0  border=0 align=center>
<tr>
  <td nowrap align=center>
    <font face=symbol>-</font>(x<font face=symbol>-</font>3)=<font face=symbol>-</font>x+3
  </td>
</tr>
</table>

		           这也符合以上发现的去括号规律。
		               我们可以利用上面的去括号规律进行整式化简。
		
		\beginexample
			例4 化简下列各式:
			(1)  8a+2b+(5a<font face=symbol>-</font>b);  (2)  (5a<font face=symbol>-</font>3b)<font face=symbol>-</font>3(a<sup>2</sup><font face=symbol>-</font>2b)
			 解: (1)  8a+2b+(5a<font face=symbol>-</font>b)=8a+2b+5a<font face=symbol>-</font>b=13a+b
			      (2)  (5a<font face=symbol>-</font>3b)<font face=symbol>-</font>3(a<sup>2</sup><font face=symbol>-</font>2b)=5a<font face=symbol>-</font>3b<font face=symbol>-</font>3a<sup>2</sup>+6b=<font face=symbol>-</font>3a<sup>2</sup>+5a+3b
			例5  两船从同一港口同时出发反向而行,甲船顺水,乙船逆水,两船在静水中的速度都是50千米/时,水流的速度是a千米/时。
			 (1)  2h后两船相距多远?
			  (2)    2h后后甲船比乙船多航行多少千米?
			 解: 顺水航速=船速+水速=(50+a)(千米/时);
			      逆水航速=船速-水速=(50-a)(千米/时);
	              (1) 2h后两船相距(单位:km)
	                      2(50+a)+2(50<font face=symbol>-</font>a)=100+2a+100<font face=symbol>-</font>2a=200.
	              (2) 2h后甲船比乙船多航行(单位:km)
	                      2(50+a)<font face=symbol>-</font>2(50<font face=symbol>-</font>a)=100+2a<font face=symbol>-</font>100+2a=4a.
		\endexample
		
		\beginexercise
			
			
			1. 化简:
			
			(1) 12(x<font face=symbol>-</font>0.5);     (2) <font face=symbol>-</font>5a+(3a<font face=symbol>-</font>2)<font face=symbol>-</font>(3a<font face=symbol>-</font>7);
			
			2. 飞机的无风航速为a km/h,飞速为20km/h。飞机顺风飞行4h的行程是多少?飞机逆风飞行3h的行程是多少?两个行程相差多少?
			
		\endexercise
	
		\beginexample
			例6 计算:
			(1)  (2x<font face=symbol>-</font>3y)+(5x+4y);  (2)  (8a<font face=symbol>-</font>7b)<font face=symbol>-</font>(4a<font face=symbol>-</font>5b).
			解: (1)  (2x<font face=symbol>-</font>3y)+(5x+4y)=2x<font face=symbol>-</font>3y+5x+4y=7x+y
			(2)  (8a<font face=symbol>-</font>7b)<font face=symbol>-</font>(4a<font face=symbol>-</font>5b)=8a<font face=symbol>-</font>7b<font face=symbol>-</font>4a+5b=<font face=symbol>-</font>4a<font face=symbol>-</font>2b
			例7  笔记本的单价是x元,圆珠笔的单价是y元,小红买这种笔记本3本,买圆珠笔2支;小明买这种笔记本4本,买圆珠笔3支.买这种笔记本和圆珠笔,小红和小明一共花费多少元?
			解法1: 小红买笔记本和圆珠笔共花费(3x+2y)元,小明买笔记本和圆珠笔共花费(4x+3y)元。
			       小红和小明一共花费(单位:元)
			            (3x+2y)+(4x+3y)=3x+2y+4x+3y=7x+5y
			解法2: 小红和小明买笔记本共花费(3x+4x)元,买圆珠笔共花费(2y+3y)元。
			        小红和小明一共花费(单位:元)
			            (3x+4x)+(2y+3y)=7x+5y
		\endexample
		 通过上面的学习,我们可以得到整式加减的预算法则:
		 一般地,几个整式相加减,如果有括号就先去括号,然后再合并同类项。
		\beginexample
			例8 计算: (1)/(2)x<font face=symbol>-</font>2(x<font face=symbol>-</font>(1)/(3)y<sup>2</sup>)+(<font face=symbol>-</font>(3)/(2)x+(1)/(3)y<sup>2</sup>)的值,其x=-2,y=frac23.
			解:   (1)/(2)x<font face=symbol>-</font>2(x<font face=symbol>-</font>(1)/(3)y<sup>2</sup>)+(<font face=symbol>-</font>(3)/(2)x+(1)/(3)y<sup>2</sup>)
			     =(1)/(2)x<font face=symbol>-</font>2x+(1)/(3)y<sup>2</sup><font face=symbol>-</font>(3)/(2)x+(1)/(3)y<sup>2</sup>
			     =<font face=symbol>-</font>3x+y<sup>2</sup>
			当x=-2,y=frac23时,原式=(<font face=symbol>-</font>3)&times;(<font face=symbol>-</font>2)+((2)/(3))<sup>2</sup>=6(2)/(3)
		\endexample
	    \beginexercise
	    	1. 化简:
	    	
	    	(1) 3xy<font face=symbol>-</font>4xy<font face=symbol>-</font>(<font face=symbol>-</font>2xy);     (2) <font face=symbol>-</font>(1)/(3)ab<font face=symbol>-</font>(1)/(4)a<sup>2</sup>+(1)/(3)a<sup>2</sup><font face=symbol>-</font>(<font face=symbol>-</font>(2)/(3)ab);
	    	
	    	2. 计算:
	    	(1) (<font face=symbol>-</font>x+2x<sup>2</sup>+5)+(4x<sup>2</sup><font face=symbol>-</font>3<font face=symbol>-</font>6x);   (2)(3a<sup>2</sup><font face=symbol>-</font>ab+7)<font face=symbol>-</font>(<font face=symbol>-</font>4a<sup>2</sup>+2ab+7).
	    	
	    	3. 先化简下式,再求值:
	    	                5(3a<sup>2</sup>b<font face=symbol>-</font>ab<sup>2</sup>)<font face=symbol>-</font>(ab<sup>2</sup>+3a<sup>2</sup>b),
	    	   其中a=(1)/(2),b=(1)/(3).
	    	
	    \endexercise
         复习巩固
         \beginexercise
         	1. 计算:
         	
         	(1) 2x<font face=symbol>-</font>10.3x;     (2) 3x<font face=symbol>-</font>x<font face=symbol>-</font>5x;
         	(3) <font face=symbol>-</font>b+0.6b<font face=symbol>-</font>2.6b;  (4) m<font face=symbol>-</font>n<sup>2</sup>+m<font face=symbol>-</font>n<sup>2</sup>.
         	
         	2. 计算:
         	(1) 2(4x<font face=symbol>-</font>0.5);           (2)<font face=symbol>-</font>3(1<font face=symbol>-</font>(1)/(6)x);
         	(3) <font face=symbol>-</font>x+(2x<font face=symbol>-</font>2)<font face=symbol>-</font>(3x+5);    (4) 3a<sup>2</sup>+a<sup>2</sup><font face=symbol>-</font>(2a<sup>2</sup><font face=symbol>-</font>2a)+(3a<font face=symbol>-</font>a<sup>2</sup>).
         	
         	3. 计算:
            (1) (5a+4c+7b)+(5c<font face=symbol>-</font>3b<font face=symbol>-</font>6a);         (2)(8xy<font face=symbol>-</font>x<sup>2</sup>+y<sup>2</sup>)<font face=symbol>-</font>(x<sup>2</sup><font face=symbol>-</font>y<sup>2</sup>+8xy);
            (3) (2x<sup>2</sup><font face=symbol>-</font>(1)/(2)+3x)<font face=symbol>-</font>4(x<font face=symbol>-</font>x<sup>2</sup>+(1)/(2));    (4) 3x<sup>2</sup><font face=symbol>-</font>[7x<font face=symbol>-</font>(4x<font face=symbol>-</font>3)<font face=symbol>-</font>2x].
            
            4. 先化简下式,再求值:
                     (<font face=symbol>-</font>x<sup>2</sup>+5+4x)+(5x<font face=symbol>-</font>4+2x<sup>2</sup>),
                其中,x=-2.
            
            5. (1) 列式表示比a的5倍大4的数与比a的2倍小3的数,计算这两个数的和;
               (2) 列式表示比x的7倍大3的数与比x的6倍小5的数,计算这两个数的差。
            
            6. 某村小麦的面积是a公顷,水稻种植的面积是小麦种植面积的3倍,玉米种植面积比小麦种植面积少5公顷.列式表示水稻种植面积,玉米种植面积,并计算水稻种植面积比玉米种植面积大多少。
          	
         \endexercise
	\endarticle
\enddocument</body>
</html>
