<html>
<head>
<title>LaTeX4Web 1.4 OUTPUT</title>
<style type="text/css">
<!--
 body {color: black;  background:"#FFCC99";  }
 div.p { margin-top: 7pt;}
 td div.comp { margin-top: -0.6ex; margin-bottom: -1ex;}
 td div.comb { margin-top: -0.6ex; margin-bottom: -.6ex;}
 td div.norm {line-height:normal;}
 td div.hrcomp { line-height: 0.9; margin-top: -0.8ex; margin-bottom: -1ex;}
 td.sqrt {border-top:2 solid black;
          border-left:2 solid black;
          border-bottom:none;
          border-right:none;}
 table.sqrt {border-top:2 solid black;
             border-left:2 solid black;
             border-bottom:none;
             border-right:none;}
-->
</style>
</head>
<body>
\documentclass[UTF8]article
\usepackageCTEX
\usepackagecolor
\usepackageindentfirst
\setlength<br>indent2em
\usepackagetxfonts
\usepackagetextcomp
\usepackagegraphicx
\begindocument
	<p>\heiti 1.2.4 绝对值<br>

	<p>&nbsp;&nbsp;&nbsp;&nbsp;两辆汽车从同一处O出发,分别向东、西方向行驶10km,到达A,B两处(图1.2-6).它们的行驶路线相同吗?它们的行驶路程相同?<br>

	\beginfigure[ht]
		\centering
		 <font face=symbol>Î</font> cludegraphics[scale=0.5]1.png
	\endfigure

	<p>&nbsp;&nbsp;&nbsp;&nbsp;一般地,数轴上表示数a的点与原点的距离叫做数a的\textcolorblue绝对值(absolute value),记作<font face=symbol>|</font> a <font face=symbol>|</font>.例如,图1.2-6中A,B两点分别表示10和-10,它们与原点的距离10个单位长度,所以10和-10的绝对值都是10,即<br>

	<p>&nbsp;&nbsp;&nbsp;&nbsp;<font face=symbol>|</font> 10 <font face=symbol>|</font> = 10,<font face=symbol>|</font> <font face=symbol>-</font>10 <font face=symbol>|</font> = 10.<br>

	<p>&nbsp;&nbsp;&nbsp;&nbsp;显然<font face=symbol>|</font> 0 <font face=symbol>|</font> = 0.<br>

	<p>&nbsp;&nbsp;&nbsp;&nbsp;由绝对值的定义可知:<br>

	<p>&nbsp;&nbsp;&nbsp;&nbsp;\textcolorblue一个正数的绝对值是它本身;一个负数的绝对值是它的相反数;0的绝对值是0。即<br>

	<p>&nbsp;&nbsp;&nbsp;&nbsp;(1) 如果a&gt;0,那么<font face=symbol>|</font> a <font face=symbol>|</font> = a;<br>

	<p>&nbsp;&nbsp;&nbsp;&nbsp;(2)如果a=0,那么<font face=symbol>|</font> a <font face=symbol>|</font> = 0;<br>

	<p>&nbsp;&nbsp;&nbsp;&nbsp;(3)如果a&lt;0,那么<font face=symbol>|</font> a <font face=symbol>|</font> = -a;<br>

	<p>&nbsp;&nbsp;&nbsp;&nbsp; 这里的数a可以使正数、负数和0.<br>

	\textcolorblue\heiti 练习<br>

	1.写出下列个数的绝对值:<br>

	
<table cellspacing=0  border=0 align=center>
<tr>
  <td nowrap align=center>
    6,<font face=symbol>-</font>8,<font face=symbol>-</font>3.9,
  </td>
  <td nowrap align="center">
    <table cellspacing=0 border=0 >
    <tr>
      <td nowrap align="center">
        5
      </td>
    </tr>
    </table>
    <div class=hrcomp><hr noshade size=1></div>
    <table cellspacing=0 border=0 >
    <tr>
      <td nowrap align=center>
        2
      </td>
    </tr>
    </table>
  </td>
  <td nowrap align=center>
    ,<font face=symbol>-</font>
  </td>
  <td nowrap align="center">
    <table cellspacing=0 border=0 >
    <tr>
      <td nowrap align="center">
        2
      </td>
    </tr>
    </table>
    <div class=hrcomp><hr noshade size=1></div>
    <table cellspacing=0 border=0 >
    <tr>
      <td nowrap align=center>
        11
      </td>
    </tr>
    </table>
  </td>
  <td nowrap align=center>
    ,100,0
  </td>
</tr>
</table>

	2.判断下列说法是否正确:
	<p>&nbsp;&nbsp;&nbsp;&nbsp;(1)符号相反的数互为相反数;<br>

	<p>&nbsp;&nbsp;&nbsp;&nbsp;(2)一个数的绝对值越大,表示它的点在数轴上越靠右;<br>

	<p>&nbsp;&nbsp;&nbsp;&nbsp;(3)一个数的绝对值越大,表示它的点在数轴上离原点越远;<br>

	<p>&nbsp;&nbsp;&nbsp;&nbsp;(4)当a <font face=symbol>¹</font> 0时,<font face=symbol>|</font> a <font face=symbol>|</font>总是大于0.<br>

	3.判断下列各式是否正确:<br>

	(1)<font face=symbol>|</font> 5 <font face=symbol>|</font>=<font face=symbol>|</font> <font face=symbol>-</font><font face=symbol>-</font>5 <font face=symbol>|</font>;(2)-<font face=symbol>|</font> 5 <font face=symbol>|</font>=<font face=symbol>|</font> <font face=symbol>-</font>5 <font face=symbol>|</font>;(3)-5=<font face=symbol>|</font> <font face=symbol>-</font>5 <font face=symbol>|</font>.
	<p>&nbsp;&nbsp;&nbsp;&nbsp;我们已知两个正数(或0)之间怎样比较大小,例如<br>

	
<table cellspacing=0  border=0 align=center>
<tr>
  <td nowrap align=center>
    0&lt;1,1&lt;2,2&lt;3···.
  </td>
</tr>
</table>

	<p>&nbsp;&nbsp;&nbsp;&nbsp;任意两个有理数(例如-4和-3,-2和0,-1和1)怎样比较大小呢?<br>

	\textcolorblue\heiti 思考<br>

	<p>&nbsp;&nbsp;&nbsp;&nbsp;图1.2-7给出了未来一周中每天的最高气温和最低气温,其中最低气温是多少?最高气温呢?你能将这七天中每天的最低气温按从低到高的顺序排列吗?<br>

	\beginfigure[ht]
		\centering
		 <font face=symbol>Î</font> cludegraphics[scale=0.5]2.png
	\endfigure
	<p>&nbsp;&nbsp;&nbsp;&nbsp;这七天中每天的最低气温按从低到高排列为<br>

	
<table cellspacing=0  border=0 align=center>
<tr>
  <td nowrap align=center>
    <font face=symbol>-</font>4,<font face=symbol>-</font>3,<font face=symbol>-</font>2,<font face=symbol>-</font>1,0,1,2
  </td>
</tr>
</table>
.
	<p>&nbsp;&nbsp;&nbsp;&nbsp;按照这个顺序排列的温度,在温度计上所对应的点事从下到上的,按照这个顺序把这些书表示在数轴上,表示它们的各点的顺序是从左到右的(图1.2-8).
	\beginfigure[ht]
		\centering
		 <font face=symbol>Î</font> cludegraphics[scale=1]3.png
	\endfigure
	<p>&nbsp;&nbsp;&nbsp;&nbsp;数学中规定:在数轴上表示有理数,它们从左到右的顺序,就是从小到大的顺序,即左边的数小于右边的数.
	<p>&nbsp;&nbsp;&nbsp;&nbsp;由这个规定可知
	
<table cellspacing=0  border=0 align=center>
<tr>
  <td nowrap align=center>
    <font face=symbol>-</font>6&lt;<font face=symbol>-</font>5,<font face=symbol>-</font>5&lt;<font face=symbol>-</font>4,<font face=symbol>-</font>4&lt;<font face=symbol>-</font>3,<font face=symbol>-</font>2&lt;0,<font face=symbol>-</font>1&lt;1,···.
  </td>
</tr>
</table>

	\textcolorblue思考
	<p>&nbsp;&nbsp;&nbsp;&nbsp;对于正数、0和负数这三类数,它们之间有什么大小关系?两个负数之间如何比较大小?前面最低气温由低到高的排列与你的结论一致吗?
	一般地,<br>

	\textcolorblue(1)正数大于0,0大于负数,正数大于负数;<br>

	\textcolorblue(2)两个负数,绝对值大的反而小。<br>

	例如,10,0-1,1-1,-1-2.<br>

	\textcolorblue例 比较下列各对数的大小:<br>

	(1) -(-1)和-(+2); <br>

	(2) <font face=symbol>-</font>(8)/(21) 和 <font face=symbol>-</font>(3)/(7)<br>
     
	(3)-(-0.3)和 <font face=symbol>|</font> <font face=symbol>-</font>(1)/(3) <font face=symbol>|</font>.<br>

	\textcolorblue解:(1)先化简,-(-1) = 1,-(+2) = -2.<br>

	因为正数大于负数,所以1&gt; <font face=symbol>-</font>2,即<br>

	
<table cellspacing=0  border=0 align=center>
<tr>
  <td nowrap align=center>
    <font face=symbol>-</font>(<font face=symbol>-</font>1)&gt;<font face=symbol>-</font>(+2).
  </td>
</tr>
</table>

	(2)这是两个负数比较大小,先求它们的绝对值.<br>

	<font face=symbol>|</font> <font face=symbol>-</font>(8)/(21) <font face=symbol>|</font> = (8)/(12),<font face=symbol>|</font> <font face=symbol>-</font>(3)/(7) <font face=symbol>|</font> = (3)/(7)=(9)/(21).<br>

	因为 (8)/(21) &lt; (9)/(21),<br>

	即	<font face=symbol>|</font> <font face=symbol>-</font>(8)/(21) <font face=symbol>|</font>  &lt; <font face=symbol>|</font> <font face=symbol>-</font>(3)/(7) <font face=symbol>|</font>,<br>

	所以 <font face=symbol>-</font>(8)/(21)   &lt;  <font face=symbol>-</font>(3)/(7).<br>

	(3)先化简,-(-0.3) = 0.3,<font face=symbol>|</font> <font face=symbol>-</font>(1)/(3) <font face=symbol>|</font> = (1)/(3).<br>

	因为 0.3  &lt; (1)/(3).<br>

	所以 -(-0.3)  &lt;  <font face=symbol>|</font> <font face=symbol>-</font>(1)/(3) <font face=symbol>|</font>.<br>

	<p>&nbsp;&nbsp;&nbsp;&nbsp; 异号两数比较大小,要考虑它们的正负:同号两数比较大小,要考虑它们的绝对值。<br>

	\textcolorblue\heiti 练习<br>

	比较下列各对数的大小:<br>

	(1) 3和-5;                        (2) -3和-5;<br>

	(3) -2.5和<font face=symbol>-</font><font face=symbol>|</font> <font face=symbol>-</font>2.25 <font face=symbol>|</font>;     (4) <font face=symbol>-</font>(3)/(5)和<font face=symbol>-</font>(3)/(4).<br>

	\textcolorblue\heiti 习题1.2
	\heiti 复习巩固
	1.把下面的有理数填写在相应的大括号里(将各数用逗号分开):
	
<table cellspacing=0  border=0 align=center>
<tr>
  <td nowrap align=center>
     15,<font face=symbol>-</font>
  </td>
  <td nowrap align="center">
    <table cellspacing=0 border=0 >
    <tr>
      <td nowrap align="center">
        3
      </td>
    </tr>
    </table>
    <div class=hrcomp><hr noshade size=1></div>
    <table cellspacing=0 border=0 >
    <tr>
      <td nowrap align=center>
        8
      </td>
    </tr>
    </table>
  </td>
  <td nowrap align=center>
    ,0,0.15,<font face=symbol>-</font>30,<font face=symbol>-</font>12.8,<font face=symbol>-</font>
  </td>
  <td nowrap align="center">
    <table cellspacing=0 border=0 >
    <tr>
      <td nowrap align="center">
        22
      </td>
    </tr>
    </table>
    <div class=hrcomp><hr noshade size=1></div>
    <table cellspacing=0 border=0 >
    <tr>
      <td nowrap align=center>
        5
      </td>
    </tr>
    </table>
  </td>
  <td nowrap align=center>
    ,+20,<font face=symbol>-</font>60 
  </td>
</tr>
</table>
.
	正数:               负数:                 
	2.在数轴上表示下列各数:
	
<table cellspacing=0  border=0 align=center>
<tr>
  <td nowrap align=center>
    <font face=symbol>-</font>5,+3,<font face=symbol>-</font>3.5,0,
  </td>
  <td nowrap align="center">
    <table cellspacing=0 border=0 >
    <tr>
      <td nowrap align="center">
        2
      </td>
    </tr>
    </table>
    <div class=hrcomp><hr noshade size=1></div>
    <table cellspacing=0 border=0 >
    <tr>
      <td nowrap align=center>
        3
      </td>
    </tr>
    </table>
  </td>
  <td nowrap align=center>
    ,<font face=symbol>-</font>
  </td>
  <td nowrap align="center">
    <table cellspacing=0 border=0 >
    <tr>
      <td nowrap align="center">
        3
      </td>
    </tr>
    </table>
    <div class=hrcomp><hr noshade size=1></div>
    <table cellspacing=0 border=0 >
    <tr>
      <td nowrap align=center>
        2
      </td>
    </tr>
    </table>
  </td>
  <td nowrap align=center>
    ,0.75 
  </td>
</tr>
</table>
.
	3.在数轴上,点A表示-3,从点A出发,沿着数轴移动4个单位长到达点B,则点B表示的数是多少?
	4.写出下列各数的相反数,并将这些数连同它们的相反数在数轴有上表示出来:
	
<table cellspacing=0  border=0 align=center>
<tr>
  <td nowrap align=center>
     <font face=symbol>-</font>4,+2,<font face=symbol>-</font>1.5,0,
  </td>
  <td nowrap align="center">
    <table cellspacing=0 border=0 >
    <tr>
      <td nowrap align="center">
        1
      </td>
    </tr>
    </table>
    <div class=hrcomp><hr noshade size=1></div>
    <table cellspacing=0 border=0 >
    <tr>
      <td nowrap align=center>
        3
      </td>
    </tr>
    </table>
  </td>
  <td nowrap align=center>
    ,<font face=symbol>-</font>
  </td>
  <td nowrap align="center">
    <table cellspacing=0 border=0 >
    <tr>
      <td nowrap align="center">
        9
      </td>
    </tr>
    </table>
    <div class=hrcomp><hr noshade size=1></div>
    <table cellspacing=0 border=0 >
    <tr>
      <td nowrap align=center>
        4
      </td>
    </tr>
    </table>
  </td>
  <td nowrap align=center>
     
  </td>
</tr>
</table>
.
	5.写出下列各数的绝对值:
	
<table cellspacing=0  border=0 align=center>
<tr>
  <td nowrap align=center>
    <font face=symbol>-</font>125,+23,<font face=symbol>-</font>3.5,0,
  </td>
  <td nowrap align="center">
    <table cellspacing=0 border=0 >
    <tr>
      <td nowrap align="center">
        2
      </td>
    </tr>
    </table>
    <div class=hrcomp><hr noshade size=1></div>
    <table cellspacing=0 border=0 >
    <tr>
      <td nowrap align=center>
        3
      </td>
    </tr>
    </table>
  </td>
  <td nowrap align=center>
    ,<font face=symbol>-</font>
  </td>
  <td nowrap align="center">
    <table cellspacing=0 border=0 >
    <tr>
      <td nowrap align="center">
        3
      </td>
    </tr>
    </table>
    <div class=hrcomp><hr noshade size=1></div>
    <table cellspacing=0 border=0 >
    <tr>
      <td nowrap align=center>
        2
      </td>
    </tr>
    </table>
  </td>
  <td nowrap align=center>
    ,<font face=symbol>-</font>0.05
  </td>
</tr>
</table>
.
	上面的数中哪个数的绝对值最大?那个数的绝对值最小?
	6.将下列各数按从小到大的顺序排列,并用“&lt;”号连接:
	
<table cellspacing=0  border=0 align=center>
<tr>
  <td nowrap align=center>
    <font face=symbol>-</font>0.25,+2.3,<font face=symbol>-</font>0.15,0,<font face=symbol>-</font>
  </td>
  <td nowrap align="center">
    <table cellspacing=0 border=0 >
    <tr>
      <td nowrap align="center">
        2
      </td>
    </tr>
    </table>
    <div class=hrcomp><hr noshade size=1></div>
    <table cellspacing=0 border=0 >
    <tr>
      <td nowrap align=center>
        3
      </td>
    </tr>
    </table>
  </td>
  <td nowrap align=center>
    ,<font face=symbol>-</font>
  </td>
  <td nowrap align="center">
    <table cellspacing=0 border=0 >
    <tr>
      <td nowrap align="center">
        3
      </td>
    </tr>
    </table>
    <div class=hrcomp><hr noshade size=1></div>
    <table cellspacing=0 border=0 >
    <tr>
      <td nowrap align=center>
        2
      </td>
    </tr>
    </table>
  </td>
  <td nowrap align=center>
    ,<font face=symbol>-</font>
  </td>
  <td nowrap align="center">
    <table cellspacing=0 border=0 >
    <tr>
      <td nowrap align="center">
        1
      </td>
    </tr>
    </table>
    <div class=hrcomp><hr noshade size=1></div>
    <table cellspacing=0 border=0 >
    <tr>
      <td nowrap align=center>
        2
      </td>
    </tr>
    </table>
  </td>
  <td nowrap align=center>
    ,0.05
  </td>
</tr>
</table>
.
	\heiti 综合应用<br>

	7.下面是我国几个城市某年一月份的平均气温,把它们按从高到低的顺序排列。<br>

	北京  武汉  广州  哈尔滨   南京<br>

	-4.6\textcelsius  3.8\textcelsius  13.1\textcelsius   -19.4\textcelsius  2.4\textcelsius  <br>

	8.如图,检测5个球球,其中超过标准的克数记为正数,不足的克数记为负数,从轻重的角度,哪个球最接近标准?<br>

		\beginfigure[ht]
		\centering
		 <font face=symbol>Î</font> cludegraphics[scale=0.5]4.png
	\endfigure
	9.某年我国人均水资源比上年的增幅是-5.6\%.后续三年各年比上年的增幅分别是-4.0\%,13.0\%,-9.6\%,这些增幅中哪个是最小?增幅是负数说明什么?<br>

	10.在数轴上,表示哪个数的点与表示-2和4的点的距离相等?<br>

	\heiti 拓广探索<br>

	11.(1) -1和0之间还有负数吗?<font face=symbol>-</font>(1)/(2)与0之间呢?如有,请举例。<br>

	(2)-3和-1之间有负整数吗?-2和2之间有哪些整数?<br>

	(3)有比-1大的负整数吗?<br>

	(4)写出3个小于-100并且大于-103的数。<br>

	12.如果<font face=symbol>|</font> x <font face=symbol>|</font> = 2,那么x一定是2吗?如果<font face=symbol>|</font> x <font face=symbol>|</font> = 0,那么x等于几?如果x = -x,那么x等于几?<br>

\enddocument</body>
</html>
