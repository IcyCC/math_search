<html>
<head>
<title>LaTeX4Web 1.4 OUTPUT</title>
<style type="text/css">
<!--
 body {color: black;  background:"#FFCC99";  }
 div.p { margin-top: 7pt;}
 td div.comp { margin-top: -0.6ex; margin-bottom: -1ex;}
 td div.comb { margin-top: -0.6ex; margin-bottom: -.6ex;}
 td div.norm {line-height:normal;}
 td div.hrcomp { line-height: 0.9; margin-top: -0.8ex; margin-bottom: -1ex;}
 td.sqrt {border-top:2 solid black;
          border-left:2 solid black;
          border-bottom:none;
          border-right:none;}
 table.sqrt {border-top:2 solid black;
             border-left:2 solid black;
             border-bottom:none;
             border-right:none;}
-->
</style>
</head>
<body>
\documentclass[11pt]article
\usepackageamsmath
\usepackageCJKutf8
\usepackagectex
\newtheoremexercise
\newtheoremarticle
\newtheoremnature
\newtheoremtip
\begindocument

多个有理数相乘,可以把它们按顺序依次相乘。

\beginexercise
思考<br>

观察下列各式,它们的积是正的还是负的?<br>

2 &times; 3 &times; 4 &times; (<font face=symbol>-</font>5),<br>

2 &times; 3 &times; (<font face=symbol>-</font>4) &times; (<font face=symbol>-</font>5),<br>

2 &times;( <font face=symbol>-</font>3) &times; (<font face=symbol>-</font>4) &times; (<font face=symbol>-</font>5),<br>

(<font face=symbol>-</font>2) &times; (<font face=symbol>-</font>3) &times; (<font face=symbol>-</font>4) &times; (<font face=symbol>-</font>5).<br>

几个不是0的数相乘,积的符号为负因数的个数之间有什么关系?<br>

\endexercise
\beginarticle
归纳<br>

	几个不是0的数相乘,负因数的个数是偶数时,积是正数;负因数的个数是奇数时,积是负数。<br>

\endarticle
\beginexercise
例3 计算:<br>

(1)
 (<font face=symbol>-</font>3) &times; (5)/(6) &times; (<font face=symbol>-</font>(9)/(5)) &times; (<font face=symbol>-</font>(1)/(4)) <br>

(2)
 (<font face=symbol>-</font>5) &times; 6 &times; (<font face=symbol>-</font>(4)/(5)) &times; (1)/(4)

解:(1)<br>

 (<font face=symbol>-</font>3) &times; (5)/(6) &times; (<font face=symbol>-</font>(9)/(5)) &times; (<font face=symbol>-</font>(1)/(4)) <br>

 = <font face=symbol>-</font>3 &times; (5)/(6) &times; (9)/(5) &times; (1)/(4) = <font face=symbol>-</font>(9)/(8) <br>

(2)<br>

 (<font face=symbol>-</font>5) &times; 6 &times; (<font face=symbol>-</font>(4)/(5)) &times; (1)/(4) <br>

 = 5 &times; 6 &times; (4)/(5) &times; (1)/(4) = 6 <br>

\endexercise

\beginexercise
思考<br>

你能看出下式的结果吗?如果能,请说明理由.<br>

7.8 &times; (<font face=symbol>-</font>8.1) &times; 0 &times; (<font face=symbol>-</font>19.6),<br>


几个数相乘,如果其中有因数为0,那么积等于0<br>

\endexercise

\beginexercise
练习<br>

1、口算:<br>

(1) (<font face=symbol>-</font>2) &times; 3 &times; 4 &times; (<font face=symbol>-</font>1);  (2) (<font face=symbol>-</font>5) &times; (<font face=symbol>-</font>3) &times; 4 &times; (<font face=symbol>-</font>2);<br>

(3) (<font face=symbol>-</font>2) &times; (<font face=symbol>-</font>2) &times; (<font face=symbol>-</font>2) &times; (<font face=symbol>-</font>2);  (4) (<font face=symbol>-</font>3) &times; (<font face=symbol>-</font>3) &times; (<font face=symbol>-</font>3) &times; (<font face=symbol>-</font>3).<br>


2、计算:<br>

(1) (<font face=symbol>-</font>5) &times; 8 &times; (<font face=symbol>-</font>7) &times; (<font face=symbol>-</font>0.25);<br>

(2)  (<font face=symbol>-</font>(5)/(12)) &times; (8)/(15) &times; (1)/(2) &times; (<font face=symbol>-</font>(2)/(3)) ;<br>

(3) (<font face=symbol>-</font>1) &times; (<font face=symbol>-</font>(5)/(4)) &times; (8)/(15) &times; (3)/(2) &times; (<font face=symbol>-</font>(2)/(3)) &times; 0 &times; (<font face=symbol>-</font>1) ;<br>

\endexercise
\beginnature
    像前面那样规定有理数乘法法则后,就可以使交换律、结合律与分配律在有理数乘法中仍然成立.<br>

    例如,<br>

         5&times;(<font face=symbol>-</font>6)=<font face=symbol>-</font>30 <br>

         (<font face=symbol>-</font>6)&times; 5 = <font face=symbol>-</font>30 <br>

    即,<br>

         5&times; (<font face=symbol>-</font>6) = (<font face=symbol>-</font>6) &times; 5 <br>

    一般地,在有理数乘法中,两个数相乘,交换因数地位置,积相等。<br>

        乘法交换律:ab=ba.<br>

        又如,      [3&times; (<font face=symbol>-</font>4)] &times; (<font face=symbol>-</font>5) = (<font face=symbol>-</font>12) &times; (<font face=symbol>-</font>5) = 60.<br>

         3&times; [(<font face=symbol>-</font>4) &times; (<font face=symbol>-</font>5)] = 3 &times; 20 = 60 .
        即<br>

        [3&times; (<font face=symbol>-</font>4)] &times; (<font face=symbol>-</font>5) =3&times; [(<font face=symbol>-</font>4) &times; (<font face=symbol>-</font>5)]<br>

        一般地,有理数乘法中,三个数相乘,先把前两个数相乘,或者先把后两个数相乘,积相乘。<br>

            乘法结合律:(a b)c = a(bc).<br>

        再如,<br>

            5 &times; [3+(<font face=symbol>-</font>7)] = 5 &times; (<font face=symbol>-</font>4) = <font face=symbol>-</font>20, <br>

            5 &times; 3+5&times;(<font face=symbol>-</font>7) = 15<font face=symbol>-</font>35 = <font face=symbol>-</font>20 .<br>

        即<br>

             5 &times; [3+(<font face=symbol>-</font>7)] = 5 &times; 3+5&times;(<font face=symbol>-</font>7)<br>

        一般地,有理数乘法中,一个数同两个数地和相乘,等于把这个数分别同这两个数相乘,再把积相加。<br>

            分配律:a(b+c) = ab+ac.<br>

\endnature

\beginexercise
例 4 用两种方法计算 ((1)/(4) + (1)/(6) <font face=symbol>-</font>(1)/(2))&times; 12 .<br>

解法1: ((1)/(4) + (1)/(6) <font face=symbol>-</font>(1)/(2))&times; 12 .<br>

=((3)/(12) + (2)/(12) <font face=symbol>-</font>(6)/(12))&times; 12 <br>

=<font face=symbol>-</font>(1)/(12) &times; 12=<font face=symbol>-</font>1 .<br>

解法2: ((1)/(4) + (1)/(6) <font face=symbol>-</font>(1)/(2))&times; 12 .<br>

 =(1)/(4) &times; 12 +(1)/(6) &times; 12 <font face=symbol>-</font> (1)/(2) &times; 12 <br>

 =3+2<font face=symbol>-</font>6 = <font face=symbol>-</font>1 .
\endexercise

\beginexercise
思考<br>

    比较上面两种解法,它们再运算顺序上有什么区别?解法2用了什么运算律?哪种解法运算量小?<br>

\endexercise

\beginexercise
练习<br>


计算:<br>

(1) (<font face=symbol>-</font>85) &times; (<font face=symbol>-</font>25) &times; (<font face=symbol>-</font>4);
(2)  ((9)/(10) <font face=symbol>-</font> (1)/(15)) &times; 30 ;<br>

(3)  (<font face=symbol>-</font>(7)/(8)) &times; 15 &times; (<font face=symbol>-</font>1(1)/(7)) ;
(4) (<font face=symbol>-</font>(6)/(5)) &times; (<font face=symbol>-</font>(2)/(3)) + (<font face=symbol>-</font>(6)/(5)) &times; (+(17)/(3)) ;<br>

\endexercise

\beginarticle
1.4.2 有理数的除法 <br>

怎么计算  8<font face=symbol>¸</font>(<font face=symbol>-</font>4) 呢?<br>

根据除法是乘法的逆运算,就是要求一个数,使它与-4相乘得8.<br>

因为   (<font face=symbol>-</font>2)&times;(<font face=symbol>-</font>4) = 8 .<br>

所以   8<font face=symbol>¸</font>(<font face=symbol>-</font>4) = <font face=symbol>-</font>2.  ①<br>

另一方面,我们有  <br>

      8&times; (<font face=symbol>-</font>(1)/(4))=<font face=symbol>-</font>2  ②<br>

于是有 <br>

     8 <font face=symbol>¸</font> (<font face=symbol>-</font>4) = 8&times; (<font face=symbol>-</font>(1)/(4))  ③<br>

③式表明,一个数除以-4可以转化为乘 <font face=symbol>-</font>(1)/(4)来进行,即一个数除以-4,等于乘于-4的倒数  <font face=symbol>-</font>(1)/(4).<br>

    与小学学过的除法一样,对于有理数除法,我们有如下法则;<br>

    除以一个不等于0的数,等于乘于这个数的倒数。<br>

    这个法则也可以表示成<br>

         a<font face=symbol>¸</font> b=a &#183; (1)/(b).(b!= 0) <br>

        从有理数除法法则,很容易得出:<br>

        两数相除,同号得正,异号得负,并把绝对值相除,0除以任何一个不等于0的数,都得0.<br>

\endarticle


\beginexercise
例5 计算:<br>

(1)  (<font face=symbol>-</font>36)<font face=symbol>¸</font> 9 . (2)  (<font face=symbol>-</font> (12)/(25) <font face=symbol>¸</font> (<font face=symbol>-</font> (3)/(5))) .<br>

解:(1)  (<font face=symbol>-</font>36)<font face=symbol>¸</font> 9 = <font face=symbol>-</font>(36 <font face=symbol>¸</font> 9) = <font face=symbol>-</font>4 .<br>


(2): <font face=symbol>-</font> (12)/(25) <font face=symbol>¸</font> (<font face=symbol>-</font> (3)/(5))) = (<font face=symbol>-</font> (12)/(25)) &times; (<font face=symbol>-</font> (5)/(3)) = (4)/(5) .<br>

\endexercise

\beginexercise
练习<br>


计算:<br>

(1) (<font face=symbol>-</font>18) <font face=symbol>¸</font> 6 ;
(2)(<font face=symbol>-</font>63) <font face=symbol>¸</font> (<font face=symbol>-</font>7);
(3)  1 <font face=symbol>¸</font> (<font face=symbol>-</font>9) ; <br>

(4) 0 <font face=symbol>¸</font> (<font face=symbol>-</font>8) ;
(5)(<font face=symbol>-</font>6.5) <font face=symbol>¸</font> 0.13;
(6)(<font face=symbol>-</font>(6)/(5)) <font face=symbol>¸</font> (<font face=symbol>-</font>(2)/(5)).
\endexercise

\beginexercise
例6 化简下列分数:<br>

(1)  (<font face=symbol>-</font>12)/(3) ; (2)  (<font face=symbol>-</font>45)/(<font face=symbol>-</font>12) .<br>

解:(1)  (<font face=symbol>-</font>12)/(3) = (<font face=symbol>-</font>12) <font face=symbol>¸</font> 3 = <font face=symbol>-</font>4 ;<br>


(2): (<font face=symbol>-</font>45)/(<font face=symbol>-</font>12) = (<font face=symbol>-</font>45) <font face=symbol>¸</font> (<font face=symbol>-</font>12) = 45 <font face=symbol>¸</font> 12 = (15)/(4) .<br>

\endexercise

\begintip
分数可以理解为分子除以分母<br>

\endtip

\beginarticle
因为有理数得除法可以化为乘法,所以可以利用乘法的运算性质简化运算.乘除混合运算往往先将除法化为乘法,然后确定积的符号,最后求出结果.<br>

\endarticle

\beginexercise
例7 计算:<br>

(1)  (<font face=symbol>-</font>125(5)/(7)) <font face=symbol>¸</font> (<font face=symbol>-</font>5) ; (2)  <font face=symbol>-</font>2.5 <font face=symbol>¸</font> (5)/(8) &times; (<font face=symbol>-</font>(1)/(4)) .<br>

解:(1)  (<font face=symbol>-</font>125(5)/(7)) <font face=symbol>¸</font> (<font face=symbol>-</font>5) ;<br>

 =(125+ (5)/(7)) &times; (1)/(5)  <br>

 =125 &times; (1)/(5) + (5)/(7) &times; (1)/(5) <br>

 =25 + (1)/(7) <br>

 =25(1)/(7) <br>


(2) <font face=symbol>-</font>2.5 <font face=symbol>¸</font> (5)/(8) &times; (<font face=symbol>-</font>(1)/(4)) .<br>

 (5)/(2) &times; (8)/(5) &times; (1)/(4) <br>

 =1
\endexercise


\enddocument
</body>
</html>
