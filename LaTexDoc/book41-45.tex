\documentclass{article}
\usepackage[utf8]{ctex}

\begin{document}


观察与猜想

翻牌游戏中的数学道理

桌上有9米正西向上的扑克牌,每次自动其中任意2张(包括已翻过的牌)。使它们从一面向上变为另一面向上,这样一直做下去,观察能否使所有的牌都反面向上?

你不妨动手试一试,看看会不会出现所有牌都反面向上。

事实上,不论你翻多少次,都不能使9张牌都反面向上,从这个结果,你能想到其中的数学道理吗?

如果在每张牌的正西都写1.反面都写-1.考虑所有牌朝上一面的数的积开始9张牌都正面向上,上面的数的积是1.每次翻动2来。就是说有2张牌同时改变符号,这能改变朝上一面的数的积是1这一结果吗? 9张牌都反面向上时,上面的数的积是什么数?这种现象为什么不能出现?

你能解释为什么不会使9张牌都反面向上了吗?如果桌上有任意奇数张牌,猜想结果会是怎样?



\section*{1.5有理数的乘方}
\subsection*{1.5.1 乘方}

前面学了有理数的乘法,下面研究各个乘数都相同时的乘法运算,

我们知道,边长为$2cm$的正方形的面积是$2\times2=4(cm^2)$; 棱长为$2cm$的正方体的体积是$2\times2\times2=8(cm^2)$.

$2\times2$, $2\times2\times2$都是相同因数的乘法.

为了简便,我们将它们分别记作$2^2$, $2^3$. $2^2$读作“2的平方”(或“2的二次方"),$2^3$读作“2的立方”(或“2的三次方").

同样:
$(-2)\times(-2)\times(-2)\times(-2)$记作$(-2)^4$,读作“-2的四次方";$(-\frac{2}{5})\times(-\frac{2}{5})\times(-\frac{2}{5})\times(-\frac{2}{5})\times(-\frac{2}{5})$记作$(-\frac{2}{5})^5$,读作“$-\frac{2}{5}$的五次方”。
一般地,n个相同的因数a相乘,即$\underbrace{a\bullet a\bullet\cdots\bullet a}_{\text{n}} $记作$a^n$",读作“a的n次方”。


\definition{乘方}
\begin{definition}
求n个相同因数的积的运算,叫做乘方,乘方  指数的结果叫做幂(power). 在$a^n$中,a叫做底数(base number), n叫做指数(exponent), 当$a^n$看作a的n次方的结果时,也可读作“a的n次幂"。
\end{definition}

例如,在$9^4$中,底数是9,指数是4, $9^4$读作“9的4次方",或“9的4次幂".

一个数可以看作这个数本身的一次方例如,5就是$5^1$.指数1通常省略不写。

因为$a^n$就是n个a相乘,所以可以利用有理数的乘法运算来进行有理数的乘方运算.

\begin{example}
\paragraph{例1}计算:
(1) $(-4)^3$;  (2) $(-2)^4$;  (3) $(-\frac{2}{3})^3$.
\paragraph{解:}
(1) $(-4)^3=(-4)\times(-4)\times(-4)=-64$;
(2) $(-2)^4=(-2)\times(-2)\times(-2)\times(-2)=16$;
(3) $(-\frac{2}{3})^3=(-\frac{2}{3})\times(-\frac{2}{3})\times(-\frac{2}{3})=-\frac{8}{27}$.
\end{example}

\begin{exercise}
\paragraph{思考}
从例1,你发现负数的暴的正负有什么规律?

当指数是___数时,负数的幂是___数;

当指数是___数时, 负数的幂是__数.
\end{exercise}

\property{乘方}
\begin{property}
根据有理数的乘法法则可以得出:

负数的奇次幂是负数,负数的偶次幕是正数

显然,正数的任何次幂都是正数,0的任何正整数次幂都是0.
\end{property}

\begin{example}
例2 用计算器计算$(-8)^5$和$(-3)^4$.

解:用带符号键"-"的计算器.

{((-)8)^5=}

显示:(-8)^5

-32768

{((-)3)^6=}

显示:(-3)^6

729.

所以$(-8)^5=- 32768$, $(-3)^4=729$.

\end{example}

\begin{exercise}
练习

1. (1) $(-7)^8$中,底数、指数各是什么?

(2) $(-10)^8$中-10叫做什么数? 8叫做什么数? $(-10)^4$是正数还是负数?

2.计算:

(1) $(-1)^10$;  (2) $(-1)^7$:  (3) $8^3$;  (4) $(-5)^3$;

(5) $0.1^3$;    (6) $(-\frac{1}{2})^6$; (7) $(-10)^4$; (8) $(-10)^5$;
\end{exercise}

做有理数的混合运算时,应注意以下运算顺序:

1.先乘方,再乘除,最后加减:

2.同级运算,从左到右进行

3.如有括号,先做括号内的运算,按小括号、中话号、大括号依次进行.

\begin{example}

例3 计算:

(1) $2\times(-3)^3-4\times(-3)+15$;

(2) $(-2)^3+(-3)\times[(-4)^2+2]-(-3)^2\div(-2)$.

解: (1)原式\begin{align}
        &=2\times(-27)-(-12)+15 \\
        &=-54+12+15     \\
        &=-27;
\end{align}

(2)原式\begin{align}
    &=-8+(-3)\times(16+2)-9+(-2) \\
    &=-8+(-3)\times18-(-4.5) \\
    &=-8-54+4.5 \\
    &=-57.5.
    \end{align}

例4 观察下面三行数:

-2, 4, -8, 16, -32, 64, \dots ①

0, 6, -6, 18, -30, 66, \dots  ②

-1, 2, -4, 8,-16, 32, \dots  ③

(1)第①行数按什么规律排列?

(2)第②③行数与第①行数分别有什么关系?

(3)取每行数的第10个数,计算这三个数的和

分析:观察①,发现各数均为2的倍数.联系数的乘方,从符号和绝对值两方面考虑,可发现持列的规律.

解: (1)第①行数是

-2, $(-2)^2$, $(-2)^3$, $(-2)^4$, \dots.

(2)对比①②两行中位置对应的数,可以发现:

第②行数是第①行相应的数加2,即

-2+2, $(-2)^2+2$,  $(-2)^2+2$, $(-2)^4+2$, \dots;

对比①③两行中位置对应的数,可以发现:

第③行数是第①行相应的数的0.5倍,即

$-2\times0.5$, $(-2)^2\times0.5$, $(-2)^3\times0.5$, $(-2)^4\times0.5$, \dots.

(3)每行数中的第10个数的和是

\begin{align}
    &(-2)^{10}+[(-2)^{10}+2]+(-2)^{10}\times0.5 \\
    &=1024+(1024+2)+1024\times0.5 \\
    &=1024+1026+512 \\
    &=2562.
\end{align}

\begin{exercise}
    计算:

    (1) $(-1)^{20}\times2+(-2)^3\div4$;

    (2) $ (-5)^3-3\times(-\frac{1}{2})^4$;

    (3) $ \frac{11}{5}\times(\frac{1}{3}-\frac{1}{2})\times\frac{3}{11}\div\frac{5}{4}$;
    
    (4) $ (-10)^4+[(-4)^2-(3+3^2)\times2] $.
\end{exercise}

\subsection*{1.5.2 科学记数法}

现实中,我们会遇到一些比较大的数。 例如,太阳的半径、光的速度、目前世界人口等。读、写这样大的数有一定困难。
观察10的乘方有如下特点:

{$10^2=100$, $ 10^3=1000 $, $ 10^4=10000 $, \cdots.}

一般地,10的n次幂等于10\dots 0(在1的后面有n个0),所以可以利用10的乘方表示一些大数,例如

$567000000=5.67\times100000000=5.67\times10^{8}$

读作“5.67乘10的8次方(幂)”。

这样不仅可以使书写简短,同时还便于读数。

	\definition{科学计数法}
\begin{definition}

像上面这样,把一个大于10的数表示成$a\times10^{n}$的形式(其中a大于或等于1且小于10. n是正整数),使用的是科学记数法.

\end{definition}   

对于小于-10的数也可以类似表示,例如:

$ -567000000=-5.67\times10^{8} $.

\begin{example}

例5 用科学记数法表示下列各数:

1000000, 57000000, -123000000000.

解:

$1000000=10^{6}$,

$ 57000000=5.7\times10^{7} $,

$ -123000000000=-1.23\times10^{11} $.

\end{example}

\begin{exercise}
思考:

上面的式子中,等号左边整数的位数与右边10的指数有什么关系?用科学记数法表示一个n位整数,其中10的指数是__.
\end{exercise}

\begin{exercise}
1.用科学记数法写出下列各数:

10000, 800000, 56 000 000, -7400 000.

2.下列用科学记数法写出的数,原来分别是什么数?

$1\times10^{7}$, $4\times10^{9}$, $8.5\times10^{6}$, $7.04\times10^{5}$, $-3.96\times10^{4}$.

3.中国的陆地面积约为$9600000km^{2}$,领水面积约为$370000km^{2}$,用科学记数法表示上述两个数字.

\end{exercise}

\subsection*{1.5.3 近似数}

先看一个例子,对于参加同一个会议的人数,有两个报道。一个报道说:“会议秘书处宣布,参加今天会议的有513人”这里数字513确切地反映了实际人数,它是一个准确数。另一报道说: 

\end{document}