\documentclass{article}
\usepackage[utf8]{ctex}

\begin{document}
\begin{article}

\begin{exercise}

练习\\1.解下列方程:

\indent(1)$6x-7=4x-5$;

\indent (2)$\frac{1}{2}x-6=\frac{3}{4}x$.\\2.王芳和李丽同时采摘樱桃,王芳平均每小时采摘8kg,李丽平均每小时采摘7kg.采摘结束后王芳从她采摘的樱桃中取出0.25kg给了李丽,这时两人的樱桃一样多。她们采摘用了多少时间?
\end{exercise}

\begin{exercise}

习题3.2

复习巩固

1.解下列方程:

      (1) $2x+3x+4x=18$;\indent (2) $13x-15x+x=-3$;
\\
      (3) 2.5y+10y-6y=15-21.5;\indent (4)$\frac{1}{2}b-\frac{2}{3}b+b=\frac{2}{3}\times6-1$.\\2.举例说明解方程时怎样“移项”,你知道这样做的根据吗?\\3.解下列方程:

      (1) x+3x=- 16;\indent (2) 16y- -2.5y-7.5y= 5;\indent(3) 3x+5=4x+1;\indent(4) 9-3y=5y+5.\\4.用方程解答下列问题:

      (1) x的5倍与2的和等于工的3倍与4的差,求x;\\(2) y与-5的积等于y与5的和,求y.

5.小新出生时父亲28岁,现在父亲的年龄是小新年龄的3倍,求现在小新的年龄,6.洗衣机厂今年计划生产洗衣机25500台,其中工型、型、型三种洗衣机的  量比为1:2:14,计划生产这三种洗衣机各多少台?

7.用一根长60m的绳子围出一个长方形,使它的长是宽的1.5倍,长和宽各应是多少?

综合运用

8.随着农业技术的现代化,节水型灌溉得到连步推广。
噴灌和滴灌是比漫灌节水的灌溉方式,灌溉三块同样大的实验团,第一块用漫灌方式,第二块用喷灌方式,第三块用滴灌方式,后两种方式用水量分别是漫灌的25\%和15\%.

(1)设第一块实验四用水工t,则另两块实验田的用水量各如何表示?\\(2)如果三块实验田共用水420t,每块实验田各用水多少晚?

9.某造纸厂为节约木材,大力扩大再生纸的生产。它去年10月生产再生纸2 050 t,这比它前年10月再生纸产量的2倍还多150L它前年10月生产再生纸多少晚?

10.把一根长100 cm的水棍锯成两段,要使其中一段长比另一段长的2倍少5 cm,应该在木棍的哪个位置锯开?

11.几个人共同种一批树苗,如果每人种10棵,则刺下6棵树苗未种:如果每人种12颗,则缺6裸树苗。求参与种树的人数.
\end{exercise}

拓广探索

12. 在一张普通的月历中,相邻三行里同一列的三个日期数之和能否为30?如果能,这三个数分別是多少?

13.一个两位数的个位上的数的3倍加1是十位上的数。个位上的数与十住上的数的
和等于9,迄个両位数是多少?

突驗与探究
无隈循珎小数化分数
我伯如道分敷言写カ小敵形式叩a3,反辻来,无灰希坏小数a謁カ分数形式叩言·一殼地,任何一不无限循珎小数都可以写カ分数形式喝?如栗可以,虚怱祥写兜?
先以无限循坏小薮o.7カ例迸行対絶.
没0.7=x。由o.77--可知。107-77--所以l1o--x=7. 解方程,得x= ;子チ是。得ai-子
想一想t如何把像ai, a2. - a9迄祥的无艮循坏小娠化カ分数形式?効手斌一拭。再以无限循坏小数の.3カ側。傲迸一歩的け池.
无限循珎小数0.730.737 373-它的循坏梦有再位,奥比上涵的け沿可以想到如下的傲法.
筱0.73=x.由0.30.737 373--可知.100x-73.737 3--所以100r x=73.解方程,得z=3.于是,得oa.3=る、
愡一悠1如何把像o.i0. o.iz, - o98迭祥的无限希芥小数化カ分数形式?朸手武一武。
想一想:如何把无限循坏小数o.735. o.823i化カ分藪形式?劫手武一武,并思拮把无隈循珎小数化カ分数形式的一敖規律.

\subsection*{3.3 解一元一次方程(二)\\\indent \indent \indent -------
去括号与去分母}

当方程的形式较复杂时,解方程的步骤也相应更多些,本节重点讨论如何利用“去括号”和“去分母”解-元一次方程.
\begin{example}

问题1某工厂 加强节能措施,去年下半年与上半年相比,月平均用电量减少2000kW.h (千瓦●时),全年用电15万kW.h.这个工厂去年上半年每月平均用电是多少?

设上半年每月平均用电x kW. h,则下半年每月平均用电(x- 2 0kW.h;上半年共用电6x kW. h,下半年共用电6(x--2 000) kW. h.
\\根据全年用电15万kw.h,列得方程
6x+6(x - 2000)= 150 000.\\如果去括号,就能简化方程的形式下面的框图表示了解这个方程的流程.\\
6x+6(x-2 000)= 150000\\
↓去括号\\
6x +6x- 12 000= 150 000\\
↓移项\\
6x+6x= 150000+12000\\
↓合并同类项\\12x=162 000\\
↓系数化为1\\x=13500\\

1 kW. h的电量即1 kW的电器1 h的用电量

由上可知,这个工厂去年上半年每月平均用电13 500 kW.h.

思考

本题还有其他列方程的方法吗?用其他方法列出的方程应怎样解?

      方程中有带括号的式子时,去括号是常用的化简步骤.

      例1解下列方程:

      (1) $2x-(x+10)=5x+2(x- 1)$;\\(2) $3x- 7(.x- 1)=3- 2(x+3)$.\\解: (1) 去括号,得

      $2x-x-10=5x+2x-2$.\\移项,得

      $2x-x-5x-2x=-2+10$.\\合并同类项,得

      $-6x=8$.\\系数化为1.得

      $x=-\frac{4}{3}$\\(2)去括号,得

      $3x-7x+7=3-2x-6$.\\移项,得

      $3x-7x+2x=3-6-7$.\\合并同类项,得

      $-2x=-10$.\\系数化为1,得

      $x=5$.

      例2一艘船从甲码头到乙码头顺流而行, 用了2h;从乙码头返回甲码头逆流面行,用了2.5h.已知水流的速度是3 km/h,求船在静水中的平均速度.

      分析:一般情况下可以认为这艘船往返的路程相等,由此填空:

      顺流速度\underline{\hbox to 20mm}顺流时间\underline{\hbox to 20mm} 逆流速度\underline{\hbox to 20mm}逆流时间,

      解:设船在静水中的平均速度为x km/h,则顺流速度为(x+3)km/h,逆流速度为(x - 3)km/h.

      根据往返路程相等,列得

      $2(x+3)=2.5(x- 3)$.
      
去括号,得\\
$2x+6=2.5x-7.5$.
移项及合并同类项,得\\
$0.5x=13.5$.\\
系数化为1,得\\
$x=27$.\\
答:船在静水中的平均速度为27 km/h.\\
\end{example}


\begin{exercise}
练习
解下列方程:\\
(1) $2(x+3)=5x$;\indent\indent(2) $4x+3(2x- 3)-12- (x+4)$;\\
(3)$6\left( \frac{1}{2}x-4 \right)+2x=7-\left( \frac{1}{3}-1 \right)$;\indent\indent(4) $2- 3(x+1)=1-2(1+0.5x)$.
\end{exercise}

英国伦敦博物馆保存着-部极其珍贵的文物纸草书这是古代埃及人用象形文字写在一种用纸莎草压制成的草片上的著作,它于公元前1700年左右写成这部书中记载了许多有关数学的问题,下面的问题2就是书中-道著名的求未知数的问题

\begin{example}
问题2一个数,它的三分之二。它的半,它的七分之-, 它的全部,加起来总共是33.

这个问题可以用现在的数学符号表示设这个数是x.根据题意得方程

$\frac{2}{3}x+\frac{1}{2}x+\frac{1}{7}x+x=33$.

当时的埃及人如果采用了这种形式,它定是“最早”的方程。
这样的方程中有些系数是分数,如果能化去分母,把系数化成整数,则可以使解方程中的计算更简便些.

我们知道,等式两边乘同个数,结果仍相等.这个方程中各分母的最小公倍数是42,方程两边乘42,得

$42\times\frac{2}{3}x+42\times\frac{1}{2}+42\times\frac{1}{7}+42x=42\times33$

\end{example}

\end{article}
\end{document}
