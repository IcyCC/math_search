\documentclass{ctexart}
\title{56-60}
\begin{document}
\maketitle
\begin{article}
\begin{ex}
1.某种商品每袋4.8元,在一个月内的销售量是m 袋,用式子表示在这个月内销售这种商品的收入。\\
2.圆柱体的底面半径,高分别是r,h,用式子表示圆柱体的体积。\\
3.有两片棉田,一片有$hm^{2}$(公顷,$1hm^{2}=10^{4}m^{2}$),平均每公顷产棉花akg,另一片有n$hm^{2}$,平均每公顷产棉花bkg,用式子表示两片棉花田上棉花的总产量。\\
4.在一个大正方形铁片中挖去一个小正方形铁片,大正方形的边长是amm,小正方形的边长是bmm,用式子表示剩余部分的面积。\\
\end{ex}
思考\\
我们来看引言与例一中的式子\\
100t,0.8p,mn,$a^{2}h$,-n,\\
这些式子有什么特点?\\

	\definition{单项式}
\begin{definition}
这些式子都是数或字母的积,像这样的式子叫做单项式(monomial),单独的一个数或一个字母也是单项式。\\
单项式中的数字因数叫做这个单项式的系数(coeffcient).例如,单项式100t,$a^{2}h$,-n的系数分别是100,1,-1,单项式表示数与字母相乘时,通常把数写在前面。\\
一个单项式中,所有字母的指数的和叫做这个单项式的次数(degree of a monomial).例如,在单项式100t中,字母t的指数是1,100t的次数是1;在单项式$a^{2}h$中,字母a与h 的指数的和是3,$a^{2}h$的次数是3.\\
\end{definition}
\begin{example}
例3 用单项式填空,并指出他们的系数和次数:\\
(1)每包书有12册,n包书有\_册;\\
(2)底边长为acm,高为hcm的三角形的面积是\_$cm^{2}$;\\
(3)棱长为acm的正方体的体积是\_$cm^{2}$;\\
(4)一台电视机原价b元,现按原价的9折出售,这台电视机现在的售价是\_元;\\
\end{example}
\end{article}

\begin{article}

\begin{example}
(5)一个长方形的长是0.9m,宽是bm,这个长方形的面积是_$m^{2}$.\\
解:(1)12n,他的系数是12,次数是1;\\
(2)$$1/2ah$$,他的系数是$$1/2$$,次数是2;\\
(3)$a^{3}$,他的系数是1,次数是3;\\
(4)0.9b,他的系数是0.9,次数是1;\\
(5)0.9b,他的系数是0.9,次数是1;\\

\end{example}
用字母表示数后,同一个式子可以表示不同的含义。例如,在例三的第(4),(5)小题中,0.9b既可以表示电视机的售价,又可以表示长方形的面积,当然它还可以表示更多的含义,你能赋予0.9b一个含义吗?\\
\begin{ex}
1填表\\
\begin{tabular}{|c|c|c|c|c|c|}
\hline  % 在表格最上方绘制横线
单项式&$2a^{2}$&-1.2h& $xy^{2}$& $-t^{2}$&$\frac{-2vt}{3}$\\
\hline
系数&&&&&\\
\hline
次数&&&&&\\
\hline
  \end{tabular}
2填空:\\
(1)全校学生总数是x,其中女生占总数的48$\%$,则女生人数是\_,男生人数是\_;\\
(2)一辆长途汽车从杨柳村出发,3h后到达距出发地skm的溪河镇,这辆长途汽车的平均速度是\_km/h;\\
(3)产量由mkg增长10$\%$,就达到\_kg.\\

\end{ex}
思考\\
我们来看例二中的式子\\
v+2.5,v-2.5,3x+5y+2z,$\frac{1}{2}ab$-$\pir^{2}$,$x^{2}$+2x+18,\\
这些式子有什么特点?\\
\end{article}

\begin{article}
\usepackage{graphicx}
这些式子都可以看做几个单项式的和。例如,v-2.5可以看作是v与-2.5的和;$x^{2}$+2x+18 可以看作单项式$x^{2}$,2x与18的和。\\

	\definition{多项式}
\begin{definition}
像这样,几个单项式的和叫做多项式(polynomial).其中,每个单项式叫做多项式的项(term),不含字母的项叫作常数项(constant term).例如,多项式v-2.5的项是v与-2.5,其中-2.5是常数项;多项式$x^{2}$+2x+18的项是$x^{2}$,2x和18,其中18是常数项。\\
多项式里,次数最高的项数,叫做这个多项式的次数(degree of a polynomial).例如,多项式v-2.5中次数最高项是一次项v,这个多项式的次数是1;多项式$x^{2}$+2x+18中次数最高项是二次项$x^{2}$,这个多项式的次数是2.\\
单项式与多项式统称整式(intergral expression).例如,上面见到的单项式100t,0.8p,mn,$a^{2}h$,-n,以及多项式v+2.5,v-2.5,3x+5y+2z,$\frac{1}{2}ab$-$\pir^{2}$,$x^{2}$+2x+18等都是整式。
\end{definition}
\begin{example}
例4 如图2.1-3,用式子表示圆环的面积,当R=15cm,r=10cm时,求圆环的面积(\pi取3.14).\\
\includegraphics{1.png}
解:外圆的面积减去内圆的面积就是圆环的面积,所以圆环的面积是$\piR^{2}-\pir^{2}$.\\
当R=15cm,r=10cm时,圆环的面积(单位:$cm^{2})是$\piR^{2}-\pir^{2}$=3.14\times$15^{2}-3.14\times$10^{2}$=392.5.\\
这个圆环的面积是392.5$cm^{2}$.\\
\end{example}
\begin{ex}
1填空:
(1)a,b分别表示长方形的长和宽,则长方形的周长l=\_,面积S=\_,当a=2cm,b=3cm时,l=\_cm,S=\_$cm^{2}$;\\
(2)a,b分别表示梯形的上底和下底,h表示梯形的高,则梯形的面积S=\_,当a=2cm,b=4cm,h=5cm时,S=\_$cm^2$.

\end{ex}
\end{article}
\begin{article}
\usepackage{graphicx}
\begin{ex}
2.用整形填空,指出单项式的次数以及多项式的次数和项:\\
(1)每袋大米5kg,x袋大米()kg;
(2)如图(图中长度单位:m),阴影部分的面积时()m^2;
\includegraphics{2.png}\\
(3)体重由xkg增加2kg后时()kg.\\
\end{ex}
\begin{ex}
复习巩固\\
1列示表示:
(1)棱长为acm的正方体的表面积。\\
(2)每件a元的上衣,降价20$\%$后的售价是多少元?\\
(3)一辆汽车的行驶速度时vkm/h,th行驶多少千米?\\
(4)长方形绿地的长,宽分别时am,bm,如果长增加xm,新增的绿地面积时多少平方米?\\
2列式表示\\
(1)温度由t$^{\circ}$C上升5$^{\circ}$C后是多少?\\
(2)两车同时,同地,同向出发,快车行驶速度时xkm/h,慢车行驶速度时ykm/h,3h后辆车相距多少千米?\\
(3)某种苹果的售价时每千克x元(x<10),用50 元买5kg这种苹果,应找回多少钱?\\
(4)如图(途中长度单位:cm),钢管的体积是多少?
\includegraphics{3.png}\\
3填表:
\begin{tabular}{|c|c|c|c|c|c|}
\hline  % 在表格最上方绘制横线
整式&-15ab&$4a^{2}b^{2}& $\frac{$3x^{2}y}{5}$& $4x^{2}-3$&$a^{4}-2a^{2}b^{2}+b^{4}$\\
\hline
系数&&&&&\\
\hline
次数&&&&&\\
\hline
项&&&&&\\
\hline
  \end{tabular}
综合运用
4.测得一种树苗的高度与树苗生长的年数的有关数据如下页表(树苗原高100cm);
\end{ex}
\end{article}
\begin{article}
\usepackage{graphicx}
\begin{ex}
\begin{tabular}{|c|c|}
\hline  % 在表格最上方绘制横线
年数&高度/cm\\
\hline
1&100+5\\
\hline
2&100+10\\
\hline
3&100+15\\
\hline
4&100+20\\
\hline
.....&...\\
\hline
  \end{tabular}
前四年树苗高度的变化与年数有什么关系?假设以后各年树苗高度的变化与年数保持上述关系,用式子表示生长了n 年的树苗的高度.\\
5.礼堂第一排有a个座位,后面每排都比前一排多一个座位,前2排有多少个座位?第三排呢?用式子表示第n排的座位数。如果第一排有20个座位,计算第19排的座位数。\\
6.一块三角尺的形状和尺寸如图所示。如果圆孔的半径是r,三角尺的厚度是h,用式子表示折块三角尺的体积V。若a=6cm,r=0.5cm,h=0.2cm,求V的值。(\pi 取3)。
\includegraphics{4.png}\\
拓广探索\\
7.设n表示任意一个整数,用含n的式子表示\\
(1)任意一个偶数;(2)任意一个奇数\\
8.3个球队进行单循环比赛(参加比赛的每一队都与其他所有的队各赛一场),总的比赛场数是多少?4个队呢?5个队呢?n个队呢?\\
9.对于密码L dp vwxghaw,你能看出它代表什么意思吗?如果给你一把破译它的钥匙x-3,联想英语字母表中字母的顺序,你再试试能不能解读他,英语字母表中字母是按一下顺序排列的:\\
abcdefghiklmnopqrstuvwxyz\\
如果规定a又接在z的后面,使26个字母排成圈,并能联想到x-3可以代表“把一个字母换成字母表中从他向前移动3
位的字母”,按这个规律就有\\
L dp vwxghaw \rightarrow I am a student\\
这样你就能解读它的意思了。
为了保密,许多情况下都要采用密码,这时候就需要有破译密码的“钥匙".上面的例子中,如果写和读密码的双方事先约定了作为”钥匙“的式子x-3的含义,那么他们就可以用一种保密方式通信了,你和同伴不妨也利用数学式子来制定一种类似的”钥匙",并互相合作,通过游戏试试如何保密通信。
\includegraphics{5.png}\\
\end{ex}
\end{article}
\end{document}
