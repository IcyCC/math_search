\documentclass{article}
\usepackage{xeCJK}
\usepackage{tikz}
\usepackage{indentfirst}
\setlength{\parindent}{2em}
\usepackage{enumitem}
\usepackage{latexsym}
\usepackage{graphicx}
\usepackage{indentfirst}
\usepackage{amsmath}
\newtheorem{exercise}{ }
\newtheorem{text}{ }
\newtheorem{propertory}{ }
\newtheorem{example}{ }
\newtheorem{concept}{ }
	
\begin{document}

	\section*{1.2 有理数}
	\subsection*{1.2.1 有理数}
	思考\\
	\indent 回想一下,我们认识了那些数?\\
	\indent 我们学过的数有:\\
	\indent 正整数,如$1,2,3,...;$\\
	\indent 零,$0$;\\
	\indent 负整数,如$-1,-2,-3,...;$\\
	\indent 正分数,如$\frac{1}{2},\frac{2}{3},\frac{15}{7},0.1,5.32,...;$\\
	\indent 负分数,如 $-0.5,-\frac{5}{2},-\frac{2}{3},-\frac{1}{7},-150.25,...;$\\
	\indent 正整数、0、负整数统称为正整数;正分数、负分数统称为分数.\\
	\definition{有理数}
		\begin{definition}
			整数和分数统称为有理数(rational number).\\
		\end{definition}

	\indent 从小学开始,我们首先认识了正整数和负分数后,对数的认识就扩充到了有理数范围.\\
	
	\exercise{练习}
		\begin{exercise}
			1.所有正整数组成正数集合,所有负数组成负数集合,把下面的有理数填入它属于的集合的范围内:\\
			\indent 15,-$\frac{1}{9}$,-5,$\frac{2}{15}$,-$\frac{13}{8}$,0.1,-5.32,-80,123,2.333\\
			
			\begin{tikzpicture}
			\draw (0,0) ellipse (1cm and 0.5cm);
			\draw (8,0) ellipse (1cm and 0.5cm);
			\end{tikzpicture}
			
			2.指出下列各数中的正数、负数、整数、分数:\\
			\indent -15,+6,-2,-0.9,1,$\frac{3}{5}$,0,$\frac{1}{4}$,0.63,-4.95.\\
		\end{exercise}
	
	\subsection*{1.2.2 数轴}
	问题 在一条东西向的马路上,有一个汽车站牌,汽车站东3m和7.5m处分别有一棵柳树和一棵杨树,汽车站牌西3m和4.8m处分别有一颗槐树和一根电线杆,试画图表示这一情景.\\
	\indent 如图1.2-1,画一条直线表示马路,从左到右表示从西到东的方向,在直线上任取一点O表示汽车站牌的位置,规定1个单位长度(线段OA的长)代表1m长,于是,在点O右边,与点O距离3个和7.5个单位长度的点B和点C,分别表示柳树和杨树的位置:点O左边,与点O距离3个和4.8个单位长度的点D和点E,分别表示槐树和电线杆的位置.\\
	\begin{figure}[ht!]
		\centering
		\includegraphics[width=1\textwidth,natwidth=400,natheight=63]{./1.2-1.PNG}
		\caption{1.2-1}
	\end{figure}

	思考\\
	\indent 怎样用数简明地表示这些树,"东"与"西"、"左"与"右"都具有相反 意义.如图1.2-2,在一条直线上取一个点O为基准点,用0表示它,在用负数表示O左边的点,用正数表示点O右边的点.这样,我们就用负数、0、正数表示出了这条直线上的点.\\
	\begin{figure}[ht!]
		\centering
		\includegraphics[width=1\textwidth,natwidth=400,natheight=60]{./1.2-2.PNG}
		\caption{1.2-2}
	\end{figure}
	\indent 用上述方法,我们就可以把这些树电线杆与汽车站牌的相对位置关系表示出来了.例如,-4.8表示位于汽车站牌西侧4.8m处的电线杆,等等.\\
	
	思考\\
	\indent 图1.2-3中的温度计可以看作表示正数、0和负数的直线.它和图1.2.2有什么共同点,有什么不同点?
	\begin{figure}[ht!]
		\centering
		\includegraphics[width=1\textwidth,natwidth=600,natheight=140]{./1.2-3.PNG}
		\caption{1.2-3}
	\end{figure}

	\definition{数轴}
	\begin{defintion}
		在数学中,可以用一条直线上的点表示数,则条直线叫做数轴(number axis).它满足以下要求:\\
	\end{defintion}
	
	\property{数轴}
	\begin{propertory}
		\indent (1)在一条直线上任取一个点表示数0,这个点叫做原点(origin);\\
		\indent (2)通常规定直线上从原点向右(或上)为正方向,从原点向左(向下)为负方向;\\
		\indent (3)选取适当的长度为单位长度,直线上从原点向右,每隔一个长度单位取一个点,一次表示1,2,3,...;从原点向左,用类似方法一次表示-1,-2,-3,...(图1.2-4).\\
	\end{propertory}
	
	\begin{figure}[ht!]
		\centering
		\includegraphics[width=1\textwidth,natwidth=400,natheight=40]{./1.2-4.PNG}
		\caption{1.2-4}
	\end{figure}
	\indent 分数或小数也可以用数轴上的点表示,例如从原点向右6.5个单位长度的点表示小数6.5,从原点向左$\frac{3}{2}$个单位长度的点表示分数$-\frac{3}{2}$(图1.2-4).\\
	
	归纳\\
	\indent 一般的,设a是一个正数,则数轴上表示数a的点再原点的\underline{\hbox to 10mm{}}边,与原点的距离是\underline{\hbox to 10mm{}}个单位长度;表示-a的点在原点的\underline{\hbox to 10mm{}}边,与原点的距离是\underline{\hbox to 10mm{}}个单位长度.\\
	\indent 用数轴上的点表示对数学的发展起到了重要作用,以它做基础,可以借助图直观地表示很多与数相关的问题.\\
	
	\exercise{练习}
	\begin{exercise}
		1.如图,写出数轴上点A,B,C,D,E表示的数.\\
		\begin{figure}[ht!]
			\centering
			\includegraphics[width=1\textwidth,natwidth=400,natheight=40]{./exe1.1.PNG}
			\caption{1}
		\end{figure}
		2.画出数轴并表示下列有理数:\\
		\indent
		3.数轴上,如果表示a的点在原点的左边,那么a是一个\underline{\hbox to 10mm{}}数;如果表示数b的点在原点的右边,那么b是一个\underline{\hbox to 10mm{}}数.\\
	\end{exercise}

	\subsection*{1.2.3有理数}
	探究\\
	\indent 在数轴上,与原点的距离是2的点有几个?这些点表示哪个数?\\
	设a是一个正数,数轴上与原点的距离等于a的点有几个?这些点表示的数有什么关系?\\
	\indent 可以发现,数轴上与原点的距离是2的点有两个,他们表示的数是-2和2.\\
	
	归纳\\
	\indent 一般地,设a是一个正数,数轴上与原点的距离是a的点有两个,它们分别在原点左右,表示-a和a(图1.2-5),我们说这两点关于原点对称.\\
	\begin{figure}[ht!]
		\centering
		\includegraphics[width=1\textwidth,natwidth=400,natheight=40]{./1.2-5.PNG}
		\caption{1.2-5}
	\end{figure}
	\indent 像2和-2,5和-5这样,只有符号不同的两个数叫做互为相反数(oposite number).这就是说,2的相反数是-2,-2的相反数是2;5的相反数是-5,-5的相反数是5.\\
	\indent 一般地,a和-a互为相反数.特别地,0的相反数是0.这里,a表示任意一个数,可以是正数、负数,也可以是0.例如:\\
	\indent 当a=1时,-a=-1,1的相反数是-1;同时,-1的相反数是1.\\
	思考\\
	\indent 设a表示一个数,-a一定是负数吗?\\
	\indent 容易看出,在正数前面添加"-"号,就得到这个正数的相反数.在任意一个数前面添上"-"号,新的数表示原数的相反数.例如.\\
	\indent $-(+5)=-5, -(-5)=+5, -0=0.$\\
	\exercise{练习}
	\begin{exercise}
		1.判断下列说法是否正确:\\	
		\newcommand{\fourch}[4]{
			\indent\makebox[262pt][l]{\qquad(A) #1}\\
			\indent\makebox[262pt][l]{\qquad(B) #2}\\
			\indent\makebox[262pt][l]{\qquad(C) #3}\\
			\indent\makebox[262pt][l]{\qquad(D) #4}\\}
		2.写出下列各数的相反数:\\
		\indent 6,-8,-3.9,$\frac{5}{2}$,$-\frac{2}{11}$,100,0.\\
		3.如果a=-a,那么表示a的点在数轴上的什么位置?\\
		4.化简下列各数:\\
		\indent -(-68),-(+0.75),-(-\frac{3}{5}),-(+3.8).\\
	\end{exercise}	

\end{document}