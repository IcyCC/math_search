\documentclass[11pt,apaper]{article}
\usepackage{framed} 
\usepackage{diagbox}
\usepackage{amssymb}
\usepackage{graphicx}
\usepackage{CJKutf8}
\usepackage{ctex}
\begin{document}


\begin{framed}
练习\\

1.某商店有两种书包,每个小书包比大书包的进价少10元,而它们的售后利润额相同,其中,每个小书包的盈利率为30$\%$,每个大书包的盈利率为20$\%$, 试求两种书包的进价。\\
2.用A4纸在某誊印社复印文件,复印页数不超过20时,每页收费0.12元;复印页数超过20时,超过部分每页收降为0.09元在某图书馆复印同样的文件,不论复印多少页,每页收费0.1元复印张数为多少时,两处的收费相同?\\
3.下表是某校七至九年级某月课外兴趣小组活动时间统计表,其中各年级同一兴趣小组每次活动时间相同.\\
\begin{center}
  \includegraphics[scale=0.6]{1.JPG}\\
\end{center}


请将九年级课外兴趣小组活动次数填入上表
\end{framed}

\begin{framed}
复习巩固\\

1.结合本节内容体会例2后归纳的框图\\
2.制作一张桌子要用一个桌面和4条桌腿,1m3木材可制作20个桌面,或者制作400条桌腿,现有12m木材,应怎样计划用料才能制作尽可能多的桌子?\\
3.某车间每天能制作甲种零件500只,或者制作乙种零件250只,甲、乙两种零件各一只配成一产品,现要在30天内制作最多的成套产品,则甲、乙两种零件各应制作多少天?\\
4.某中学的学生自己动手整修操场,如采让七年级学生单独工作,需要7.5h 完成,如果让八年级学生单独工作,需要5h完成,如果让七、八年级学生一起工作1h,再由八年级学生单独完成剩余部分,共需多少时间完成?\\
5.整理一批数据,由一人做需80h完成,现在计划先由一些人做2h,再增加5人做8h,完成这项工作的$\frac{3}{4}$,怎样安排参与整理数据的具体人数?\\
\end{framed}

\begin{framed}
综合运用\\

6.(古代问题)某人工作一年的报是年终给他一件衣服和10枚银币,但他干满7个月就决定不再继续干了,结账时,给了他一件衣服和2枚银币。这件衣服值多少枚银币?\\
7.用A型和B型机器生产样的产品,已知5台A型机器一天的产品装满8箱后还剩4个,7台B型机器一天的产品装满11箱后还制1个,每台A型机器比B型机器一天多生产1个产品,求每箱装多少个产品\\
8.下表中记录了一次试验中时间和温度的数据\\
\begin{center}
  \includegraphics[scale=0.6]{2.JPG}\\
\end{center}

(1)如果温度的变化是均匀的,21min时的温度是多少?\\
(2)什么时间的温度是34℃?\\
9.某糕点厂中秋节前要制作一批盒装月饼,每盒中装2块大月饼和4块小月饼,制作1块大月饼要用0.05kg面粉,1块小月饼要用0.02kg面粉.现共有面粉4500kg,制作两种月饼应各用多少面粉,才能生产最多的盒装月饼?
10.小刚和小强从A,B两地同时出发,小刚骑自行车,小强步行,沿同一条路线相向匀遠而行,出发后2h两人相遇,相遇时小刚比小强多行进24km,相遇后0.5h小刚到达B地两人的行进速度分别是多少?相遇后经过多少时间小强到达A地?\\

\end{framed}




\begin{framed}
拓广探索\\

11.现对某商品降价20\%促铕,为了使售总金额不变,销售量要比按原价销售时增加百分之几?\\
12.甲组的4名工人3月份完成的总工作量比此月人均定额的4倍多20件,乙组的5名工人3月份完成的总工作量比此月人均定额的6倍少20件\\
(1)如果两组工人实际完成的此月人均工作量相等,那么此月人均定额是多少件?\\
(2)如果甲组工人实际完成的此月人均工作量比乙组的多2件,那么此月人均定额是多少件?\\
(3)如果甲组工人实际完成的此月人均工作量比乙组的少2件,那么此月人均定额是多少件?\\
(古代问题)希腊数学家丢番图(公元3~4世纪)的墓碑上记载着:“他生命的六分之一是幸福的业年;再活了他生命的十二分之一,两颊长起了细细的胡须;他结了婚,又度过了一生的七分之再过五年,他有了儿子,感到很幸福可是儿子只活了他父亲全部年龄的一半;儿子死后,他在极度悲病中度过了四年,也与世长辞了\\
根据以上信息,请你算出:\\
(1)丢番图的寿命;\\
(2)丢番图开始当爸爸时的年龄;\\
(3)儿子死时丢番图的年龄\\

\end{framed}


\begin{framed}
活动1\\

统计资料表明,山水市去年居民的人均收入为11664元,与前年相比增长8\%, 扣除价格上涨因素,实际增长6.5\%你理解资料中有关数据的含义吗?如果不明白,请通过查阅资料或请教他人弄懂它们根据上面的数据,你能用一元一次方程解决下面的问题吗?\\
(1)山水市前年居民的人均收入为多少元?\\
(2)在山水市,去年售价为1000元的商品在前年的售价为多少元?\\
再收集一些统计数据,想一想利用这些数据之间的关系能否从中再计算出一些新数据\\
\end{framed}

\begin{framed}
活动2\\

用一根质地均匀的木杆和一些等重的小物体,做下列实验:\\
(1)在木杆中间处拴绳,将木杆吊起并使其左右平衡,吊绳处为木杆的支点;\\
(2)在木杆两端各悬挂一重物,看看左右是否保持平街;\\
(3)在木杆左端小物体下加挂一重物,然后把这两个重物一起向右移动,直至左右平街,记录此时支点到木杆左右两边挂重物处的距离;\\
(4)在木杆左端两小物体下再加挂一重物,然后把这三个重物一起向右移动,直至左右平街,记录此时支点到木杆左右两边挂重物处的距离\\
(5)在木杆左边继续加挂重物,并重复以上操作和记录\\
根据记录你能发现什么规律?\\
如图,在木杆右端挂一重物,支点左边挂n个重物,并使左右平衡,设木杆长为1cm,支点在木杆中点处,支点到木杆左边挂重物处的距离为xcm,把n,l作为已知数,列出关于x的一元一次方程\\
\begin{center}
  \includegraphics[scale=0.6]{3.JPG}\\
\end{center}
\end{framed}

\begin{framed}
小结\\
一、本章知识结构图\\
\begin{center}
  \includegraphics[scale=0.6]{4.JPG}\\
\end{center}
二、回顾与思考\\
方程是一种重要的描述现实世界的数学模型.本章中,通过一些实际问题,我们学习了最基本的方程一一元一次方程,为进一步学习方程打下了基础\\
方程是含有未知数的等式,解方程就是求出方程中的未知数,解方程的过程是使方程形式逐步化简,最终得出未知数的值(如x=a(已知数)).在此过程中,化归的思想方法起了重要作用,西等式的性质及运算律是化归的根据\\
利用方程解决实际间题,应认真分析其中的数量关系,关键是要找出相等关系,由此设未知数、列方程,从把实际问题转化为数学问题;然后通过解方程获得数学结论;最后用数学结论解释实际河题,这是一个“实际问题数学问题一实际问题”的过程,今后,我们将在不断经历这一过程中,提高应用数学解决实际问题的能力\\
请你带着下面的问题,复习一下全章的内容吧\\
1.举例说明方程与等式的关系以及一元一次方程的特征\\
2.回顾解一元一次方程的一般步骤,结合例子体会:解关于x的方程,就是运用等式性质和运算律,根据方程的具体特点,通过灵活变形将方程逐步化简,最后变为x=a(已知数)而得解\\
3.你能举例说明用字母表示数、列含字母的算式和列方程的区别和联系吗?\\
4.用方程解决实际问题,是把实际问则转化为数学问题(方程)的过程,其中要特别关注从实际问题中分析出关键性的相等关系,你能举例对此加以解释吗?\\
5.请收集一些重要问题(例如气侯、节能、经济等)的有关数据,经过分析后编出可以利用一元一次方程解决的问题,并正确地表述问题及其解决过程\\

\end{framed}
\end{document}