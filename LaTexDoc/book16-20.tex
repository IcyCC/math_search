\documentclass{article}
\usepackage[utf8]{ctex}

\begin{document}

\maketitle

\section{1.3 有理数的加减法}

\subsection*{1.3.1 有理数的加法}

在小学,我们学过正数及$0$的加法运算引入负数后,怎样进行加法运算呢?实际问题中,有时也会遇到与负数有关的加法运算例如,在本章引言中, 把收入记作正数,支出记作负数,在求“结余”时,需要计算$8.5+(-4.5)$,$4+(-5.2)$等.\\


小学学过的加法是正数与正数相加、正数与$0$相加。引入负数后,加法有哪几种情况?\\

引人负数后,除已有的正数与正数相加、正数与$0$相加外,还有负数与负;数相加、负数与正数相加、负数与$0$相加等。下面借助具体情境和数轴来讨论有理数的加法,\\
看下面的问题\\
一个物体作左右方向的运动,我们规定向左为负,向右为正,向有运动$5$m记作$5$m,向左运动$5$m记作$-5$m.\\


思考\\

如果物体先向右运动$5$, 再向右运动$3$m, 那么两次运动的最后结果是什么?可以用怎样的算式表示?\\

两次运动后物体从起点向右运动了$8$m.写成算式就是$5+3=8$.\\

将物体的运动起点放在原点,则这个算式可用数轴表示为图1.3-1.\\



思考\\

如果物体先向左运动$5$m,再向左运动$3$m,那么两次运动的最后结果是什么?可以用怎样的算式表示? \\

两次运动后物体从起点向左运动了8m.写成算式就是$(-5)+(-3)=- 8$.\\

这个运算也可以用数轴表示,其中假设原点O为运动起点(图1.3-2).\\

从算武①②可以看出:符号相同的两个数相加,结果的符号不变,绝对值相加\\

探究\\

(1)如果物体先向左运动$3$m,再向右运动$5$m,那么两次运动的最后结果怎样?如何用算式表示?\\ 
(2)如果物体先向右运动$3$m,再向左运动$5$m.那么两次运动的最后结果怎样?如何用算式表示?\\

(1)结果是物体从起点向右运动了$2$m.写成算式就是$(-3)+5=2$.\\
(2)结果是物体从起点向左运动了2m.写成算式就是$3+(-5)=-2$.\\

从算式③①可以看出:符号相反的两个数相加,结果的符号与绝对值较大的加数的符号相同,并用较大的绝对值减去较小的绝对值.\\

探究\\

如果物体先向右运动$5$m.再向左运动$5$m,那么两次运动的最后结果如何?\\

结果是仍在起点处,写成算式就是$5+(-5)=0$.\\

算式⑤表明,互为相反数的两个数相加,结果为0.\\

如果物体第$1$s向右(或左)运动$5$m,第$2$s原地不动,$2$s后物体从起点向右(或左)运动了$5$m.写成算式就是\\

$5+0=5$ (或$(-5)+0= -5$).  ⑥\\

思考\\

从算式⑥可以得出什么结论?\\

从算式①~⑥可知,有理数加法运算中,既要考虑符号,又要考虑绝对值你能从这些算式中归纳出有理数加法的运算法则吗?\\


	\property{有理数加法}
\begin{propertory}\\
有理数加法法则:\\

1.同号两数相加,取相同的符号,并把绝对值相加\\

2.绝对值不相等的异号两数相加,取绝对值较大的加数的符号,并用较大的绝对值减去较小的绝对值互为相反数的两个数相加得0.\\

3.一个数同0相加,仍得这个数\\
\end{propertory}\\

\begin{example}
    
    例1计算:\\
    
    (1) $(-3)+(-9)$;\\
    
    (2) $(-4.7)+3.9$.\\
    
    解: \\
    (1) $(-3)+(-9)=-(3+9)=-12$;\\

    (2) $(-4.7)+3.9=-(4.7-3.9)=-0.8$.\\

\end{example}
    
\begin{exercise}
    
    练习\\
    
    1.用算式表示下面的结果:\\
    
    (1)温度由一4C上升7C;\\
    (2)收入7元,又支出5元,\\
    
    2.口算:\\

    (1) $-4-6$;\\
    
    (2) $4+(- 6)$;\\
    
    (3) $(-4)+6$;\\

    (4) $(-4)+4$;\\
    
    (5) $(-4)+14$;\\
    
    (6) $(-14)+4$;\\
    
    (7) $6+(-6)$;\\

    (8) $0+(-6)$.\\
    
    
    3.计算:\\

    (1) $15+(-22)$;\\
    
    (2) $(-13)+(-8)$;\\
    
    (3) $(-0.9)+1.5$;\\

    (4) $\frac{1}{2}+(-\frac{2}{3})$\\

    4.请你用生活实例解释$5+(-3)=2$, $(-5)+(-3)=-8$的意义,\\
    
\end{exercise}

    我们以前学过加法交换律、结合律,在有理数的加法中它们还适用吗?\\
    
    探究\\
    
    计算\\
    
    $30+(-20)$, $(-20)+30$.\\
    
    两次所得的和相同吗?换几个加数再试一试。\\
    
    
    从上述计算中,你能得出什么结论? \\
    
	\property{有理数加法交换律}
\begin{property}
    
有理数的加法中,两个数相加,交换加数的位置,和不变\\

加法交换律: $a+b=b+a$.\\

\end{property}
    
    探究\\
    
    计算\\

$[8+(-5)]+(-4)$, $8+[(-5)+(-4)]$.两次所得的和相同吗?换几个加数再试一试.\\

从上述计算中,你能得出什么结论? \\

	\property{有理数加法结合律}
\begin{property}
    
有理数的加法中,三个数相加,先把前两个数相加,或者先把后两个数相加,和不变\\
加法结合律: $(a+b)+c=a+(b+c)$.\\

\end{property}

\begin{example}
    
    例2
    
    计算 $16+(-25)+24+(-35)$.\\
    
    解: $16+(-25)+24+(-35)=16+24+[(-25)+(-35)]=40+(-60)=-20$.\\
    
    例2中是怎祥使计算简化的?根据是什么?\\
    
    利用加法交换律、结合律,可以使运算简化认识运算律对于理解运算有很重要的意义。\\
    
    例3\\
    
    10袋小麦称后记录如图1.3-3所示(单位: kg). 10袋小麦一共多少千克?如果每袋小麦以90 kg为标准。10 袋小麦总计超过多少千克或不足多少千克?\\
    
    解法1:\\
    
    先计算10袋小麦-共多少千克:\\

    $91+91+91.5+89+91.2+91.3+88.7+88.8+91.8+91.1=905.4$.\\
    
    再计算总计超过多少千克:\\
    
    $905.4-90\times10=5.4$.\\

    解法2:\\
    每袋小麦超过90 kg的千克数记作正数,不足的千克数记作负数.10袋小麦对应的数分别为$+1$, $+1$, $+1.5$, $-1$, $+1.2$, $+1.3$, $-1.3$, $- 1.2$, $+1.8$, $+1.1$.\\

    $1+1+1.5+(-1)+1.2+1.3+(-1.3)+(-1.2)+1.8+1.1=[1+(-1)]+[1.2+(-1.2)]+[1.3+(- 1.3)]+(1+1.5+1.8+1.1)=5.4$.\\
    
    $90X10+5.4=905.4$.\\

    答: 10袋小麦一共905.4kg, 总计超过5.4kg.\\
    
\end{example}

\begin{exercise}
    
    练习\\

    1.计算:\\

    (1) $23+(-17)+6+(-22)$;\\
    
    (2) $(-2)+3+1+(-3)+2+(-4)$.\\

    2.计算:\\

    (1) $1+(-\frac{1}{2})+\frac{1}{3}+(-\frac{1}{6})$\\
    
    (2) $3\frac{1}{4}+(-2\frac{3}{5})+5\frac{3}{4}-8\frac{2}{5}$.\\
    
\end{exercise}
    
    
    
    \end{document}