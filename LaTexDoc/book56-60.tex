% ----------------------------------------------------------------
% Report Class (This is a LaTeX2e document)  *********************
% ----------------------------------------------------------------
\documentclass[UTF8]{report}
\usepackage{CTEX}
\usepackage[english]{babel}
\usepackage{amsmath,amsthm}
\usepackage{amsfonts}
\usepackage{exercise}
% THEOREMS -------------------------------------------------------
\newtheorem{thm}{Theorem}[chapter]
\newtheorem{cor}[thm]{Corollary}
\newtheorem{lem}[thm]{Lemma}
\newtheorem{prop}[thm]{Proposition}
\theoremstyle{definition}
\newtheorem{defn}[thm]{Definition}
\theoremstyle{remark}
\newtheorem{rem}[thm]{Remark}
% ----------------------------------------------------------------
\begin{document}
\paragraph{数字1和字母X的对话\\1:数学是由数产生的,数才是数学王国的真正主人。\\X:我是字母,我虽然不是具体的数,但是可以表示各种各样的数,我可以代表你1,也可以代表其他数,\\1:由我们数组成的式子有确切的大小例如,人们一见到1+2就知道是1与2的和,即3. 你们字母能这样做吗?\\X:有我们字母的式子进行运算和推理时具有一般性。例如,x+y可以表示任何两个数的和,包括1+2. x+y=y+工能表示任何两数相加时都可以交换顺序,即加法交换律.\\1:人们解决实际问题时,必须根据已知的具体数字进行计算,而字母有什么用呢?X;在解决实际问题时,用字母表示来知数,把字母列入算式(方程),能更方便地表示数量关系。数和字母一起运算会使问题的解法更简单.\\1:数是人们经过长期实践创造出来的,并建立了专门研究数及其运算的学科一 算术,你们字母行吗?\\X:随着实践的发展,人们发现只有算术还不够,用字母表示数会起到更大的作用,于是产生了代数这门学科。它首要研究的就是用字母表示的式子的运算法则和方程的解法,从算术发展到代数是数学的一大进步。\\1:算术几乎是伴随着人类社会活动的产生和发展而逐渐形成的,它有着非常悠久的历史,代数有怎样的历史呢?\\X:代数的历史可以追意到约3800年首的古埃及和古巴比伦时期,那时就有了代数的萌芽。到了公元3世纪,代数在希腊获得显著的发展,其代表人物是被誉为代数学鼻祖的丢香图。之后,印度的代数发展很快。同时,阿拉位地区的代数研究取得很大进展,其中著名的代表作是数学家阿尔花拉子米于公元820年左右发表的《代数学》(这本书的拉丁文译本取名为《对消与还原》).这本书第一次提出了这门学科的名称.}
\section*{整式的加减}
\paragraph{我们来看本章引言中的问题(2)。\\在西宁到拉萨路段,如果列车通过冻土路段的时间是$t$h,那么它通过非冻土地段的时间是2.1$2.1t$h,这段铁路的全长(单位:km)是\\$100t+120\times2.1t$,\\ 即\\$100t+252t$,\\ 类比数的运算,我们应如何为化简式子$100t+252t$ 呢?}

\paragraph{在(1)中,我们知道,根据分配律可得\\$100\times2+252\times2=(100+252)\times2=352\times2$.\\$100\times(-2)+252\times(-2)=(100+252)\times(-2)=352\times(-2)$.\\ 在(2)中,式子$100t+252t$表示100t与252t两项的和。式子\\$100t+252t$\\和\\$100\times(-2)+252\times(-2)$\\有相同的结构,并且字母$t$代表的是一个因(乘)数,因此根据分配律也应该有\\$100\times{t}+252\times{t}=(100+252)\times{t}=352t$.}

\paragraph{
对于上面的(1)(2)(3),利用分配律可得
\\$100t-252t=(100-252)t=-152t$,
\\$3x^{2}+2x^{2}=(3+2)x^{2}=5x^{2}$,
\\$3ab^{2}-4ab^{2}(3-4)ab^{2}=-ab^{2}$,
\\  观察(1)中的多项式的项$100t$和$-252t$, 它们含有相同的字母$t$,并且$t$的指数都是1; (2)中的多项式的项$3x^{2}$ 和$2x^{2}$, 含有相同的字母$x$, 并且$x$的指数都是2; (3)中的多项式的项$3ab^{2}$与$-4ab^{2}$,都含有字母a, b,并且a的指数都是1次,b 的指数都是2次。像$100t$与$-252t$, $3x^{2}$与$2x^{2}$,$3ab$与$-4ab^{2}$这样,所含字母相同,并且相同字母的指数也相同的项叫做同类项,几个常数项也是同类项
\\  因为多项式中的字母表示的是数,所以我们也可以运用交换律、结合律、分配律把多项式中的同类项进行合并,例如,
\\$4x^{2}+2x+7+3x-8x^{2}-2=4x^{2}-8x^{2}+2x=3x+7-2=(4x^{2}-8x2^{2}+(2x+3x)+(7-2)=(4-8)x^{2}+(2+3)x+(7-2)=-4x^{2}+(2+3)x+(7-2)=-4x^{2}+5x+5$.
}
\defn{把多项式中的同类项合并成一项,叫做合并同类项。合并同类项后,所得项的系数是合并前各同类项的系数的和,且字母连同它的指数不变。}
%\rem{注意分配律的使用:\\"100t-252t=[1000+(-252t)]t=(100-252)t".}



\begin{exercise}合并下列各式的同类项:
 \\(1)$xy^{2}-\frac{1}{5}xy^{2}$;
 \\(2)$-3x^{2}y+2x^{2}y-3xy^{2}-2xy$ ;
 \\(3)$4a^{2}+3b^{2}+2ab-4a^{2}-4b^{2}$.
 \end{exercise}
 \begin{Answer}
 解:(1)$xy^{2}-\frac{1}{5}xy^{2}=(1-\frac{1}{5})xy^{2}=\frac{4}{5}xy^{2}$;
 \\(2)$-3x^{2}y+2x^{2}y-3xy^{2}-2xy$
 \\$=(-3+2)x^{2}y+(3-2)xy^{2}$
 \\$=-x^{2}y+xy^{2}$;
 \\(3)$4a^{2}+3b^{2}+2ab-4a^{2}-4b^{2}$
 \\$=(4a^{2}-4a^{2})+(3-4)b^{2}+2ab$
 \\$-b^{2}+2ab$.
 \end{Answer}

\begin{exercise}
(1) 求多项式$2x^{2}-5x+x^{2}+4x-3x^{2}-2$的值,其中$x=\frac{1}{2}$;
 \\(2)求多项式$3a-abc-\frac{1}{3}c^{2}-3a+\frac{1}{3}c^{2}$的値,其中$a=-\frac{1}{6}$,b=2,c=-3.
\end{exercise}
 \begin{Answer}
分析:在求多项式的值时,可以先将多项式中的同类项合并,然后再求值,这样做往往可以简化计算.
\\ 解: (1) $2x^{2}-5x+x^{2}+4x-3x^{2}-2$
      \\$=(2+1-3)x^{2}+(-5+4)x-2$
      \\$=-x-2$.
\\ 当$x=\frac{1}{2}$时,原式$=-\frac{1}{2}-2=-\frac{2}{5}$.
(2)$3a+abc-\frac{1}{3}c^{2}-3a+\frac{1}{3}c^{2}$
\\$=(3-3)a+abc+(-\frac{1}{3})c^{2}$
\\$=abc$.
\\ 当$a=-\frac{1}{6}$,b=2,c=-3时,原式=$-\frac{1}{6}\times2\times(-3)=1$.
\end{Answer}

\begin{exercise}
1.计算:
      \\(1)$12x-20x$;  (2)$x+7x-5x$;
      \\(3)$-5a+0.3a-2.7a$  (4)$\frac{1}{3}y-\frac{2}{3}y+2y$;
      \\(5)$-6ab-ba+8ab$;  (6)$10y^{2}-0.5y^{2}$;
\end{exercise}

\begin{exercise}2.求下列各式的値:
\\(1) $3a+2b-5a-b$,其中a=-2,b=l;\\(2) $3x-4x^{2}+7-3x+2x^{2}+1$,其中x=-3.
\\3. (1) x的4倍与x的5倍的和是多少?
\\(2) x的3倍比x的一半大多少?
\\4.如图,大圆的半径是R,小圆的半径是大圆面积的$\frac{4}{9}$,求阴影部分的面积。
\end{exercise}

\begin{exercise}
(1) 水库水位第一天连续下降了$a$h,每小时平均下降2cm;第二天连续上升了$a$h,每小时平均上升0.5cm,这两天水位总的变化情况如何?
\\(2)某商店原有5袋大米,每袋大米为$x$kg.上午卖出3袋, 下午又购进同样包装的大米4袋。进货后这个商店有大米多少千克?
\\ 解: (1) 把下降的水位变化量记为负,上升的水位变化量记为正.第一天水位的变化量是$-2a$ cm,第二天水位的变化量是$0.5a$ cm,
\\ 两天水位的总变化量(单位: cm)是
\\  $-2a+0.5a=(-2+0.5)a= -1.5a$.
\\ 这两天水位总的变化情况为下降了$1.5a$ cm.
\\ (2)把进货的数量记为正,售出的数量记为负进货后这个商店共有大米(单位: kg)
\\  $5x-3x+4x=(5-3+4)x=6x$.
\end{exercise}
\end{document}
% ----------------------------------------------------------------
