\documentclass{article}
\usepackage[utf8]{ctex}
\usepackage{underscore}

\begin{document}

\begin{exercise}
7.一天早晨的气温是 ${-7}^\circ$C,中午上升了11$^\circ$C,半夜又下降了9$^\circ$C,半夜的气温是多少摄氏度?

8.食品店一周中各天的盈亏状况如下(盈余为正):
	132元,${-12.5}$元,${-10.5}$元,127元,${-87}$元,136.5元,98元.
一周总的盈亏状况如何?

9.有8筐白菜,以每筐25kg为准,超过的千克数记作正数,不足的千克数记作负数,称后的记录如下:

1.5,-3,2,-0.5,1,-2,-2,-2.5,

这8筐白菜一共多少千克?

10.某地一周内每天的最高气温与最低气温记录如下表,哪天的昼夜温差最大?哪天的昼夜温差最小?

\begin{tabular}{|c|c|c|c|c|c|c|c|}
     \hline
     星期&一&二&三&四&五&六&日\\
     \hline
     最高气温&10$^\circ$C&12$^\circ$C&11$^\circ$C&9$^\circ$C&7$^\circ$C&5$^\circ$C&7$^\circ$C\\
     \hline
     最低气温&2$^\circ$C&1$^\circ$C&0$^\circ$C&-1$^\circ$C&-4$^\circ$C&-5$^\circ$C&-5$^\circ$\\
     \hline
\end{tabular}

拓广探索

11.填空:

    (1)__$+11=27$          (2)$7+$__$=4$

    (3)$(-9)+$__$=9$       (4)$12+$__$=0$
    
    (5)$(-8)+$__$=-15$     (6)__$+(-13)=-6$
    
12.计算下列各式的值:

    $(-2)+(-2)$                 $(-2)+(-2)+(-2)$
    
    $(-2)+(-2)+(-2)+(-2)$       $(-2)+(-2)+(-2)+(-2)+(-2)$
    
猜想下列各式的值:

    ${(-2)}\times 2$, ${(-2)}\times 3$, ${(-2)}\times 2$, ${(-2)}\times 5$
    
你能进一步猜出负数乘正数的法则吗?

13.一种股票第一天的最高价比开盘价高0.3元,最低价比开盘价低0.2元;

第二天的最高价比开盘价高0.2元,最低价比开盘价低0.2元;

第三天的最高价等于开盘价,最低价比开盘价低0.13元,

计算每天最高价和最低价的差,以及这些差的平均值。

股票交易是市场经济中的一种金融活动,它可以促进投资和资金流通。
\end{exercise}

\begin{article}

\begin{center}
    中国人最先使用负数
\end{center}

中国人很早就开始使用负数。著名的中国古代数学著作《九章算术》的“方程”一章,
在世界数学史上首次正式引入负数及其加减乘除运算法则,并给出名为“正负数”的算法,
魏晋时期的数学家刘徽在其著作《九章算术注》中用不同颜色的算筹(小棍形状的计数工具)
分别表示正数和负数(红色为正,黑色为负),

“正负术”是正负数加减法则,其中有一段话是“同名相除,异名相异,正无入负之,负无入正之。”
你知道它的意思吗?其实它就是减法法则,以现代算式为例,可以将这段话解释如下:

“同名相除”,即同号两数相减时,括号前为被减数的符号,括号内为被减数的绝对值加减数的绝对值,例如

${(-5)}-{(+3)}={+(5-3)}$,

${(-5)}-{(-3)}={-(5-3)}$,

"异名相异",即异号两数相减时,括号前为被减数的符号,括号内为被减数的绝对值加减数的绝对值,例如

${(+5)}-{(-3)}={+(5+3)}$

${(-5)}-{(-3)}={-(5+3)}$

"正无入负之,负无入正之",即0减正得负,0减负得正,例如

$0-{(+3)}={(-3)}$

$0-{(-3)}={(+3)}$

史料证明:追溯到两千多年前,中国人已经开始使用负数,并应用到生产和生活中,例如,在古代商业活动中,以收入为正,支出为负;以盈余为正,亏损为负,在古代农业活动中,
以增产为正,减产为负,中国人使用负数在世界上是首创。

\end{article}

\begin{article}

\section{1.4 有理数的乘除法}

\subsection{1.4.1 有理数的乘法}

我们已经熟悉正数及0的乘法运算。与加法类似,引入负数后,将出现${3}\times{(-3)}$,
${(-3)}\times{3}$,${(-3)\times{3}}$这样的乘法,该怎样进行这一类的运算呢?

\begin{example}

\subsubsection{思考}

观察下面的乘法算式,你能发现什么规律吗?

$3\times3=9$,

$3\times2=6$,

$3\times1=3$,

$3\times0=0$,

\end{example}

可以发现,上述算式有如下规律:随着后一乘数的逐次递减1,积逐次递减3.

要使这个规律在引入负数后仍然成立,那么应有:

$3\times(-1)=-3$

$3\times(-2)=$__

$3\times(-3)=$__

\begin{example}

观察下面的算式,你又能发现什么规律?

$3\times3=9$

$2\times3=6$

$1\times3=3$

$0\times3=0$

\end{example}

可以发现,上述算式有如下规律,随着前一乘数逐次递减1,积逐次递减3.

要使上述规律在引入负数后仍然成立,那么你认为下面的空格应该填写什么数?

$(-1)\times3=$__,

$(-2)\times3=$__,

$(-3)\times3=$__,

从符号和绝对值两个角度观察上述所有算式,可以归纳如下:

正数乘正数,积为正数;正数乘负数,积为负数;负数乘正数,积也是负数,
积的绝对值等于各乘数绝对值的积.

\begin{example}

利用上面归纳的结论计算下面的公式,你发现有什么规律?

$(-3)\times3=$__,

$(-3)\times2=$__,

$(-3)\times1=$__,

$(-3)\times0=$__,

\end{example}

可以发现,上述算式有如下规律:随着后一乘数逐次递减1,积逐次增加3.

按照上述规律,下面的空格可以各填什么数?

$(-3)\times(-1)=$__

$(-3)\times(-2)=$__

$(-3)\times(-3)=$__

可以归纳出如下结论:

负数乘负数,积为正数,乘积的绝对值等于各乘数绝对值的积

一般地,我们有有理数乘法法则:

两数相乘,同号得正,异号得负,并把绝对值相乘。

任何数与0相乘,都得0.

例如,  $(-5)\times(-3)$, ............................同号两数相乘

        $(-5)\times(-3)=+( )$,................................得正
        
所以    $5\times3=15$

又如    $(-7)\times4$,................................____________

        $(-7)\times4=-()$,............................____________
        
        $7\times4=28$,................................____________
        
所以    $(-7)\times4=$................................____________        

也就是:有理数相乘,可以先确定积的符号,再确定积的绝对值。

\begin{example}

例1 计算:

(1)$(-3)\times9$;

(2)$8\times(-1)$;

(3)$(-1/2)\times(-2)$

解:(1)$(-3)\times9=-27$

(2)$8\times(-1)=8$

(3)$(-\frac{1}{2})\times(-2)=1$

例1(3)中,$(-\frac{1}{2})\times(-2)=1$,我们说$(-\frac{1}{2})$和(-2)互为倒数,一般地,
在有理数中仍然有:

乘积是1的两个数互为倒数。

例2 用正负数去表示气温的变化量,上升为正,下降为负,登山队攀登一座山峰,每登高1km气温的变化量为
$(-6)^\circ$C,攀登3km后,气温有什么变化?

解:$(-6)\times3=-18$

答:气温下降$(18)^\circ$C

\end{example}

\begin{exercise}

\subsubsection{练习}

1.计算:

(1)$6\times(-9)$

(2)$(-4)\times6$

(3)$(-6)\times(-1)$

(4)$(-6)\times0$

(5)$\frac{2}{3}\times(-\frac{9}{4})$

(6)$(-\frac{1}{3})\times\frac{1}{4}$

2.商店降价销售某种商品,每件降5元,售出60件后,与按原价销售同样数量的商品相比,销售额有什么变化?

3.写出下列各数的倒数

1,-1,$\frac{1}{3}$,$-\frac{1}{3}$,5,-5,$\frac{2}{3}$,$-\frac{2}{3}$

\end{exercise}

多个有理数相乘,可以把它们按顺序依次相乘

\subsubsection{思考}

观察下列各式,它们的积是正的还是负的?

$2\times3\times4\times(-5)$

$2\times3\times(-4)\times(-5)$

$2\times(-3)\times(-4)\times(-5)$

$(-2)\times(-3)\times(-4)\times(-5)$

几个不是0的数相乘,积的符号与负因数的个数之间有什么关系?

\subsubsection{归纳}

几个不是0的数相乘,负因数的个数是偶数时,积是正数;负因数的个数是奇数时,积是负数。

\begin{example}

例3 计算:

(1)$(-3)\times\frac{5}{6}\times(-\frac{9}{5})\times(-\frac{1}{4})$;

(2)$(-5)\times6\times(-\frac{4}{5})\times\frac{1}{4}$,

解:(1)$(-3)\times\frac{5}{6}\times(-\frac{9}{5})\times(-\frac{1}{4})$ = 
$-3\times\frac{5}{6}\times\frac{9}{5}\times(-\frac{1}{4})$=$-\frac{9}{8}$

(2)$(-5)\times6\times(-\frac{4}{5})\times\frac{1}{4}$=$5\times6\times\frac{4}{5}\times\frac{1}{4}$=6

多个不是0的数相乘,先做哪一步,再做哪一步?

\end{example}

\subsubsection{思考}

你能看出下式的结果吗?如果能,请说明理由:

$7.8\times(-8.1)\times0\times(-19.6)$

几个数相乘,如果其中有因数为0,那么积等于0.

\end{article}

\end{document}
