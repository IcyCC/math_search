\documentclass{article}
\usepackage[utf8]{ctex}

\begin{document}

\maketitle

\subsubsection*{1.5.3 近似数}

\begin{article}

先看一个例子。对于参加同一个会议的人数,有两个报道。一个报道说:“会议秘书处宣布,参加今天会议的有513人。”这里数字513确切地反映了实际人数,它是一个准确数。另一报道说:“约有五百人参加了今天的会议。”五百这个数只是接近实际人数,但与实际人数还有差别,它是一个\begin{definition}近似数\end{definition}(approximate number)。

在许多情况下,很难取得准确数,或者不必使用准确数,而可以使用近似数。例如,宇宙现在的年龄约为200亿年,长江长约6300km,圆周率\pi约为3.13,这里的数都是近似数。

近似数与准确数的接近程度,可以用精确度表示。例如,前面的五百是精确到百位数的近似数,它与准确数513的误差为13。

按四舍五入对圆周率\pi取近似数时,有

$\pi \approx 3$(精确到个位),

$\pi \approx 3.1$(精确到0.1,或叫做精确到十分位),

$\pi \approx 3.14$(精确到0.01,或叫做精确到百分位),

$\pi \approx 3.142$(精确到\underline{\hbox to 20mm},或叫做精确到\underline{\hbox to 20mm}),

$\pi \approx 3.1416$(精确到\underline{\hbox to 20mm}),或叫做精确到\underline{\hbox to 20mm})。

······

\begin{example}
例6 按括号内的要求,用四舍五入法对下列各数取近似数:

(1) 0.0158(精确到0.001);

(2) 304.35(精确到个位);

(3) 1.804(精确到0.1);

(4) 1.804(精确到0.01)。

解:(1) $0.0158 \approx 0.016$;

(2) $304.35 \approx 304$;

(3) $1.804 \approx 1.8$;

(4) $1.804 \approx 1.80$;

\end{example}

\begin{exeicise}

用四舍五入法对下列各数取近似数:

(1) 0.00356(精确到万分位);    (2) 61.235(精确到个位);

(3) 1.8935(精确到0.001);      (4) 0.0571(精确到0.1)。

习题1.5

复习巩固

1. 计算:

    (1) ${(-3)}^3$;    (2) ${(-2)}^4$;
    
    (3) ${(-1.7)}^2$;  (4) ${-\frac{4}{3}}^3$;
    
    (5) $-{(-2)}^3$;   (6) ${(-2)}^2 \times {(-3)}^2$。

2. 用计算器计算:

    (1) ${(-12)}^8$;    (2) $103^4$;
    
    (3) $7.12^3$;       (4) ${-45.7}^3$;

3. 计算:

    (1) ${(-1)}^100 \times 5 + {(-2)}^4 \div 4$;    
    
    (2) ${(-3)}^2-3\times {(-\frac{1}{3})}^4$;
    
    (3) $\frac{7}{6} \times (\frac{1}{6}-\frac{1}{3})\times \frac{3}{14}\div \frac{3}{5}$;  
    
    (4) ${(-10)}^3+[{(-4)}^2-(1-3^2)\times 2]$;
    
    (5) $-2^3\div \frac{4}{9} \times {(-\frac{2}{3})}^2$;   
    
    (6) $4+{(-2)}^3\times 5-(-0.28)\div 4$。

4. 用科学计数法表示下列各数:

    (1) 235 000 000;       (2) 188 520 000;
    
    (3) 701 000 000 000;   (4) -38 000 000;

5. 下列用科学计数法表示的数,原来各是什么树?
    
    $3\times 10^2$,$1.3\times 10^3$,$8.05\times 10^6$,$2.004\times 10^5$,$-1.96\times 10^4$。

6. 用四舍五入法对下列各数取近似值:

    (1) 0.00356(精确到0.0001);    
    
    (2) 566.1235(精确到个位);
    
    (3) 3.8963(精确到0.01);  
    
    (4) 0.0571(精确到千分位);

综合运用

7. 平方等于9的数是几?立方等于27的数是几?

8. 一个长方体的长、宽都是a,高是b,它的体积和表面积怎样计算?当a=2cm,b=5cm时,它的体积和表面积是多少?

9. 地球绕太阳公转的速度是$1.1\times 10^5$km/h,声音在空气中的传播速度是340m/s,试比较两个速度的大小。

10. 一天有$8.64 \times 10^4$s,一年按365天计算,一年有多少秒(用科学计数法表示)?

拓展探索

11. (1) 计算$0.1^2$,$1^2$,$10^2$,$100^2$,观察这些结果,底数的小数点向左(右)移动一位时,平凡数小数点有什么移动规律?

    (2) 计算$0.1^3$,$1^3$,$10^3$,$100^3$,观察这些结果,底数的小数点向左(右)移动一位时,平凡数小数点有什么移动规律?

    (3) 计算$0.1^4$,$1^4$,$10^4$,$100^4$,观察这些结果,底数的小数点向左(右)移动一位时,平凡数小数点有什么移动规律?

12。 计算${(-2)}^2$,$2^2$,${(-2)}^3$,$2^3$。联系这类具体的数的乘方,你认为当a<0时下列各式是否成立?

    (1) $a^2>0$;       (2) $a^2={(-a)}^2$;
    
    (3) $a^2=-a^2$;    (4) $a^3=-a^3$;

\end{exeicise}

数学活动

活动1

帮助家庭记录一个月(或一周)的生活收支账目,收入记为正数,支出记为负数,计算当月(周)的总收入、总支出、总节余以及每日平均支出等数据。

妥善保存账目,作为日后家庭理财的参考资料。

活动2

熟悉你所用的计算器有关有理数运算的功能和操作方法,对于包含乘方、乘除与加减运算的算式,考虑怎样操作计算器最简便,实习这样的操作,并与同学进行交流。

活动3 

收集现实生活中你认为非常大的数据的实例,体会科学记数法和近似数等在实际中的应用。

小结

一、本章知识结构图

\includegraphics[width=3in]{51-55小结.png}

二、回顾与思考

本章我们在小学学习的基础上,进一步认识了负数,使数的范围扩充到有理数,引入负数不仅可以表示具有相反意义的量,而且还扩展了减法运算的范围,由此,类似于x+2=1的方程就可以解了。

我们知道,有理数是整数与分数的统称,由于整数可以看成是分母为1的分数,因此有理数可以写成$\frac{p}{q}$(p, q是整数, $q\neq 0$)的形式;另一方面,形如$\frac{p}{q}$(p,q是整数,$q\neq0$)的数都是有理数。所以,有理数可用$\frac{p}{q}$(p,q是整数,$q \neq 0$)表示。

本章我们研究了有理数的加、减、乘、除和乘方运算。实际上,与负数有关的运算,我们都借助绝对值,将它们转化为正数之间的运算。数轴不仅能直观表示数,而且还能帮助我们理解数的运算。在运算的过程中,数形结合、转化是很重要的思想方法。

我们从具体数的加法和乘法中,扫纳出了交换律、结合律和分配律等运算律,运算律不仅能给数的运算带来方便,而且还是今后研究代数问题(如解方程、不等式等)的基础。

    请你带着下面的问题,复习一下全章的内容吧。
    1.你能举出一些实例,说明正数、负数在表示相反意义的量时的作用吗?
    2.你能用一个图表示有理数的分类吗?引入负数后,减法中哪些原来不能进行的运算可以进行了?
    3.怎样用数轴表示有理数?数轴与普通直线有什么不同?怎祥利用数轴

\end{article}
\end{document}
