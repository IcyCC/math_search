
\documentclass{article}
\usepackage[utf8]{ctex}
\usepackage{graphicx}
\usepackage{textcomp}
\begin{document}

\maketitle



\begin{article}
解释一个数的绝对值和相反数?
4.有理数的加法与减法、乘法与除法各有什么关系?有理数的混合运算
都能转化为加法与乘法运算吗?
5.有理数有哪些运算律?结合例子说明运算律在有理数运算中的作用。
\begin{exeicise}
复习巩固
1.在数轴上表示下列各数,并按从小到大顺序用``<''号把这些数连接起来:
   3.5,-3.5,0,2,-2,-1.6,$-\frac{1]{3}$,0.5
2.已知x是整数,并且-3<x<4,在数轴上表示x可能取的所有数值。
3.设a=-2,b=$-\frac{2}{3}$,c=5.5,分别写出a,b,c的绝对值、相反数和倒数。
4.互为倒数的两数的和是多少?互为倒数的两数的积是多少?
5.计算:
(1)$-150+250$
(2)$-15+(-23)$
(3)$-5-65$
(4)$-26-(-15)$
(5)$-6\times(-16)$
(6)$-\frac{1}{4}\times27$
(7)$8\div(-16)$
(8)$-25\div(-\frac{2}{3})$
(9)$(-0.02)\times(-20)\times(-5)\times4.5$
(10)$(-6.5)\times(-2)\div(-\frac{1}{3})\div(-5)$
(11)$6+(-\frac{1}{5})-2-(-1.5)$
(12)$-66\times4-(-2.5)\div(-0.1)$
(13)${(-2)}^2\times5-{(-2)}^2\div4$
(14)$-(3-5)+{3}^2\times(1-3)$
6.用四舍五入法,按括号内的要求,对下列各数取近似值:
(1)245.635(精确到0.1);
(2)175.65(精确到个位);
(3)12.004(精确到百分位);
(4)6.5378(精确到0.01);
7.把下列各数用科学记数法表示:
(1)100000000;
(2)-4500000;
(3)692400000000;
8.计算:
(1)$-2-\left(-3)\right$;
(2)$\left-2-(-3)\right$;
9.下列各数是10名学生的数学考试成绩:
82,83,78,66,95,75,56,93,82,81,
先估计他们的平均成绩,然后在此基础上计算平均成绩,由此检验你的估算能力。
10.a,b是有理数,它们在数轴上的对应点的位置如图所示,把a,-a,b,-b按照从小到大顺序排列,正确的是(),
\begin{figure}
  \centering
  % Requires \usepackage{graphicx}
  \includegraphics[width=5cm]{1}\\
  \caption{}\label{}
\end{figure}
(A)-b<-a<a<b      (B)-a<-b<a<b
(C)-b<a<-a<b      (D)-b<b<-a<a
11.某文具店在一周的销售中,盈亏情况如下表(盈余为正,单位:元):
\begin{figure}
  \centering
  % Requires \usepackage{graphicx}
  \includegraphics[width=13cm]{2}\\
  \caption{}\label{}
\end{figure}
表中星期六的盈亏数据被墨水涂污了,请你算出星期六的盈亏数,并说明星期六是盈还是亏?盈亏是多少?
12.当温度每升高1$^\circ$C\par时,某种金属丝长0.002mm,反之,当温度下降1$^\circ$C\par时,金属丝缩短0.002mm,把这种金属丝加热到60$^\circ$C\par,再使它冷却降温到5$^\circ$C\par,金属丝的长度经历了怎样的变化?最后的长度比原长伸长多少?
13.一年中地球与太阳之间距离随时间变化而变化,一个天文单位是地球与太阳治安平均距离,即1.4960亿km,试用科学记数法表示一个天文单位是多少千米
14.结合具体的数的运算,归纳有关特例,然后比较下列数的大小:
(1)小于1的正数a,a的平方,a的立方;
(2)大于-1的负数b,b的平方,b的立方
15.结合具体的数,通过特例进行归纳,然后判断下列说法的对错,认为对,说明理由,认为错,举出反例
(1)任何数都不等于它的相反数;
(2)互为相反数的两个数的同一偶数次方相等;
(3)如果a大于b,那么a的倒数小于b的倒数
17.用计算器计算下列各式,并将结果写在横线上:
$1\times1&=_____;         $11\times11=_____$;
$111\times111=_____$;      $1111\times1111=______$;
(1)你会发现什么?
(2)不用计算器,你能直接写出$11111111\times11111111$的结果吗?

\end{exeicise}

\section*{第二章  整式的加减}
青藏铁路线上,在格尔木到拉萨之间有一段很长的冻土地段.列车在冻土地段、非冻土地段的行驶速度分别是100 km/h和120 km/h,请根据这些数据回答下列问题:

      (1)列车在冻土地段行驶时,2 h行驶的路程是多少? 3h呢? I h呢?

      (2)在西宁到拉萨路段,列车通过非冻土地段所需时间是通过冻土地段所需时间的2.1倍,如果通过冻土地段需要Ih,能用含1的式子表示这段铁路的全长吗?

      (3)在格尔木到拉萨路段,列车通过冻土地段比通过非冻土地段多用0.5h,如果通过冻土地段需要uh,则这段铁路的全长可以怎样表示?冻土地段与非冻土地段相差多少千米?

      在小学,我们学过用字母表示数,知道可以用字母或含有字母的式子表示数和数量关系,这样的式子在数学中有重要作用.在本章,我们将学习整式及其加减运算,进一步认识含有字母的数学式子,并为一元一次方程等后续内容的学习打下基础
\subsection*{2.1 整式}
我们来看本章引言中的问题(1).

      列车在冻土地段的行驶速度是100 km/h,根据速度、时间和路程之间的关系

      路程=速度X时间,列车2 h行驶的路程(单位: km)是
      $100\times2= 200$,
      3 h行驶的路程(单位: km)是
      $100\times3= 300$,
      ih行驶的路程(单位: km)是
      $100\times1= 100$,  ①
      在式子①中,我们用字母1表示时间,用含有字母1的式子100表示路程

      下面,我们再来看几个用含有字母的式子表示数量关系的问题,
\begin{example}
例1 (1) 苹果原价是每千克p元,按8折优惠出售,用式子表示现价:
     (2)某产品前年的产量是n件,去年的产量是前年产量的m倍,用式子表示去年的产量; ;

      (3) -个长方体包装盒的长和宽都是a cm,高是h cm,用式子表示它的体积:

      (4)用式子表示数n的相反数.
      解: (1)现价是每千克0.8p元;(2)去年的产量是mn 件;

      (3)由长方体的体积=长X宽X高,得这个长方体包装盒的体积是$a\cdot a\cdot h$ cm^{3}, 即$a^{2}h$ cm^{2};

      (4)数n的相反数是-n.
例2 (1) 一条河的水流速度是2.5 km/h,船在静水中的速度是v km/h,用式子表示船在这条河中顺水行驶和逆水行驶时的速度;
(2)买一个篮球需要工元,买一个排球需要y元,买一个足球需要z元,用式子表示买3个篮球、5个排球、2个足球共需要的钱数:
(3)如图2. 1-1 (图中长度单位: cm),用式子表示三角尺的面积:
\begin{figure}
  \centering
  % Requires \usepackage{graphicx}
  \includegraphics[width=10cm]{3}\\
  \caption{}\label{}
\end{figure}
(4)图2.1-2 是一所住宅的建筑平面图(图中长度单位: m),用式子表示这所住宅的建筑面积。

      分析: (1) 船在河流中行驶时,船的速度需要分两种情况讨论:顺水行驶时,船的速度=加在静水中的速度+水流速度;逆水行驶时,船的速度=船在静水中的速度一水流速度.

      解: (1)船在这条河中顺水行驶的速度是$(v+2.5)$ km/h, 逆水行驶的速度是$(v+ 2.5)$ km/h.

      (2)买3个篮球、5个排球、2个足球共需要$(3x+5y+2z)$ 元

      (3)三角尺的面积等于三角形的面积减去圆的面积,根据图中的数据,得三角形的面积是$\frac{1}{2}ab$ cm^{2},圆的面积是$\pi r^{2}$icm^{2}.因此三角尺的面积(单位: cm^{2}) 是$\frac{1}{2}ab-\pi r^{2}$.

      (4)住宅的建筑面积等于四个长方形面积的和根据图中标出的尺寸,可得这所住宅的建筑面积(单位: m^{2})是$x^{2}+2x+18$.

      从上:面的例子可以看出,用字母表示数,字母和数样可以参与运算,可以用式子把数量关系简明地表示出来,
\end{example}
\begin{exeicise}
1. 某种商品母袋4.8元,在一个月内的销售量是m袋,用式子表示在这个月内销

      售这种商品的收入,

      2.國柱体的底面半径、高分别是r,h.用式子表示筒柱体的体积。

      3.有两片棉田,一片有m ${km}^{2}$ (公顷。1${km}^{2}$=10${km}^{2}$),平均每公顷产棉花akg; 另一片有${km}^{2}$,平均每公项产棉花6 kg,用式于表示两片棉四上棉花的总产量,

      4.在一个大正方形铁片中挖去一个小正方形铁片,大正方形的边长是a mm.小

      正方形的边长是b mm。用式子表示剩余部分的面机
\end{exeicise}
这些式子都是数或字母的积,像这样的式子叫做单项式(monomial). 单独的一个数或-个字母也是单项式
单项式中的数字因数叫做这个单项式的系数(cofficient). 例如,单项式$100t$, $a^{2}h$, -n 的系数分别是100, 1, -1.单项式表示数与字母相乘时,通常把数写在前面。

      一个单项式中, 所有字母的指数的和叫做这个单项式的次数(degree of a monomial).例如,在单项式$100t$中,字母1的指数是1;$100t$的次数是1;在单项式ah中,字母a与h的指数的和是3, $a^{2}h的次数是3.
\begin{example}

例3 用单项式填空,并指出他们的系数和次数;
(1)没包书有12册,n包书有__册;
(2)底边长为a cm,高为h cm的三角形面积是__${cm}^{2}$;
(3)棱长为a cm的正方体体积是___${cm}^3$;
(4)一台电视机原价 b元,现在原价九折出售,这台电视机售价是__元;
\end{example}
\end{article}
\end{document}
