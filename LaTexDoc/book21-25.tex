\documentclass[UTF8]{article}
\usepackage{CTEX}
\usepackage{textcomp}
\usepackage{graphicx}
\newtheorem{exercise}{ }
\newtheorem{text}{ }
\newtheorem{propertory}{ }
\newtheorem{example}{ }
\newtheorem{concept}{ }
\begin{document}
实验与探究
填幻方
有人建议向火星发射如图1的困案。它叫做幻方,其中9个格中的点数分别是1, 2.3,4.5.6, 7, 8.9.每一横行、每一整列以及两条斜对角线上的点数的和都是15.如果火星上有智能生物,那么他们可以从这种“数学语言”了解到地球上也有智能生物(人).

你能将4, -3, -2. -1, 0.1, 2, 3, 4这9个数分别填入图2的幻方的9个空格中,使得处于同一横行、同一整列、同一斜对角线上的3个数相加都得0吗?
你是将0填入中央的格中吗?与同学交流一下,你们镇这个幻方的方法相同吗?

\subsection*{1.3.2 有理数的减法}
实际问题中有时还要涉及有理数的减法。例如,本章引言中,北京某天的气温是一-3${\textcelsius }$~3${\textcelsius}$,这天的温差(最高气温减最低气温,单位:${\textcelsius }$)就是3-(-3).这里遇到正数与负数的减法
减法是加法的逆运算,计算3-(- 3), 就是:要求出一个数x,使得x与-3相加得3.因为6与-3相加得3,所以工应该是6,即
3-(-3)=6.①另方面,我们知道
3+(+3)=6,②由①②,有
3-(- 3)=3+(+3).③
探究

从③式能看出减3相当于加哪个数吗?把3换成0,-1,-5,用上面的方法考虑

0-(-3),(-1)-(-3),(-5)-(-3).这些数减一3的结果与它们加十3的结果相

同吗?  换几个数再试一试

计算

9 -8, 9+(-8); 15-7,15+(-7).从中又有什么新发现?

可以发现,有理数的减法可以转化为加法来进行有理数减法法则:

减去一个数。等于加这个数的相反数。有理数减法法则也可以表示成

a-b=a+(-b).

\begin{example}
例4计算:

(1) (-3)-(-5);  (2) 0-7;

(3) 7.2-(- 4.8);  (4)$-3\frac{1}{2}-5\frac{1}{4}$;
解: 
(1) (--3)-(- 5)=(- -3)+5=2;(2) 0-7=0+(-7)= -7;

(3) 7.2-(-4.8)=7.2+4.8=12;

(4)($-3\frac{1}{2}$)-$5\frac{1}{4}$=($-3\frac{1}{2}$)+($-5\frac{1}{4}$) =($-8\frac{3}{4}$)
\end{example}

思考

在小学,只有当a大于或等于b时,我们才会做a-b (例如2- 1.1-1).现在,当a小于b时,你会做a-b (例如1-2,(-1)-1)吗?

一般地,较小的数减去较大的数,所得的差的符号是什么?
\begin{exercise}
1. 计算:
(1)$6-9$
(2)$(+4)-(-7)$
(3)$(-5)-(-8)$
(4)$0-(-5)$
(5)$(-2.5)-5.9$
(6)$1.9-(-0.6)$

2.计算:
(1)比$2^\circ$C \par 低$8^\circ$C \par 的温度;
(2)比$-3^\circ$C \par 低$6^\circ$C \par 的温度。

\end{exercise}

下面我们研究怎样进行有理数的加减混合运算.

\begin{example}
例5 计算(- 20)+(+3)-(-5)- (+7).
分析:这个算式中有加法,也有减法。可以根据有理数减法法则,把它改写为
	(- 20)+(+3)+<+5>+<-7).
使问题转化为几个有理数的加法,
解:  (一20)+<+3)-<-5>-(+7>
	= ( 20>+<+3)+(+5)+(- 7)
	=[(- -20)+(-7)]+[(+5)+(+3>]=(-27)+<+8)= -19.
\end{example}
这里使用了哪些运算律?

归纳
引入相反数后,加减混合运算可以统一为加法运算。
	$a+b-c=a+b+(-c).$	

算式
	$(- 20)+<+3>+<+5>+<-7)$
是$一20, 3, 5, -7$这四个数的和,为书写简单,可以省略算式中的括号和加号,把它写为

	$一20+3+5-7.$
这个算式可以读作“负20、正3、正5、负7的和",或读作“负20加3

加5减7”.例5的运算过程也可以简单地写为
	$( 20)+(+3)-(-5)- (+7)
	= - 20+3+5-7
	=-20-7+3+5
	=-27+8
	=- 19.$

探究
在数轴上,点A, B分别表示数a, b.利用有理数减法,分别计算下列情况下点A, B之间的距离:
	$a=2, b=6; a=0. b=6; a=2. b=-6; a=-2, b=-6.$
你能发现点A. B之间的距离与数a. b之间的关系吗?

\begin{exercise}
练习
计算:
(1) $1-4+3-0.5$;
(2) $-2.4+3.5- 4.6+3.5$;
(3) $(-7)-(+5)+(-4)-(-10)$;
(4) $\frac{3}{4} - \frac{7}{2}+(-\frac{1}{6})-(-\frac{2}{3})-1$


习题1.3
复习巩固
1.计算:
	(1) $(-10)+(+6)$;
	(2) $(+12)+(-4)$;
	(3) $(-5)+(-7)$;
	(4) $(+6)+(-9)$;
	(5) $(-0.9)+(-2.7)$;
	(6) $\frac{2}{5}+(-\frac{3}{5})$;
	(7) $(-\frac{1}{3})+(\frac{2}{5})$;
	(8)	$(-3\frac{1}{4})+(-1\frac{1}{12})$.
2.计算:
(1) $(-8)+10+2+(-1)$;
(2) $5+(-6)+3+9+(-4)+(-7)$;

(3) $(-0.8)+1.2+(-0.7)+(-2.1)+0.8+3.5$;

(4) $\frac{1}{2}+(-\frac{2}{3})+\frac{4}{5}+(-\frac{1}{2})+(-\frac{1}{3})$.

3. 计算

(1) $(-8)-8$;

(2) $(-8)-(-8)$;

(3) $8-(-8)$;

(4) $8-8$;

(5) $0-6$;

(6) $0-(-6)$;

(7) $16-47$;

(8) $28-(-74)$;

(9) $(-3.8)-(+7)$;

(10) $(-5.9)-(-6.1)$.

4. 计算

(1) $(+\frac{2}{5}-(-\frac{3}{5}))$;

(2) $(-\frac{2}{5})-(-\frac{3}{5})$;

(3) $\frac{1}{2}-\frac{1}{3}$;

(4) $(-\frac{1}{2})-\frac{1}{3}$;

(5) $-\frac{2}{3}-(-\frac{1}{6})$;

(6) $0-(-\frac{3}{4})$;

(7) $(-2)-(+\frac{2}{3})$;

(8) $-16\frac{3}{4}-(-10\frac{1}{4})-(+1\frac{1}{2})$.

5. 计算

(1) $-4.2+5.7-8.4+10$

(2) $-\frac{1}{4}+\frac{5}{6}+\frac{2}{3}-\frac{1}{2}$;

(3) $12-(-18)+(-7)-15$;

(4) $4.7-(-8.9)-7.5+(-6)$;

(5) $(-4\frac{7}{8})-(-5\frac{1}{2})+(-4\frac{1}{4})-(+3\frac{1}{8})$;

(6) -\frac{2}{3}+\left|0-5\frac{1}{6}\right|+\left|-4\frac{5}{6}\right|+(-9\frac{1}{3}).

综合运用

6. 如图,陆上最高处是珠穆朗玛峰的峰顶($8844.43$m), 最低处位于亚洲西部名为死海的湖($-415$m),两处高度相差多少?

\end{exercise}

\end{document}