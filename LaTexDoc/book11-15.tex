\documentclass[UTF8]{article}
\usepackage{CTEX}
\usepackage{color}
\usepackage{indentfirst}
\setlength{\parindent}{2em}
\usepackage{txfonts}
\usepackage{textcomp}
\usepackage{graphicx}
\begin{document}
	\subsection*{ 1.2.4 绝对值}\\
	\indent两辆汽车从同一处O出发,分别向东、西方向行驶10km,到达A,B两处(图1.2-6).它们的行驶路线相同吗?它们的行驶路程相同?\\
	\begin{figure}[ht]
		\centering
		\includegraphics[scale=0.5]{1.png}
	\end{figure}

	\definition{绝对值}
	\begin{definition}
	\indent一般地,数轴上表示数a的点与原点的距离叫做数a的\textcolor{blue}{绝对值}(absolute value),记作$\vert a \vert$.例如,图1.2-6中A,B两点分别表示10和-10,它们与原点的距离10个单位长度,所以10和-10的绝对值都是10,即\\
	\end{definition}
	
	\indent$\vert 10 \vert$ = 10,$\vert -10 \vert$ = 10.\\
	\indent显然$\vert 0 \vert$ = 0.\\
	
	\property{绝对值}
	\begin{propertory}
	\indent由绝对值的定义可知:\\
	\indent\textcolor{blue}{一个正数的绝对值是它本身;一个负数的绝对值是它的相反数;0的绝对值是0}。即\\
	\indent(1) 如果a$>$0,那么$\vert a \vert$ = a;\\
	\indent(2)如果a=0,那么$\vert a \vert$ = 0;\\
	\indent(3)如果a$<$0,那么$\vert a \vert$ = -a;\\
	\indent 这里的数a可以使正数、负数和0.\\
	\end{propertory}
	
	\textcolor{blue}{\heiti 练习}\\
	1.写出下列个数的绝对值:\\
	$$6,-8,-3.9,\frac{5}{2},-\frac{2}{11},100,0$$
	2.判断下列说法是否正确:
	\indent(1)符号相反的数互为相反数;\\
	\indent(2)一个数的绝对值越大,表示它的点在数轴上越靠右;\\
	\indent(3)一个数的绝对值越大,表示它的点在数轴上离原点越远;\\
	\indent(4)当$a\neq0$时,$\vert a \vert$总是大于0.\\
	3.判断下列各式是否正确:\\
	(1)$\vert 5 \vert$=$\vert --5 \vert$;(2)-$\vert 5 \vert$=$\vert -5 \vert$;(3)-5=$\vert -5 \vert$.
	\indent我们已知两个正数(或0)之间怎样比较大小,例如\\
	$$0<1,1<2,2<3···.$$
	\indent任意两个有理数(例如-4和-3,-2和0,-1和1)怎样比较大小呢?\\
	\textcolor{blue}{\heiti 思考}\\
	\indent图1.2-7给出了未来一周中每天的最高气温和最低气温,其中最低气温是多少?最高气温呢?你能将这七天中每天的最低气温按从低到高的顺序排列吗?\\
	\begin{figure}[ht]
		\centering
		\includegraphics[scale=0.5]{2.png}
	\end{figure}
	\indent这七天中每天的最低气温按从低到高排列为\\
	$$-4,-3,-2,-1,0,1,2$$.
	\indent按照这个顺序排列的温度,在温度计上所对应的点事从下到上的,按照这个顺序把这些书表示在数轴上,表示它们的各点的顺序是从左到右的(图1.2-8).
	\begin{figure}[ht]
		\centering
		\includegraphics[scale=1]{3.png}
	\end{figure}
	\indent数学中规定:在数轴上表示有理数,它们从左到右的顺序,就是从小到大的顺序,即左边的数小于右边的数.
	\indent由这个规定可知
	$$-6<-5,-5<-4,-4<-3,-2<0,-1<1,···.$$
	\textcolor{blue}{思考}
	\indent对于正数、0和负数这三类数,它们之间有什么大小关系?两个负数之间如何比较大小?前面最低气温由低到高的排列与你的结论一致吗?
	一般地,\\
	\textcolor{blue}{(1)正数大于0,0大于负数,正数大于负数;}\\
	\textcolor{blue}{(2)两个负数,绝对值大的反而小。}\\
	例如,10,0-1,1-1,-1-2.\\
	\textcolor{blue}{例} 比较下列各对数的大小:\\
	(1) -(-1)和-(+2); \\
	(2) $-\frac{8}{21}$ 和 $-\frac{3}{7}$\\     
	(3)-(-0.3)和 $\vert -\frac{1}{3} \vert$.\\
	\textcolor{blue}{解}:(1)先化简,-(-1) = 1,-(+2) = -2.\\
	因为正数大于负数,所以$1> -2$,即\\
	$$-(-1)>-(+2).$$
	(2)这是两个负数比较大小,先求它们的绝对值.\\
	$\vert -\frac{8}{21} \vert$ = $\frac{8}{12}$,$\vert -\frac{3}{7} \vert $= $\frac{3}{7}=\frac{9}{21}$.\\
	因为 $\frac{8}{21}$ $<$ $\frac{9}{21},$\\
	即	$\vert -\frac{8}{21} \vert $ $<$ $\vert -\frac{3}{7} \vert$,\\
	所以 $-\frac{8}{21}  $ $<$ $ -\frac{3}{7}.$\\
	(3)先化简,-(-0.3) = 0.3,$\vert -\frac{1}{3} \vert$ = $\frac{1}{3}.$\\
	因为 0.3  $<$ $\frac{1}{3}.$\\
	所以 -(-0.3)  $<$ $ \vert -\frac{1}{3} \vert.$\\
	\indent 异号两数比较大小,要考虑它们的正负:同号两数比较大小,要考虑它们的绝对值。\\
	\textcolor{blue}{\heiti 练习}\\
	比较下列各对数的大小:\\
	(1) 3和-5;                        (2) -3和-5;\\
	(3) -2.5和$-\vert -2.25 \vert$;     (4) $-\frac{3}{5}$和$-\frac{3}{4}$.\\
	\textcolor{blue}{\heiti 习题1.2}
	{\heiti 复习巩固}
	1.把下面的有理数填写在相应的大括号里(将各数用逗号分开):
	$$ 15,-\frac{3}{8},0,0.15,-30,-12.8,-\frac{22}{5},+20,-60 $$.
	正数:${              }$ 负数:$ {               } $
	2.在数轴上表示下列各数:
	$$-5,+3,-3.5,0,\frac{2}{3},-\frac{3}{2},0.75 $$.
	3.在数轴上,点A表示-3,从点A出发,沿着数轴移动4个单位长到达点B,则点B表示的数是多少?
	4.写出下列各数的相反数,并将这些数连同它们的相反数在数轴有上表示出来:
	$$ -4,+2,-1.5,0,\frac{1}{3},-\frac{9}{4} $$.
	5.写出下列各数的绝对值:
	$$-125,+23,-3.5,0,\frac{2}{3},-\frac{3}{2},-0.05$$.
	上面的数中哪个数的绝对值最大?那个数的绝对值最小?
	6.将下列各数按从小到大的顺序排列,并用“$<$”号连接:
	$$-0.25,+2.3,-0.15,0,-\frac{2}{3},-\frac{3}{2},-\frac{1}{2},0.05$$.
	{\heiti 综合应用}\\
	7.下面是我国几个城市某年一月份的平均气温,把它们按从高到低的顺序排列。\\
	北京  武汉  广州  哈尔滨   南京\\
	-4.6{\textcelsius } 3.8{\textcelsius } 13.1{\textcelsius }  -19.4{\textcelsius } 2.4{\textcelsius } \\
	8.如图,检测5个球球,其中超过标准的克数记为正数,不足的克数记为负数,从轻重的角度,哪个球最接近标准?\\
		\begin{figure}[ht]
		\centering
		\includegraphics[scale=0.5]{4.png}
	\end{figure}
	9.某年我国人均水资源比上年的增幅是-5.6\%.后续三年各年比上年的增幅分别是-4.0\%,13.0\%,-9.6\%,这些增幅中哪个是最小?增幅是负数说明什么?\\
	10.在数轴上,表示哪个数的点与表示-2和4的点的距离相等?\\
	{\heiti 拓广探索}\\
	11.(1) -1和0之间还有负数吗?$-\frac{1}{2}$与0之间呢?如有,请举例。\\
	(2)-3和-1之间有负整数吗?-2和2之间有哪些整数?\\
	(3)有比-1大的负整数吗?\\
	(4)写出3个小于-100并且大于-103的数。\\
	12.如果$\vert x \vert$ = 2,那么x一定是2吗?如果$\vert x \vert$ = 0,那么x等于几?如果x = -x,那么x等于几?\\
\end{document}