\documentclass{article}
\usepackage[utf8]{ctex}

\begin{document}

\maketitle

\begin{example}
解:设安排x人先做4h.\newline
根据先后两个时段的工作量之和应等于总工作量,列出方程\newline
$\frac{4x}{40}+\frac{8(x+2)}{40}=1$.\newline
解方程,得\newline
$4x+8(x+2)=40$,\newline
$4x+8x+16=40$,\newline
$12x=24$,\newline
$x=2$.\newline
答:应安排2人先做4h.\newline

这类问题中常常把工作总量看做1,并利用“工作量=人均效率$\times$人数$\times$时间”的关系考虑问题.\newline

归纳\newline

用一元一次方程解决实际问题的基本过程如下:\newline

这一过程一般包括设、列、解、答等步骤,即设未知数,列方程,解方程,检验所得结果,确定答案. 正确分析问题中的相等关系是列方程的基础.\newline
\end{example}
\begin{exercise}
练习\newline

1. 一套仪器由一个A部件和三个B部件构成. 用${1m^2}$钢材可做40个A部件或240个B部件. 现要用$6{m^2}$钢材制作这种仪器,应用多少钢材做A部件,多少钢材做B部件,恰好配成这种仪器多少套?\newline

2. 一条地下管线由甲工程队单独铺设需要12天,由乙工程队单独铺设需要24天. 如果由这两个工程队从两端同时施工,要多少天可以铺好这条管线?\newline
\end{exercise}
\begin{example}

有些实际问题中,数量关系比较隐蔽,需要仔细分析才能列出方程. 下面我们进一步探究几个这样的问题. \newline

探究1\newline

销售中的盈亏\newline

一商店在某一时间以每件60元的价格卖出两件衣服,其中一件盈利$25\%$,另一件亏损$25\%$,卖这两件衣服总的是盈利还是亏损,或是不盈不亏?\newline

分析:两件衣服共卖了$120(=60\times2)$元,是盈是亏要着这家商店买进这两件衣服时花了多少钱,如果进价大于售价就亏损,反之就盈利. \newline

先大体估算盈亏,再通过准确计算检验你的判断. \newline

假设一件商品的进价是40元,如果卖出后盈利$25\%$,那么商品利润是$40X25\%$元;如果卖出后亏损$25\%$,商品利润是$40\times(-25\%)$元. \newline

本问题中,设盈利$25\%$的那件衣服的进价是x元,它的商品利润就是0.25x元. 根据进价与利润的和等于售价,列出方程\newline

$x+0.25x=60$.\newline

由此得\newline

$x=48$.\newline

类似地,可以设另一件衣服的进价为y元,它的商品利润是-0.25y元,列出方程\newline

$y-0.25y=60$.\newline

由此得\newline

$y=80$.\newline

两件衣服的进价是$x+y=128$元,而两件衣服的售价是$60+60=120$元,进价大于售价由此可知卖这两件衣服总共亏损8元. \newline

列、解方程后得出的结论与你先前估算一致吗?通过对本题的探究,你对方程在实际问题中的应用有什么新的认识?\newline

探究2\newline

球赛积分表问题\newline

某次篮球联赛积分榜\newline

(1)用式子表示总积分与胜、负场数之间的数量关系;\newline

(2)某队的胜场总积分能等于它的负场总积分吗?\newline

分析:观察积分榜,从最下面一行数据可以看出:负一场积一分. \newline

设胜一场积x分,从表中其他任何一行可以列方程,求出x的值. 例如,从第一行得方程\newline

$10x+1\times4=24$.\newline

由此得\newline

$x=2$.\newline

用积分榜中其他行可以验证,得出结论:负一场积1分,胜一场积2分.\newline

(1)如果一个队胜m场,则负$(14-m)$场,胜场积分为2m,负场积分为$14-m$,总积分为\newline

$2m+(14-m)=m+14$. \newline

(2)设一个队胜了x场,则负了$(14-x)$场. 如果这个队的胜场总积分等于负场总积分,则得方程\newline

$2x=14-x$. \newline

由此得\newline

$x=\frac{14}{3}$. \newline

想一想,x表示什么量?它可以是分数吗?山此你能得出什么结论?

解决实际问题时,要考虑得到的结果是不是符合实际。x(所胜的场数)的值必须是整数,所以$x=\frac{14}{3}$不符合实际,由此可以判定没有哪个队的胜场总积分等于负场总积分。

这个问题说明:利用方程不仅能求具体数值,而且可以进行推理判断。

上面的问题说明,用方程解决实际问题时,不仅要注意解方程的过程是否正确,还要检验方程的解是否符合问题的实际意义。



考虑下列问题:

探究3

电话计费问题

下表中有两种移动电话计费方式。

(1) 设一个月内用移动电话主叫为t min(t是正整数).根据上表,列表说明:当t在不同时间范围内取值时,按方式一和方式二如何计费。

(2) 观察你的列表,你能从中发现如何根据主叫时间选择省钱的计费方式吗?通过计算验证你的看法。

分析: (1)由上表可知,计费与主叫时间相关,计费时首先要看主叫是否超过限定时间。因此,考虑t的取值时,两个主叫限定时间150 min和350 min是不同时间范围的划分点。

当t在不同时间范国内取值时,方式一和方式二的计费如下页表:

想一想,x表示什么量?它可以是分数吗?山此你能得出什么结论?

解决实际问题时,要考虑得到的结果是不是符合实际。x(所胜的场数)的值必须是整数,所以$x=\frac{14}{3}$不符合实际,由此可以判定没有哪个队的胜场总积分等于负场总积分。

这个问题说明:利用方程不仅能求具体数值,而且可以进行推理判断。

上面的问题说明,用方程解决实际问题时,不仅要注意解方程的过程是否正确,还要检验方程的解是否符合问题的实际意义。

考虑下列问题:

探究3

电话计费问题

下表中有两种移动电话计费方式。

(1) 设一个月内用移动电话主叫为t min(t是正整数).根据上表,列表说明:当t在不同时间范围内取值时,按方式一和方式二如何计费。

(2) 观察你的列表,你能从中发现如何根据主叫时间选择省钱的计费方式吗?通过计算验证你的看法。

分析: (1)由上表可知,计费与主叫时间相关,计费时首先要看主叫是否超过限定时间。因此,考虑t的取值时,两个主叫限定时间150 min和350 min是不同时间范围的划分点。

当t在不同时间范国内取值时,方式一和方式二的计费如下页表:
\end{example}

\end{document}
