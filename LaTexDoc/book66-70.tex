\documentclass{article}
\usepackage[utf8]{inputenc}
\usepackage[utf8]{ctex}
\usepackage{graphicx}

\date{}
\begin{document}
\title{}
\maketitle
\subsection*{ 3.2解一元一次方程(一)——合并同类项与移项}

\begin{article}
    \indent 我们已经知道,直接利用等式的基本性质可以解简单的方程,本节重点讨论如何利用“合并同类项”和“移项”解一元一次方程.
    
    \indent 约公元820年,中亚细亚数学家阿尔-花拉子米写了一本代数书,重点论述怎样解方程.这本书的拉丁译本取名为《对消与还原》.“对消”与“还原”是什么意思呢?我们先讨论下面的内容,然后再回答这个问题.
    
    \begin{example}
    \indent  问题1 某校三年共购买计算机140台,去年购买数量是前年的2倍,今年购买数量又是去年的2倍前年这个学校购买了多少台计算机?
    
    \indent 设前年购买计算机x台。可以表示出:去年购买计算机2x台,今年购买计算机4.x台.根据问题中的相等关系:前年购买量+去年购买量+今年购买量=140台,列得方程
    \begin{center}$x+2x+4x=140.$\end{center}
    
    \indent 把含有x的项合并同类项,得
    \begin{center}$7x= 140.$\end{center}

     \indent 下面的框图表示了解这个方程的流程:
     \begin{figure}
         \centering
         \includegraphics{p1.PNG}
     \end{figure}
     
     \indent 由下可知,前年这个学校购买了20台计算机
     \newpage 
      例1解下列方程:      
     
      (1)$ 2x-\frac{5}{2}x=6-8$;  (2) $7x-2.5x+3x-1.5x=-15\times4-6\times3$.
      
      解: (1) 合并同类项,得  
      
      $-\frac{1}{2}x=-2$
      
      系数化为1,得      
      $x=4$.
      
      (2)合并同类项,得     
     $6x=-78$.
      系数化为1,得      
      $x=-13$. 
      
      例2有一列数.按一定规律排列成1, -3, 9,一27, 81. -243. .其中某三个相邻数的和是一-1 701.这三个数各是多少?
      
      分析:从符号和绝对值两方面观察,可发现这列数的排列规律:后面的数是它前面的数与- 3的乘积,如果三个相邻数中的第1个记为x,则后两个数分别是-3x, 9x.      
      
      解:设所求三个数分别是x,- 3x, 9x.由三个数的和是一1 701.得 
      
      $x-3x+9r=-1701$. 
      
      合并同类项,得 
      
      $7x=-1 701$. 
      
      系数化为1,得
      
      $x=-243.$      
      
      所以      
      $-3x=729$ $9x=-2187$.
      答:这三个数是-243,729,- -2 187.
      \end{example}
      
      \begin{exercise}
      (1) $5x-2x=9$  (2)$\frac{x}{2}+\frac{3x}{2}=7$
      
      (3)$-3x+0.5x=10$ (4)$7x-4.5x=2.5\times3-5$.
      
      2.某工厂的产值连续增长,去年是前年的1.5倍,今年是去年的2倍。这三年的总产值为550万元,前年的产值是多少?
      \end{exercise}
      
      \begin{example}
      问题2把一些图书分给某班学生阅读,如果每人分3本,则剩余20本;如果每人分4本,则还缺25本.这个班有多少学生?
      
      设这个班有x名学生
      
      每人分3本,共分出3x本,加上剩余的20本,这批书共(3x+20)本
      
      每人分4本,需要4x:本,减去缺的25本,这批书共(4x-25)本
      
      这批书的总数是一个定值,表示它的两个式子应相等,根据这一相等关 系列得方程
      
      $3x+20=4x-25$.
     
     思考
     方程3x+20=4x- 25的两边都有含x的项(3x与4r)和不含字母的常数项(20与25),怎样才能使它向x=a(常数)的形式转化呢?
     
     为了使方程的右边没有含的项,等号两边减4.x:为了使左边没有常数项,等号两边减20.利用等式的性质1,得
     
     $3x--4x=-25-20$.
     
     上面方程的变形,相当于把原方程左边的20变为一20移到右边,把右边的4.x变为-4x移到左边把某项从等式边移到另一边时有什么变化?
     
     像上面那样把等式边的某项变号后移到另-边,叫做\begin{defination}移项\end{defination}.
    \end{example}
    
    解方程时经常要“合并同类项”和“移项”,前面提到的古老的代数书中的“对消”和“还原”,指的就是“合并同类项"和“移项".早在一千多年前,数学家阿尔-花拉子米就已经对“合并同类项”和“移项”非常重视了,
    \begin{example}
    
    例3解下列方程
    
    (1) $3x+7=32-2x$;(2) $x-3= \frac{3}{2}x+1$.
    
    解: (1)移项,得
    
    $3x+2x=32- 7$.
    
    合并同类项,得
    
    $5x=25$.
    
    系数化为1,得
    
    $x=5$.
    
    (2)移项,得
    
    $x-\frac{3}{2}x=1+3$.
    
    合并同类项,得
    
    $-\frac{1}{2}x=4$.
    
    系数化为1
    
    $x=-8$.
    
    例4某制药厂制造批药品,如用旧工艺,则废水排量要比环保限制的最大量还多200t;如用新工艺,则废水排量比环保限制的最大量少100t.新、旧工艺的废水排量之比为2:5,两种工艺的废水排量各是多少?
    
    分析:因为新、旧工艺的废水推量之比为2:5.所以可设它们分别为2xt和5xt,再根据它们与环保限制的最大量之间的关系列方程。
    
    解:设新、旧工艺的废水排量分别为2xt和5xt。
    
    根据废水排量与环保限制最大量之间的关系,得
    
    $5x-200=2x+100$.
    
    移项,得
    
    $5x-2x=100+200$.
    
    合并同类项,得
    
    $3x=300$.
    
    系数化为1,得
    
    $x=100$.
    
    所以
    
    $2x=200$,
    
    $5x=500$.
    
    答:新、旧工艺产生的废水排量分别为200t和500t.
    \end{example}
    \begin{exercise}
   
   练习
   
    1.解下列方程:
    
    (1) $6x-7=4x-5$;
    
    (2)$\frac{1}{2}x-6=\frac{3}{4}x$.
    
    2.王芳和李丽同时采摘樱桃,王芳平均每小时采摘8kg,李丽平均每小时采摘7kg.采捕结束后王芳从她采摘的樱桃中取出0.25 kg给了李丽,这时两人的樱桃一样多.她们采摘用了多少时间?
    \end{exercise}
\end{article}
\end{document}
