\documentclass[11pt]{article}
\usepackage{amsmath}
\usepackage{CJKutf8}
\usepackage{ctex}
\newtheorem{exercise}{ }
\newtheorem{article}{ }
\newtheorem{nature}{ }
\newtheorem{tip}{ }
\begin{document}

多个有理数相乘,可以把它们按顺序依次相乘。

\begin{exercise}
思考\\
观察下列各式,它们的积是正的还是负的?\\
$2 \times 3 \times 4 \times (-5)$,\\
$2 \times 3 \times (-4) \times (-5)$,\\
$2 \times( -3) \times (-4) \times (-5)$,\\
$(-2) \times (-3) \times (-4) \times (-5)$.\\
几个不是0的数相乘,积的符号为负因数的个数之间有什么关系?\\
\end{exercise}
\begin{article}
归纳\\
	几个不是0的数相乘,负因数的个数是偶数时,积是正数;负因数的个数是奇数时,积是负数。\\
\end{article}
\begin{exercise}
例3 计算:\\
(1)
$ (-3) \times \frac{5}{6} \times (-\frac{9}{5}) \times (-\frac{1}{4}) $\\
(2)
$ (-5) \times 6 \times (-\frac{4}{5}) \times \frac{1}{4} $

解:(1)\\
$ (-3) \times \frac{5}{6} \times (-\frac{9}{5}) \times (-\frac{1}{4}) $\\
$ = -3 \times \frac{5}{6} \times \frac{9}{5} \times \frac{1}{4} = -\frac{9}{8} $\\
(2)\\
$ (-5) \times 6 \times (-\frac{4}{5}) \times \frac{1}{4} $\\
$ = 5 \times 6 \times \frac{4}{5} \times \frac{1}{4} = 6 $\\
\end{exercise}

\begin{exercise}
思考\\
你能看出下式的结果吗?如果能,请说明理由.\\
$7.8 \times (-8.1) \times 0 \times (-19.6)$,\\

几个数相乘,如果其中有因数为0,那么积等于0\\
\end{exercise}

\begin{exercise}
练习\\
1、口算:\\
(1) $(-2) \times 3 \times 4 \times (-1)$;  (2) $(-5) \times (-3) \times 4 \times (-2)$;\\
(3) $(-2) \times (-2) \times (-2) \times (-2)$;  (4) $(-3) \times (-3) \times (-3) \times (-3)$.\\

2、计算:\\
(1) $(-5) \times 8 \times (-7) \times (-0.25)$;\\
(2) $ (-\frac{5}{12}) \times \frac{8}{15} \times \frac{1}{2} \times (-\frac{2}{3}) $;\\
(3)$ (-1) \times (-\frac{5}{4}) \times \frac{8}{15} \times \frac{3}{2} \times (-\frac{2}{3}) \times 0 \times (-1) $;\\
\end{exercise}
\begin{nature}
    像前面那样规定有理数乘法法则后,就可以使交换律、结合律与分配律在有理数乘法中仍然成立.\\
    例如,\\
        $ 5\times(-6)=-30 $\\
        $ (-6)\times 5 = -30 $\\
    即,\\
        $ 5\times (-6) = (-6) \times 5 $\\
    一般地,在有理数乘法中,两个数相乘,交换因数地位置,积相等。\\
        乘法交换律:ab=ba.\\
        又如,     $ [3\times (-4)] \times (-5) = (-12) \times (-5) = 60$.\\
        $ 3\times [(-4) \times (-5)] = 3 \times 20 = 60 $.
        即\\
       $ [3\times (-4)] \times (-5) =3\times [(-4) \times (-5)]$\\
        一般地,有理数乘法中,三个数相乘,先把前两个数相乘,或者先把后两个数相乘,积相乘。\\
            乘法结合律:(a b)c = a(bc).\\
        再如,\\
          $  5 \times [3+(-7)] = 5 \times (-4) = -20$, \\
          $  5 \times 3+5\times(-7) = 15-35 = -20 $.\\
        即\\
            $ 5 \times [3+(-7)] = 5 \times 3+5\times(-7)$\\
        一般地,有理数乘法中,一个数同两个数地和相乘,等于把这个数分别同这两个数相乘,再把积相加。\\
            分配律:a(b+c) = ab+ac.\\
\end{nature}

\begin{exercise}
例 4 用两种方法计算$ (\frac{1}{4} + \frac{1}{6} -\frac{1}{2})\times 12 $.\\
解法1:$ (\frac{1}{4} + \frac{1}{6} -\frac{1}{2})\times 12 $.\\
$=(\frac{3}{12} + \frac{2}{12} -\frac{6}{12})\times 12 $\\
$=-\frac{1}{12} \times 12=-1 $.\\
解法2:$ (\frac{1}{4} + \frac{1}{6} -\frac{1}{2})\times 12 $.\\
$ =\frac{1}{4} \times 12 +\frac{1}{6} \times 12 - \frac{1}{2} \times 12 $\\
$ =3+2-6 = -1 $.
\end{exercise}

\begin{exercise}
思考\\
    比较上面两种解法,它们再运算顺序上有什么区别?解法2用了什么运算律?哪种解法运算量小?\\
\end{exercise}

\begin{exercise}
练习\\

计算:\\
(1) $(-85) \times (-25) \times (-4)$;
(2) $ (\frac{9}{10} - \frac{1}{15}) \times 30 $;\\
(3) $ (-\frac{7}{8}) \times 15 \times (-1\frac{1}{7}) $;
(4)$ (-\frac{6}{5}) \times (-\frac{2}{3}) + (-\frac{6}{5}) \times (+\frac{17}{3}) $;\\
\end{exercise}

\begin{article}
1.4.2 有理数的除法 \\
怎么计算 $ 8\div(-4)$ 呢?\\
根据除法是乘法的逆运算,就是要求一个数,使它与-4相乘得8.\\
因为  $ (-2)\times(-4) = 8 $.\\
所以  $ 8\div(-4) = -2$.  ①\\
另一方面,我们有  \\
    $  8\times (-\frac{1}{4})=-2 $ ②\\
于是有 \\
    $ 8 \div (-4) = 8\times (-\frac{1}{4}) $ ③\\
③式表明,一个数除以-4可以转化为乘 $-\frac{1}{4}$来进行,即一个数除以-4,等于乘于-4的倒数 $ -\frac{1}{4}$.\\
    与小学学过的除法一样,对于有理数除法,我们有如下法则;\\
    除以一个不等于0的数,等于乘于这个数的倒数。\\
    这个法则也可以表示成\\
        $ a\div b=a \cdot \frac{1}{b}$.(b!= 0) \\
        从有理数除法法则,很容易得出:\\
        两数相除,同号得正,异号得负,并把绝对值相除,0除以任何一个不等于0的数,都得0.\\
\end{article}


\begin{exercise}
例5 计算:\\
(1) $ (-36)\div 9 $. (2) $ (- \frac{12}{25} \div (- \frac{3}{5})) $.\\
解:(1) $ (-36)\div 9 = -(36 \div 9) = -4 $.\\

(2):$ - \frac{12}{25} \div (- \frac{3}{5})) = (- \frac{12}{25}) \times (- \frac{5}{3}) = \frac{4}{5} $.\\
\end{exercise}

\begin{exercise}
练习\\

计算:\\
(1) $(-18) \div 6 $;
(2)$(-63) \div (-7)$;
(3) $ 1 \div (-9) $; \\
(4)$ 0 \div (-8) $;
(5)$(-6.5) \div 0.13$;
(6)$(-\frac{6}{5}) \div (-\frac{2}{5})$.
\end{exercise}

\begin{exercise}
例6 化简下列分数:\\
(1) $ \frac{-12}{3} $; (2) $ \frac{-45}{-12} $.\\
解:(1) $ \frac{-12}{3} = (-12) \div 3 = -4 $;\\

(2):$ \frac{-45}{-12} = (-45) \div (-12) = 45 \div 12 = \frac{15}{4} $.\\
\end{exercise}

\begin{tip}
分数可以理解为分子除以分母\\
\end{tip}

\begin{article}
因为有理数得除法可以化为乘法,所以可以利用乘法的运算性质简化运算.乘除混合运算往往先将除法化为乘法,然后确定积的符号,最后求出结果.\\
\end{article}

\begin{exercise}
例7 计算:\\
(1) $ (-125\frac{5}{7}) \div (-5) $; (2) $ -2.5 \div \frac{5}{8} \times (-\frac{1}{4}) $.\\
解:(1) $ (-125\frac{5}{7}) \div (-5) $;\\
$ =(125+ \frac{5}{7}) \times \frac{1}{5}  $\\
$ =125 \times \frac{1}{5} + \frac{5}{7} \times \frac{1}{5} $\\
$ =25 + \frac{1}{7} $\\
$ =25\frac{1}{7} $\\

(2)$ -2.5 \div \frac{5}{8} \times (-\frac{1}{4}) $.\\
$ \frac{5}{2} \times \frac{8}{5} \times \frac{1}{4} $\\
$ =1 $
\end{exercise}


\end{document}
