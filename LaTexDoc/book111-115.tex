\documentclass{article}
\usepackage[utf8]{ctex}

\title{117-121}
\author{syj}
\date{March 2019}

\begin{document}
\maketitle

\section{Introduction}
\begin{exercise}
系吗?\newline
4.用方程解决实际问题,是把实际问题转化为数学问题(方程)的过程,其中要特别关注从实际问题中分析出关键性的相等关系,你能举例对此加以解释吗?\newline
5.请收集一些重要问题(例如气候、节能、经济等)的有关数据,经过分析后编出可以利用一元二次方程解决的问题,并正确地表述问题及其解决过程。\newline
\end{exercise}

\begin{exercise}
复习巩固\newline
1.列方程表示下列语句所表示的相等关系:\newline
(1)某地2011年9月6日的温差是10$^{\circ}$C,这天最高气温是t$^{\circ}$C,最低气温是\frac{2}{3}t$^{\circ}$C;\newline
(2)七年级学生人数为n,其中男生占45$\%$,女生有110人;\newline
(3)一种商品每件的进价为a元,售价为进价的1.1倍,现每件又降价10元,现售价为每件210元;\newline
(4)在5天中,小华共植树60棵,小明共植树x(x<60)棵,平均每天小华比小明多种2棵树。\newline
2.解下列方程:\newline
(1)$\frac{4}{3}$-8x=3-$\frac{11}{2}x$;\newline
(2)0.5x-0.7=6.5-1.3x;\newline
(3)$\frac{1}{6}(3x-6)$=$\frac{2}{5}$-3;\newline
(4)$\frac{(1-2x)}{3}$=$\frac{(3x+1)}{7}$-3\newline
3.当x为何值时,下列各组中两个式子的值相等?\newline
(1)x-$\frac{(x-1)}{3}$和7-$\frac{(x+3)}{5}$;\newline
(2)$\frac{2}{5}x$+$\frac{(x-1)}{2}$和$\frac{3(x-1)}{2}$-$\frac{8}{5}x$\newline
4.在梯形面积公式S=$\frac{1}{2}(a+b)h$中,\newline
(1)已知S=30,a=6,h=4,求b;\newline
(2)已知S=60,b=4,h=12,求a;\newline
(3)已知S=50,a=6,b=$\frac{5}{3}a$,求h.\newline
\end{exercise}

\begin{exercise}
综合运用\newline
 5.(我国古代问题)跑得快的马每天走240公里,跑得慢的马每天走150公里,慢马先走12天,快马几天可以追上慢马?\newline
 6.运动场的跑道一圈长400m.小健练习骑自行车,平均每分骑350m;小康练习跑步、平均每分跑250m两人从同一处同时反向出发。经过多少时间首次相遇?又经过多少时间再次相遇?    \newline 7.有一群鸽子和一些鴿笼,如果每个鸽笼住6只鸽子,则剩余了只鸽子无钨笼可住;如果再飞来5只鸽子,连同原来的鸽子。每个鸽笼刚好住8只鸽子。原有多少只鸽子和多少个鸽笼?   \newline  8.父亲和女儿的年龄之和是91,当父亲的年龄是女儿现在年龄的2倍的时候。女儿的年龄是父亲现在年龄的一,求女儿现在的年龄.\newline
 拓广探索   \newline
 9.某电视台组织知识竞赛,共设20道选择题,各题分值相同,每题必答,右表记录了5个参赛者的得分情况。 \newline
 (1)参赛者F得分76分,他答对了几道题? \newline
 (2)参赛者G说他得分80分,你认为可能吗?为什么? \newline
 10.一家游泳馆每年6~8月出售夏季会员证,每张会员证80元,只限本人使用,凭证购入场劵每张1元,不凭证入场券每张3元,试讨论并回答: \newline
 (1)什么情况下,购会员证与不购证付一样的钱? \newline
 (2)什么情况下,购会员证比不购证更合算? \newline
 (3)什么情况下,不购会员证比购证更合算? \newline
 11.“丰收1号”油菜籽的平均每公顷产量为2400kg,含油率为40$\%$.“丰收2号”油菜籽比“丰收1号”的平均每公顷产量提高了300kg,含油率提高了10个百分点,某村去年种植“丰收1号”油菜,今年改种“丰收2号”油菜,虽然种植面积比去年减少$43{hm^2}$,但是所产油菜籽的总产油量比去年提高2750kg,这个村去年和今年种植油菜的面积各是多少公顷? \newline
\end{exercise}

\begin{article}
第四章 几何图形初步\newline
现实世界中有形态各异、丰富多彩的图形,在小学我们学过许多关于图形的知识,在章前图的建筑中,你能找到一些熟悉的图形吗?\newline
千姿百态的图形美化了我们的生活空间,也给我们带来了很多问题:怎样画出一个五角星?怎样设计一个产品包装盒?怎样绘制一张校园布局平面图?不同的图形各有什么特点和性质?所有这些,都需要我们知道更多的图形知识。\newline
几何就是研究图形的形状、大小和位置关系的一门学科,本章我们将认识更多的几何图形,进一步探索直线、线段、角等最基本的几何图形的性质,了解他们的广泛应用,为今后进一步学习各种更复杂的几何图形及其性质做好准备。\newline
北京奥林匹克公园占地约1135,总建筑面积约200万,内有可容纳9万观众的国家体育场(鸟巢)、国家游泳中心(水立方)、国家体育馆等14个比赛场馆。\newline
\end{article}

\begin{concept}
4.1几何图形\newline
从城市宏伟的建筑到乡村简朴的住宅,从四通八达的立交桥到街头巷尾的交通标志,从古老的剪纸艺术到现代的城市雕塑,从自然界生态各异的动物到北京的申奥标志······图形世界是多姿多彩的!\newline
各种各样的物体除了有颜色、质量、材质等性质外,还具有形状(如方的、圆的等)、大小(如长度、面积、体积等)和位置关系(如相交、垂直、平行等),物体的形状、大小和位置关系是几何中研究的内容。\newline
图4.2-1(1)是一个纸盒,它有两个面是正方形,其余各面是长方形,观察纸盒的外形,从整体上看,它的形状是长方体(图4.1-2(2));看不同的侧面,得到的是正方形或长方形(图4.1-2(3));只看棱、顶点等局部,得到的是线段、点(如图4.1-2(4))等,类似地观察罐头、乒乓球的外形,可以得到圆柱、球、圆等。\newline
长方体、圆柱、球、长(正)方形、圆、线段、点等,以及小学学习过的三角形、四边形等,都是从形形色色的物体外形中得出的,它们都是几何图形(gcometric figure),几何图形是数学研究的主要对象之一。\newline
\end{concept}
\begin{article}
4.1.1立体图形与平面图形\newline
有些几何图形(如长方体、正方体、圆柱、圆锥、球等)的各部分都不在同一平面内,它们都是立体图形(solid figure)。棱柱、棱锥也是常见的立体图形。图4.1-3(2)中的金字塔则给我们以棱锥的形象,你能再找出一些棱柱、棱锥的实例吗?\newline
\end{article}
\begin{exercise}
思考\newline
图4.1-4中的实物的形状对应哪些立体图形?把相应的实物与图形用线连起来。\newline
\end{exercise}

\end{document}
